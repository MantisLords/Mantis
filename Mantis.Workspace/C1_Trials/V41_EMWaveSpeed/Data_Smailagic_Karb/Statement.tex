%! Author = User
%! Date = 23.10.2023

% Preamble
\documentclass[a4paper,10pt]{article}

% Packages
\usepackage[utf8]{inputenc}  %man kann Sonderzeiche wie ü,ö usw direkt eingeben
\usepackage{amsmath}           %macht
\usepackage{amsfonts}          %       Mathe
\usepackage{amssymb}           %              mächtiger
\usepackage{graphicx}          %erlaubt Graphiken einzubinden (.eps für dvi und ps sowie .jpg für pdf)
\usepackage[T1]{fontenc}       %Zeichenbelegung der verwendeten Schrift
\usepackage{ae}                %macht schöneres ß
\usepackage{typearea}
\usepackage{amstex}
\usepackage{siunitx}
\usepackage{mathtools}
\usepackage{hyperref}
\usepackage{hhline}	         %ermöglicht änderung des Seitenspiegels
\usepackage{caption}
\usepackage{biblatex}
\usepackage{anyfontsize}
% Document
\begin{document}
        \vspace*{\stretch{1.0}}
        \begin{center}
            \Large\textbf{Statement}\\
            \large\text{S. Smailagić and T. H. Karb}
        \end{center}
        \vspace*{\stretch{2.0}}
        
    We have submitted the paper and hereby make statements to the Peer-Review committee. 
    The preliminary peer-review was received on the $4^{\text{th}}$ of October.
    We should mention that, as suggested in the peer-review, we thought about the purpose of our paper, and shifted the focus of our paper from the simple measurement
        of impedance, velocity and damping values of some cable we have lying around, to a presentation of two methods which allow for measuring the mentioned properties.
    With that said, the overall outline of the paper is much different.
        Nonetheless, all critique points were taken into account.
        
\begin{enumerate}  
    \item We expanded the Abstract and made the suggested changes.

    \item The Figure was replaced with a more fitting one.
    Global parameters like the cable length and the pulse shape were added to the description.
    
    \item We made sure our description matches the figure, and approximated the damping in order to explain our findings
    and make them more plausible.
    
    \item We did the same as in 3) and approximated the damping of the cable, which in turn could be used for a over the thumb calculation of the 
    amplitudes.

    \item This figure was removed since it did not enhance the understanding of our described effects.
    This is largely due to the fact that anyone that reads the paper knows what a standing wave is.

    \item The description was extended with more information, and the number of significant digits was
    adjusted.

    \item Description added to Table 3.

    \item The error of the first coordinate point was initially calculated incorrectly.
    This was corrected.

    \item The critique was taken into account, and the introduction was rewritten completely.

    \item The critique was taken into account.
    The section was removed as the paper was reorganized.

    \item The critique was taken into account.
    As already mentioned in 3) and 4), we approximated the damping and with that taken into account 
    calculated the expected amplitudes.
    The data exhibited much better agreement with the new predictions.
    The deformations in signal were explained thoroughly.

    \item The critique was taken into account.
    The two potentiometer measurements were not taken into account when calculating the mean of the resistance.
    The possible explanations for the bad agreement of the values was written.

    \item The critique was taken into account. 
    The $4\,$kHz was in fact the exact value.
    A comparison with the retardation cable was made.

    \item The critique was taken into account.

    \item The critique was taken into account. 
    The value of U(0) can be seen in the table descriptions as well as in the text.

    \item The critique was taken into account.
    The pulse width was set on the LED according to the Experiment instructions (Praktikumsanleitung).
    The error was not recorded since it does not play a role in our measurement.
    The error in trying to determine where the significant raise in signal from the noise is, is much bigger.
    The process of obtaining c was detailed in the paper.

    \item The critique was taken into account.
    The factory specifications of the cable were given in the Experiment instructions (Praktikumsanleitung).
    From the impedance we could assume the cable type.
    This was also written in the paper.
    

\end{enumerate}
    
\end{document}