%! Author = User
%! Date = 13.09.2023

% Preamble
\documentclass[a4paper,10pt,twocolumn]{article}

% Packages 
\usepackage[utf8]{inputenc}  %man kann Sonderzeiche wie ü,ö usw direkt eingeben
\usepackage{amsmath}           %macht
\usepackage{amsfonts}          %       Mathe
\usepackage{amssymb}           %              mächtiger
\usepackage{graphicx}          %erlaubt Graphiken einzubinden (.eps für dvi und ps sowie .jpg für pdf)
\usepackage[T1]{fontenc}       %Zeichenbelegung der verwendeten Schrift
\usepackage{ae}                %macht schöneres ß
\usepackage{typearea}
\usepackage{amstex}
\usepackage{siunitx}
\usepackage{mathtools}
\usepackage{hyperref}
\usepackage{hhline}	         %ermöglicht änderung des Seitenspiegels
\usepackage{caption}
\usepackage{biblatex}
\usepackage{anyfontsize}
\addbibresource{bibliography.bib}
\captionsetup{font=footnotesize}


\usepackage{amsmath}
\usepackage{tikz}
\usepackage{pgfplots}

\newcommand{\alphaNoError}{(4.047 \pm 0.036)}
\newcommand{\betaNoError}{(-4.73 \pm 0.29) \cdot 10^{-3}}
\newcommand{\halfTimeNoError}{(146.5 \pm 9.1)\ s}
\newcommand{\alphaGauss}{(4.04 \pm 0.10)}
\newcommand{\betaGauss}{(-4.62 \pm 0.95) \cdot 10^{-3}}
\newcommand{\halfTimeGauss}{(150 \pm 31)\ s}
\newcommand{\alphaPoisson}{(4.05 \pm 0.10)}
\newcommand{\betaPoisson}{(-4.75 \pm 0.95) \cdot 10^{-3}}
\newcommand{\halfTimePoisson}{(146 \pm 29)\ s}
\newcommand{\symN}{\delta N}


\pagestyle{scrheadings}        %sagt Koma-Skript, dass selbstdefiniers Kopfzeilen verwendet werden
\typearea{16}                  %stellt Seitenspiegel ein
\columnsep25pt								 %definiert Breite zwischen den zwei Spalten von \twocolumns

\renewcommand{\pnumfont}{%     %ändert die Schriftart der Seitennummerierung
    \normalfont\rmfamily\slshape}  %ändert die Schriftart der Seitennummerierung 



\begin{document}
    \twocolumn[{\csname @twocolumnfalse\endcsname                %erlaubt "Abstrakt" über beide Spalten
    \titlehead{                                                  %Kopfzeile
        \begin{tabular*}{\textwidth}[]{@{\extracolsep{\fill}}lr}   %Kopfzeile
            Betreuer: Lukas Elter & \today\\                          %Kopfzeile - Betreuer
        \end{tabular*}                                             %Kopfzeile
    }
    \title{Examinig Propagation Properties of Electromagnetic Waves in Coaxial Cables}  %Titel der Versuchs
    \author{Salahudin Smailagić and Thomas Karb}                     %Namen der Studenten
    \date{}                                                         %benötigt um automatisches Datum auszuschalten
    \maketitle                                                      %erzeugt Titelseite
    \vspace{-5ex}                                                   %verringert Abstand zur Überschrift
    \begin{abstract}                                                %Beginn des Abstracts
        The widespread usage of coaxial cables makes it important to understand their properties but also their limitations,
        like the bad performace at high frequencies or the possibility of pertubations due to reflections at the end of the cable.
        In this paper, we examine two methods that allow measuring these properties by rather simple means, therefore making them affordable and easy-to-hand.
        The goal is to compare the methods and find which one is more reliable and versatile.
        
        The fist method makes use of pulses whose reflections can be looked at on the oscilloscope.
        The behaviour and shape of the reflections can be altered by different resistors at the reflective end, therefore allowing
        the impedance to be calculated $Z=\resistanceMean$.
        The method was also used to measure the speed of light in air - resulting in $c=\SpeedOfLight$.
        The second method uses standing waves inside the cable.
        The properties of the waves allow for conclusions about the propagation velocity and the damping due to ohmic resistance.
        
        If we compare the methods, we find the first one to be superior.
        Its versitility allows for making the same measurements over a much wider range of frequencies, while avoiding systematical errors which arise
        in method two.
        \\
        \\
        Measurement made: 19. September 2023\\       %Datum ändern!
        Submitted: 26. September 2023 \\
        Second submittion: 18. October 2023 %Datum ändern!
        \\
        \\
    \end{abstract}
    }]
    \section{Introduction}\label{sec:introduction}
    Transmitting signals without losses is very important in all branches of everyday life.
    Coaxial cables have been known for their wide usability in signal transmission.
    Therefore, a better understanding of their transmittance properties as well as possible limitations is of high importance.

    Coaxial cables are a type of electrical cable consisting of an inner conductor surrounded by a cylindrical conducting shield(usually copper) seperated by a dielectric. 
    In an ideal coaxial cable, the electromagnetic field carrying the signals (electromagnetic waves) only exist in the space between the conductor and the shield allowing for almost no power losses making it superior to conventional
    transmission lines.
    In this paper we want to examine, the transmittance properties of coaxial cables while emphasizing the possible problems and limitations.
    
    \section{Wave propagation in cables - The coaxial cable}
    The transmission properties of a coaxial cable are characterized by the capacitance between the two conductors per unit length $C^*$ and inductance per unit length $L^*$
    From maxwells equations the voltage and current in the cable are given by:
    \begin{align*}
        \frac{\partial^2U}{\partial x^2}=L^*C^*\frac{\partial^2U}{\partial t^2}\\
        \frac{\partial^2I}{\partial x^2}=L^*C^*\frac{\partial^2I}{\partial t^2}
    \end{align*}
    These are wave equations with the wave propagation velocity
    \begin{align}
        \label{coaxial:vel}
        v = \frac{1}{\sqrt{L^*C^*}}.
    \end{align}
    Since we know the geometry, we can calculate the inductance as well as the capacitance of the coax cable.
    \begin{align}
        C^* = 2\pi\epsilon_0\epsilon_r (\ln(\frac{d_a}{di}))^{-1}\endline
        L^* = \frac{\mu_r\mu_0}{2\pi}\ln(\frac{d_a}{d_i})
        \end{align}
    By inserting the expressions for L* and C* for a coaxial cable with relative permittivity $\epsilon_r$ and relative magnetic permeability $\mu_r$ we obtain
    \begin{align}
        \label{eq:vavePropagationVelocity}
        v = \frac{c}{\sqrt{\mu_r\epsilon_r}},
    \end{align}
    where c is the speed of light in vacuum.
    Though there is negligible ohmic resistance, coaxial cables still have an impedance of
    \begin{align}
        Z=\sqrt{\frac{L^*}{C^*}}
    \end{align}
    The reflection coefficient $\rho$, which describes the ratio between the amplitude of the incident and the reflected wave at the end of the cable, is given by
    \begin{align}
        \label{eq:coaxRef}
        \rho=\frac{R-Z}{R+Z}
    \end{align}
    where $R$ is a resistor at the end of the cable.
    \section{Wave reflection properties in the coaxial cable}
    \label{sec:reflectionProperties}
    \begin{figure}[htbp]                                 %So bindet man Graphiken ein
        \begin{center}                                       %zentriert die Graphik
            \includegraphics[width=1.3\linewidth,scale=1.5]{pictures/Construction}      %das eingentliche Einbinden; "schwarz" ist der Dateiname ohne Endung
            \caption[]{The main part of the experimentall setup consisted of the Keysight-33220A function generator used to generate $30\,$ns pulses at $5\,$V initial amplitude, and a $50\,$m long
            coaxial cable with a wave impedance of $50\,\Omega$ per factory specification.
            The two channels of a Tektronix TDS 1001B oscilloscope were connected to the two Resistors $R_{\text{Near/Far}}$ which are on the ends of the cable.
            By changing the resistance at the ends of the cable, different reflection behaviour of the pulses can be observed on the oscillator reading.
            }   %Beschriftung der Graphik
            \label{fig:Construction}                                      %das wird zu Zitiern im Text gebraucht
        \end{center}
    \end{figure}
    In order to examine the reflection properties of waves inside the cable we made use of \autoref{eq:coaxRef} which suggested, that the reflection
    coefficient $\rho$ depends on the resistance at the end of the cable.
    At one end of the cable (reffered to as the generator/cable near end) a function generator the type Keysight 33220A (with an output impedance of $50\,\Omega$
    was connected,as well as CH1 of a Tektronix TDS 1001B two-channel oscilloscope
    (with an input impedance of $1\,\text{M}\Omega$.
    The other end of the cable (referred to as the generator/cable far end) was connected to CH2 of the oscilloscope.
    For this experiment we used $30\,$ns pulses generated by the frequency generator at $200\,$kHz and an amplitude of $5\,$V.
    The experimental setup can be seen in \autoref{fig:Construction}.
    The pulses were then recorded with the oscilloscope while the resistance at the far end of the cable was varied to simulate an open or fixed end.
    The used resistors varied form from $R=0\,\Omega $ realized by a short circuit, to $R=50\,\Omega$ and $R=\infty$ realized by leaving the end of the cable open.
    Furthermore, we set up a runtime measurement experiment in which we measured the time it took a signal to travel bach and forth through the cable.
    From that measured time, we calculate the propagation velocity of the pulses inside the cable.
    
    \subsection{Reflected pulses}\label{subsec:WaveProperitesInTheCoax}
    Using the setup from \autoref{fig:Construction}, pictures of the voltages shown on the oscilloscope for all of the possible resistor combinations $R_{\text{Near}}, R_{\text{Far}} \in \{\, 0,50\,\Omega,\infty\ \,\}$ can be taken.
    As we know from analogous phenomena -- like waves on a string -- waves behave differently depending on if the end of the string is fixed or not.
    From \autoref{eq:coaxRef} we know that the reflection coefficient is $\rho = 0 $ for $R = Z$ and $\rho = 1 $ for $R \to \infty$ meaning that the different resistors simulate fixed and loose ends respectively.
    To show this experimentally, we start with no resistor at the generator far end, $R_{\text{Far}} = \infty$.
    From \autoref{eq:coaxRef} we expect the complete reflection of the Signal or $\rho = 1$ and a doubling in amplitude at the generator far end.
    In order to minimize reflections on the generator near end, since we want to examine the influence of the resistor at the far end, we set $R_{\text{Near}}$ to 50$\,\Omega$.
    This can be seen in \autoref{fig:NoResistorFarEnd}.
    \begin{figure}[htbp]                                 %So bindet man Graphiken ein
        \begin{center}                                       %zentriert die Graphik
            \includegraphics[angle=90,width=0.9\linewidth]{pictures/Abb3_NoResistorFarEnd (2)}      %das eingentliche Einbinden; "schwarz" ist der Dateiname ohne Endung
            \caption[]{Oscillator reading of a $30\,$ns pulse with initial amplitude $5\,$V and its reflection.
            The pulses are inside a $50\,$m long coaxial cable with a wave impedance of $50\,\Omega$ per factory specification.
            The used frequency is $f=200\,$kHz and the resistor values $R_{\text{Near}}=50\,\Omega$ and $R_{\text{Far}} = \infty$.
            
                CH 1: At the near end of the cable, the amplitude of the reflected signal allows for approximation of the damping due to ohmic resistance.
                By looking at CH2 and comparing the shown amplitude with the reflected wave on CH1 (right), we approximate the amplitude of the reflected signal as being $40\%$ of the amplitude on CH2.
                
                CH 2: With the damping in mind the amplitude at the far end of the cable not would not show doubling of the initial amplitude but $120\%$ of it.
                The initial $5\,$V would then be $6\,$V.
                This is exactly what we observe.}   
            \label{fig:NoResistorFarEnd}                                      
        \end{center}
    \end{figure}
    The doubling in amplitude can not be exactly seen due to damping inside of the cable which follows an exponential decay, but the damping can be approximated.
    By looking at CH2 and comparing the amplitude to the reflected wave on CH1, we approximate the reflected wave to be $40\%$ of the amplitude on CH2.
    This would mean, that the amplitude at the far end of the cable should be $120\%$ of the initial one.
    This is exactly what we observe.
    The small deformations in the initial signal which can be seen in all following measurements, are caused by the pulse generator.
    
    By setting $R_{\text{Far}} = 0\,\Omega$ i.e by a short circuit plug, we continue our observations.
    From \autoref{eq:coaxRef} we expect a reflection coefficient of $\rho = -1$ and thereby a vanishing amplitude at the far end of the cable.
    This is analogous to a fixed end of a string which forces a node at that point.
    This is shown in \autoref{fig:ShortCircuitFarEnd}.
    \begin{figure}[htbp]                                 %So bindet man Graphiken ein
        \begin{center}                                       %zentriert die Graphik
            \includegraphics[angle=90,width=0.9\linewidth]{pictures/Abb4_ShortCircoutFarEnd}      %das eingentliche Einbinden; "schwarz" ist der Dateiname ohne Endung
            \caption[]{Oscillator reading of a $30\,$ns pulse with initial amplitude $5\,$V and its reflection.
            The pulses are inside a $50\,$m long coaxial cable with a wave impedance of $50\,\Omega$ per factory specification.
            The frequency is $f=200\,$kHz and the resistor values are $R_{\text{Near}}=50\,\Omega$ and $ R_{\text{Far}} = 0\,\Omega.$
                
                CH 1: At the near end of the cable, the reflected pulse exhibits a negative sign.
                This is is consistent with the reflection coefficient of $\rho = -1$.
                The damping of the cable was approximated to be $60\%$ per $50\,$m, which is in agreement with this observation.
                
                CH 2: The amplitude at the far end of the generator is vanishing.
                This is consistent with the calculated reflection coefficient $\rho = -1$, causing perfect anihilation of the incoming wave with the reflected one.}   %Beschriftung der Graphik
            \label{fig:ShortCircuitFarEnd}                                      %das wird zu Zitiern im Text gebraucht
        \end{center}
    \end{figure}
    
    As we can see form this experiment the reflections of waves on the resistors can pose problems since they can interfere with the incoming signals or even be mistaken with them.
    Therefore, tying to minimize reflections is a must.
    \autoref{eq:coaxRef} predicts a vanishing reflection at exactly $R=Z$ or in this case $R \approx 50\,\Omega$.
    This would also mean that the signal at the generator far end would be just the incoming signal since there is nothing to interfere with it.
    \autoref{fig:VanishingReflectionFarEnd} shows exactly that behaviour, besides the damping due to ohmic resistance which is omnipresent in these measurements.
    \begin{figure}[htbp]                                 
        \begin{center}                                       
            \includegraphics[angle=90,width=0.9\linewidth]{pictures/Abb5_VanishingReflectionFarEnd (2)}      
            \caption[]{Oscillator reading of a $30\,$ns pulse with initial amplitude $10\,$mV and its reflection.
            The pulses are inside a $50\,$m long coaxial cable with a wave impedance of $50\,\Omega$ per factory specification.
            The frequency is $f=200\,$kHz and the resistor values are $R_{\text{Near}}=50\,\Omega$ and $ R_{\text{Far}} = 50\,\Omega.$
            
                CH 1: The reading at the generator near end is showing only the incomming signal but no reflection.
                This was expected since the reflection coefficient is calcualted to be $\rho = 0$. 
                
                CH 2: The reading at the far end of the generator showing an amplitude which is about $40\%$ of the initial one.
                Since there is no reflected wave, seeing only the damped incomming wave is exactly what was expected.
                
            }   %Beschriftung der Graphik
            \label{fig:VanishingReflectionFarEnd}                                      
        \end{center}
    \end{figure}
    \subsection{Wave impedance in coaxial cables}
    As mentioned before, \autoref{eq:coaxRef} predicts a vanishing reflection at exactly $R=Z$, and since we want to measure the impedance of the cable ourselves, we vary $R_{\text{Far}}$ until the reflected pulse is minimal.
    We then measure the resistance of $R_{\text{Far}}$ afterwards.
    Since $R_{\text{Near}}$ is not important for this measurement, we set it to $R_{\text{Near}}=50\,\Omega$ as this would also minimize unwanted reflections.
    This was done for 4 different types of resistors, and the results are given in Tab. \autoref{tab:impedance}
    \begin{table}[htbp]          %so funktionieren die Tabellen in LaTeX
        \centering
        \begin{tabular*}{\linewidth}{@{\extracolsep{\fill}}cc}
            \hline
            \hline
            \rule[-7pt]{0pt}{23pt}  resistor  &  resistance $R(\rho=0)$ [\si\ohm]  	 \\
            \hline
            \rule[-5pt]{0pt}{23pt}   carbon film potentiometer   &   $45 \pm 5$  	 \\
            \rule[-5pt]{0pt}{23pt}   wire wound potentiometer   &   $44 \pm 5$  	 \\
            \rule[-5pt]{0pt}{23pt}   resistance decade   &   $50.0 \pm 1.1$  	 \\
            \rule[-5pt]{0pt}{23pt}   $50\,\si\ohm$ end plug   &   $52.1 \pm 1.1$  	 \\
            \hline
            \hline
        \end{tabular*}
        \normalsize
        \caption[]{These are resistance values for different resistors connected to the far end of the cable, measured at the point where the oscillator reading showed a vanishing reflection (commonly called imoedance mathicng).
        It was not always possible to reach completely vanishing reflections, causing the large deviations that can be seen.
        Therefore, the mean was calculated using only the last two resistance values.
        $Z=\resistanceMean $}  %siehe Graphik: Beschriftung
        \label{tab:impedance}                             %siehe Graphik: zum Zitieren
    \end{table}
    
    During the measurement, it was not always possible to reach completely vanishing reflections at the far end of the cable, making the values uncertain.
    The resistance values at which we measured the data, were at the point where the reflected amplitude was the lowest.
    Because of this large uncertainty, the errors of the carbon film potentiometer and the wire wound potentiometer are approximated to be $5\,\Omega$.
    Looking at the data from the two potentiometers, it seems we measured the resistance approximately $5\,\Omega$ to low.
    
    We assume that the problems with reaching a vanishing reflection were due to the internal construction of the potentiometers.
    In case of the potentiometers, one does not simply have a reflection of the wave at the cable/resistor border, but also reflections inside the resistor itself.
    This would cause multiple interfering waves, making unexpected modifications to the signal.
    For that reason, we did not consider the two resistance values of the potentiometers in the calculation of the mean.
    The final value for the impedance of the cable is calculated as the mean of the resistance decade value and the $50\,\Omega$ end plug value.
    The value of the mean is $Z = \resistanceMean$.
    This value suggest that the used coaxial cable was either of type RG-9/U or RG-56/U [4].
    \section{Propagation velocities of waves in coaxiall cables}
    \subsection{Runtime measurement}
    \label{subsec:runtimeMeasurement}
    For this measurement we used the setup from \autoref{fig:Construction} with $R_{\text{Near}}=50\,\Omega$ and $R_{\text{Far}}=\infty$.
    The open end was used to reflect a signal.
    The time it takes the signal to travel back and forth through the cable was measured with the cursor of the oscilloscope.
    This was done for both the coaxial cable and a retardation cable with a retardation of $19\,$ns per cm per factory specification.
    Since the length of the coaxial cable was given to be $50\,$m we used the following formula to determine the velocity
    \begin{align}
        \label{eq:runtimeVelocity}
        v=\frac{2l}{\Delta t}
        \end{align}
    which came out to be $v = \velocityRuntime$ at a frequency of 200KHz.
    According to \autoref{eq:vavePropagationVelocity} and assuming $\mu_r = 1$, this velocity corresponds to a relative permittivity $\epsilon_R = \epsilonRRuntime$.
    
    Comparing that to the retardation cable, which has a measured length of l = $\lengthRetardation$.
    We obtained a value of $\vRetRuntime$ at a frequency of $f = 4\,$kHz.
    This can be explained by the fact that the retardation cable contains materials with high magnetic permeability $\mu_r$ compared to the $\mu_r \approx 1 $ of
    common coaxial cables, which results in a lower velocity for the retardation cable, as can be seen in \autoref{eq:vavePropagationVelocity}.
    \subsection{Standing wave measurement}
    \label{subsec:standingWaveMeasurement}
    The Understanding of standing waves will offer an alternative method of calculating the propagation velocity inside the cable which can be calculated via:
    \begin{align}
        \label{eq:velocityStanding}
        v = f\cdot\lambda
        \end{align}
    with f being the frequency of the wave and $\lambda$ being the wavelength.
    Since the frequency can be controlled with the function generator only the wavelength needs to be determined.
    To determine the wavelength, we make use of the properties of standing waves. 
    
    Standing waves can be created in a coaxial cable by sending a sine signal and letting it be reflected on the far end of the cable.
    Depending on the resistance at the end of the cable, a node at the near end of the cable can only occur if the length is an integer multiple of $\frac{\lambda}{4}$.
    For a fixed end these integers are even while for open ends they are uneven.
    The measurements for $\frac{\lambda}{4},\frac{\lambda}{2}$ and $\frac{3\lambda}{4}$ respectively can be seen in \autoref{tab:standingWaveData}
    Analogously to the runtime measurement, the velocities and the relative permittivities can be calculated and can be seen \autoref{tab:standingWaveData} as well.
    As can be seen from the data, the velocity depends on the frequency of the pulse.
    From the obtained values, we assume the dielectric to be polyethylene whose value is $\epsilon_{R} = 2.25$ at $1\,$KHz [3].
    This would be consistent with specifications of most coaxial cables.
    \begin{table}[htbp]          %so funktionieren die Tabellen in LaTeX
        \centering
        \fontsize{8pt}{8pt}
        \begin{tabular*}{0.5\textwidth}{@{\extracolsep{\fill}}lcc}
            \hline
            \hline
            \rule[-5pt]{0pt}{23pt}  frequency $\cdot 10^6\,$[Hz]  & velocity$\cdot 10^8 \frac{\text{m}}{\text{s}}$ & $\epsilon_r$   	 \\
            \hline
            \rule[-5pt]{0pt}{23pt}   $ \frequencyTableFirst & $ \vStandingTableFirst &   $ \epsilonRFirst$  	 \\
            \rule[-5pt]{0pt}{23pt}    $\frequencyTableSecond & $ \vStandingTableSecond &   $ \epsilonRSecond$  	 \\
            \rule[-5pt]{0pt}{23pt}   $ \frequencyTableThird & $ \vStandingTableThird &   $ \epsilonRThird$  	 \\
            \hline
            \hline
        \end{tabular*}
        \caption[]{Calculated values for propagation velocity of waves inside a coaxial cable and the corresponding relative permittivity of the used dielectric.
        These values are obtained by counting the nodes of a standing wave inside a coaxial cable of length $l=50\,$m at different frequencies.
        The standing waves were generated by a function generator which produced sign waves with initial amplitude $10\,\text{V}_{\text{pp}}$
        The displayed frequencies are a mean over multiple adjustments, in which we tried to locate the node at the near end of the cable.
        The dielectric is assumed to be polyethylene which has a literature value of $\epsilon_{r} = 2.25$ at $1\,$KHz [3].}  %siehe Graphik: Beschriftung
        \label{tab:standingWaveData}                             %siehe Graphik: zum Zitieren
    \end{table} 
    
    \section{Damping coefficient of coaxial cables}\label{sec:dampingCoefficient}
    
    Contrary to the assumptions from section 1, physical coaxial cables do have a nonzero ohmic resistance.
    As we have already seen in the oscilloscope data from \autoref{fig:NoResistorFarEnd} - \autoref{fig:VanishingReflectionFarEnd}, the resistance leads to an exponential decay of the signal through the cable.
    The amount of damping is characterised by the damping coefficient
    \begin{align}
       D^*(x) = 20\cdot \frac{1}{x}\log(\frac{U_0}{U(x)})\,\text{dB} 
    \end{align}
    In this experiment we examined the frequency dependence of the damping coefficient.
    The same setup from \autoref{subsec:standingWaveMeasurement} was used to measure the minimal amplitude of the signal.
    As we have a superposition of two waves with different amplitudes, we know that the amplitude at the nodes is | $U(0) - U(2l)$ |, and since the amplitude on the near end of the generator $U(0)$ is known to be $5\,$V
    we can determine $U(2l)$ allowing us to calculate $D^*$.
    The calculated values are given in \autoref{tab:dampingData}
    \begin{table}
        \centering
        \fontsize{7pt}{8pt}\selectfont
        \begin{tabular*}{\linewidth}{@{\extracolsep{\fill}}lll}
    \hline  
    \hline
    \rule[-7pt]{0pt}{23pt}  Frequency $\cdot 10^6\,$[Hz]& Voltage [mV]& Damping coefficient $\cdot 10^{-3}\,$[$\frac{\text{dB}}{\text{m}}$]\\
    \hline
    \rule[-5pt]{0pt}{23pt}  $\frequencyTableFirst$ & $\nodeVoltageOne$& $ \dampingFirst $ \\
    \rule[-5pt]{0pt}{23pt}  $\frequencyTableSecond $ & $\nodeVoltageTwo$ & $ \dampingSecond $ \\
    \rule[-5pt]{0pt}{23pt}  $\frequencyTableThird$ &$\nodeVoltageThree$ & $\dampingThird $ \\
    \hline
    \hline   
        \end{tabular*}
        \caption[]{The calculated values for the damping per unit length of the used coaxial cable of length $l=50\,$m.
        The frequency values are a mean over multiple adjustments in which we tried to find the minimal amplitude at the generator near end.
        This amplitude is exactly the voltage shown.
        The standing waves were created with a function generator set to generate sine waves at $10\,\text{V}_{\text{pp}}$ amplitude.}
        \label{tab:dampingData}
    \end{table}
    The data shows that the frequency dependence of the damping is inversely proportional meaning that higher frequencies get damped more.
    The cable acts as a low pass filter.
    This is visualised in \autoref{fig:dampingFigure}
    \begin{figure}
    \begin{center}
        \includegraphics[width=\linewidth]{Generated/dampingPlot}
        \caption[]{Visualising the frequency dependence of the damping coefficient from the data in \autoref{tab:dampingData}.
        Standing waves were created inside a $50\,$m long coaxial cable.
        The frequencies at which a node was observed with a short circuit at the far end, are marked with the red triangle, $R_{\text{Far}}=0$.
        The black square marks the frequency which was observed with an open far end of the cable, $R_{\text{Far}}=\infty$.}   %Beschriftung der Graphik
        \label{fig:dampingFigure}
    \end{center}
    \end{figure}
    
    \section{Measurement of the speed of light}
    \label{sec:speedOfLightMeasurement}
    In this experiment the speed of light was to be determined using an light pulse generated by an LED, and a receiver connected to an oscilloscope.
    The pulse width was  set to $20\,$ns per... 
    The pulses were reflected by a mirror placed at a known distance from the diode and the time it took it to travel the distance was measured.
    From a total of 9 different measured distances, the speed of light was calculated to:
    \begin{align}
        c = \SpeedOfLight \frac{m}{s}
    \end{align}
    The obtained value coincides with the literature value $c  = 2.99742458\cdot10^8 \frac{\text{m}}{\text{s}}$ [3] within one percent.
    The uncertainty is largely due to the uncertainty in the time measurement since the significant raise in signal from the noise was decided subjectively.
    The graphical representation of the data is depicted in \autoref{fig:regressionFigure}.
    \begin{figure}
        \begin{center}
            \includegraphics[width=\linewidth]{Generated/regressionPLot}
            \caption[]{Graphical representation of the data-points from the measurement of the speed of light in air.
            The runtime of a $20\,$ns laser pulse produced by a special LED was measured for 9 different distances.
            A receiver was connected to an oszilloscope allowing to measure the time difference between the initial pulse and its reflection.
            By method of linear gaussian regression, a straight line was fitted to the data, allowing us to determine the speed of light by its incline.
            The obtained value is $c =\SpeedOfLight \frac{\text{m}}{\text{s}}$.
            }  
            \label{fig:regressionFigure}
        \end{center}
    \end{figure}
    \section{Conclusion}\label{sec:Conclusion}
    With both methods, we were able to examine properties of the coaxial cable such as the damping and the propagation velocity.
    Using method one with different resistors at the ends of our cable, we recreated different reflection phenomena, which then allowed us to 
    find the conditions under which the reflections are minimal.
    This happens to be when the resistance at the end matches the wave impedance of the cable which was measured to be $Z= \resistanceMean $.
    We showed that the propagation velocity can be measured which yielded $v_{\text{Runtime}}=\velocityRuntime$ at a frequency of $200\,$kHz.
    The damping was only...
    The second method did not allow for the measurement of the wave impedance so easily.
    However, the damping and the propagation velocity could be measured for 3 different frequencies, allowing a nonlinear dispersion relation to be observed.
    This is due to the fact, that the value of $\epsilon_{\text{R}}$ of the dielectric in the cable, changes with frequency, which could also be verified via measurement.
    
    If we compare the methods, we find the fist one to be much more versatile.
    It allows making all the measurements which can be done by the second method, without the burden of discrete frequencies.
    It also allows calculating the damping more easily, since the amplitude at the end of the cable can be displayed on the oscilloscope, while also
    completely avoiding the uncertainties of manually adjusting a minimal amplitude like in method two.
    The range of frequencies at which this can be done is also much wider for method one, allowing for measurement of the propagation velocity of light in air
    -- a measurement which is impossible to be done with the second method using the same equipment.
    
    
    
    \begin{thebibliography}  
        %so wird das Literaturverzeichnis erstellt
        \bibitem{instr} Physikalisches Grundpraktikum, Universität Würzburg, Modul C2, Versuch 41, \grqq Messung der Ausbreitungsgeschwindigkeit von elektromagnetischen Wellen auf Kabeln\grqq, 2021
        \bibitem{gerth} Meschede, Dieter, Gerthsen Physik, 25. Auflage, Springer-Verlag, Berlin, 2015
        \bibitem{codata} E. Tiesinga \textit{et al.}, \grqq CODATA recommended values of the fundamental physical constants: 2018\grqq, Rev. Mod. Phys. 93, 025010 (2021)
        \bibitem{young} Young, H. D.; Freedman, R. A.; Lewis, A. L. (2012). University Physics with Modern Physics (13th ed.). Addison-Wesley. p. 801. ISBN 978-0-321-69686-1.
        \bibitem{coax} https://web.archive.org/web/20100724040913/http: //www.madaboutcable.com/cables/coaxial_cables 
        \newline /rg_coaxial_cables.html 
        \newline
        \text{last visited 16.10.2023}
    \end{thebibliography}
    \clearpage
    \section{Appendix - Exercises from the instruction}\label{sec:appendix}
    \subsubsection{Numerical example of wave propagation in a cable}
    In this section we will look at a $50\,\si{\m}$ coaxial cable with a capacitance of $C^*=100\,\frac{\si{\pF}}{\si{\m}}$ and an inductance of $L^*=15\cdot 10^{-8}\,\frac{\si{\H}}{\si{\m}}$, where the outer diameter is linked to the inner diameter by $d_a=3d_i$.
    \\\\Following the equations in the instructions \cite{instr}, we find for the time it takes a signal to travel through our cable
    \begin{align*}
        \Delta t=\frac{\Delta x}{v}=\sqrt{L^*C^*}\Delta x = 0,25\,\si{\ms}.
    \end{align*}
    For the relative permittivity we find
    \begin{align*}
        \epsilon_r=L^*C^*c^2= 2,25.
    \end{align*}
    With this result and the assumption that $\mu_r=1$, we can calculate the characteristic impedance
    \begin{align*}
        Z=\sqrt{\frac{\mu_r}{\epsilon_r}\frac{\mu_0}{\epsilon_0}}\frac{1}{2\pi}\ln{\frac{d_a}{d_i}}=44,0\,\si{\Omega}
    \end{align*}
    and solving the same equation for $\frac{d_a}{d_i}$ to produce a cable with an impedance of $Z=50\,\si{\Omega}$ results in
    \begin{align*}
        \frac{d_a}{d_i}\big( Z=50\,\si{\Omega} \big ) = 3,54.
    \end{align*}
    \subsubsection{Mathematical description of waves}
    The most simple form of a wave is the monochromatic harmonic real valued wave given by
    \begin{align}\label{eq:simplewave}
    y(x,t)=a\cdot\sin{2\pi\nu\Big(t-\frac{x}{c}\Big)}.
    \end{align}
    The wavelength $\lambda$ is defined as the smallest length satisfying $y(x,t)={y(x+\lambda,t), \forall x,t}$. Plugging this into the wave equation results in the important relation
    \begin{align*}
        c=\lambda\nu,
    \end{align*}
    where c is the propagation velocity of the wave.
    \\\\Adding up two waves of the form in eq.~(\ref{eq:simplewave}), we can first look at the special case where the amplitudes $a_1=a_2=:a$ are the same and only the frequencies differ
    \begin{align}
        y&=y_1+y_2 \nonumber\\&=2a\cos{\Big(2\pi\nu^*\big(t-\frac{x}{c}\big)\Big)}\sin{\Big(2\pi\nu\big(t-\frac{x}{c}\big)\Big)}
        \label{eq:superpos_freq}
    \end{align}
    with $\nu^*:=\frac{\nu_2-\nu_1}{2}$ and $\nu:=\frac{\nu_2+\nu_1}{2}$. This describes a wave with amplitude $2a$, frequency $\nu$ and an envelope of the frequency $\nu^*$, where at the nodes of the envelope, the amplitude vanishes completely. Considering the superposition of two waves of the form in eq. \ref{eq:simplewave} with the same frequency $\nu$ but different amplitudes $a_1$ and $a_2$, we find
    \begin{align}
        y&=y_1+y_2\nonumber\\&=(a_1+a_2)\cos{(kx)}\sin{(\omega t)}\nonumber\\&+ (a_1-a_2)\sin{(kx)}\cos{(\omega t)}
        \label{eq:superpos_amp}
    \end{align}
    where $k=\frac{2\pi\nu}{c}$ and $\omega=2\pi\nu$. This is a superposition of standing waves with amplitudes $a_1+a_2$ and $|a_1-a_2|$ and a separation in phase of $\frac{\pi}{2}$. Thus, the envelope oscillates between the amplitudes $a_1+a_2$ and $|a_1-a_2|$.\\
    Up to this point, we only considered waves with a linear dispersion and thus constant propagation speed. But in general, the propagation velocity is a function of wavelength $c=c(\lambda)$. Implementing different velocities $c_1=c(\lambda_1)$ and $c_2=c(\lambda_2)$ in eq.~\ref{eq:superpos_freq} and replacing $c_2=c_1-\Delta c$ as well as $\lambda_2=\lambda_1-\Delta\lambda$, where we look at the case of very similar propagation media such that $\Delta\lambda \ll \lambda_1$ and $\Delta c \ll c_1$, we find
    \begin{align}
        y=&2a\cos{\Big(\pi\frac{(\lambda_1\Delta c - \Delta\lambda c_1)\cdot t + \Delta\lambda\cdot x}{\lambda_1^2}\Big)}\nonumber\\&\cdot\sin{\Big(2\pi\frac{c_1\cdot t - x}{\lambda_1}\Big)}.
    \end{align}
    To compare this result with eq.~\ref{eq:superpos_freq}, we can take the limit $\Delta c \to 0$ and approximate $\lambda_1\lambda_2\approx\lambda_1^2$. Inserting $\nu_i=\frac{c_1}{\lambda_i}$, we receive exactly eq.~\ref{eq:superpos_freq}.
\end{document}