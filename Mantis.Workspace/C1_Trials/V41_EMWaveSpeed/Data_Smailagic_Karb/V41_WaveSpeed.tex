%! Author = User
%! Date = 13.09.2023

% Preamble
\documentclass[a4paper,10pt,twocolumn]{article}

% Packages 
\usepackage[utf8]{inputenc}  %man kann Sonderzeiche wie ü,ö usw direkt eingeben
\usepackage{amsmath}           %macht
\usepackage{amsfonts}          %       Mathe
\usepackage{amssymb}           %              mächtiger
\usepackage{graphicx}          %erlaubt Graphiken einzubinden (.eps für dvi und ps sowie .jpg für pdf)
\usepackage[T1]{fontenc}       %Zeichenbelegung der verwendeten Schrift
\usepackage{ae}                %macht schöneres ß
\usepackage{typearea}
\usepackage{amstex}
\usepackage{siunitx}
\usepackage{mathtools}
\usepackage{hyperref}
\usepackage{hhline}	         %ermöglicht änderung des Seitenspiegels
\usepackage{caption}
\usepackage{biblatex}
\captionsetup{font=footnotesize}


\usepackage{amsmath}
\usepackage{tikz}
\usepackage{pgfplots}

\newcommand{\alphaNoError}{(4.047 \pm 0.036)}
\newcommand{\betaNoError}{(-4.73 \pm 0.29) \cdot 10^{-3}}
\newcommand{\halfTimeNoError}{(146.5 \pm 9.1)\ s}
\newcommand{\alphaGauss}{(4.04 \pm 0.10)}
\newcommand{\betaGauss}{(-4.62 \pm 0.95) \cdot 10^{-3}}
\newcommand{\halfTimeGauss}{(150 \pm 31)\ s}
\newcommand{\alphaPoisson}{(4.05 \pm 0.10)}
\newcommand{\betaPoisson}{(-4.75 \pm 0.95) \cdot 10^{-3}}
\newcommand{\halfTimePoisson}{(146 \pm 29)\ s}
\newcommand{\symN}{\delta N}


\pagestyle{scrheadings}        %sagt Koma-Skript, dass selbstdefiniers Kopfzeilen verwendet werden
\typearea{16}                  %stellt Seitenspiegel ein
\columnsep25pt								 %definiert Breite zwischen den zwei Spalten von \twocolumns

\renewcommand{\pnumfont}{%     %ändert die Schriftart der Seitennummerierung
    \normalfont\rmfamily\slshape}  %ändert die Schriftart der Seitennummerierung 



\begin{document}
    \twocolumn[{\csname @twocolumnfalse\endcsname                %erlaubt "Abstrakt" über beide Spalten
    \titlehead{                                                  %Kopfzeile
        \begin{tabular*}{\textwidth}[]{@{\extracolsep{\fill}}lr}   %Kopfzeile
            Betreuer: Lukas Elter & \today\\                          %Kopfzeile - Betreuer
        \end{tabular*}                                             %Kopfzeile
    }
    \title{Examinig Propagation Properties of Electromagnetic Waves in Coaxial Cables}  %Titel der Versuchs
    \author{Salahudin Smailagić and Thomas Karb}                     %Namen der Studenten
    \date{}                                                         %benötigt um automatisches Datum auszuschalten
    \maketitle                                                      %erzeugt Titelseite
    \vspace{-5ex}                                                   %verringert Abstand zur Überschrift
    \begin{abstract}                                                %Beginn des Abstracts
        The widespread usability of coaxial cables makes it important to understand their properties and limitations.
        Some of the limitations being their bad performance at high frequencies or the possibility of pertubations due to reflections inside the cable.
        In order to understand these better, we explored a method of analyzing pulses inside the coaxial cable by looking at their reflections of of the end of the cable.
        By changing the resistance at the end, we succesfully simulated different border conditions allowing us to measure the impedance of the cable.
        The impedance was calculated to be $Z=\resistanceMean$ which is in agreement....Since they are typically normed to $Z=50\Ohm$
        Another important property, the propagation velocity of signals inside the cable was examined using two different approaches which yielded 
        the values $v=\velocityRuntime$ and $v=\vStandingFirst$.
        The second method allowed observation of the frequency dependence of this velocity showing a nonlinear dispersion relation.
        Lastly, the damping inside the coaxial cable was calcualted showing an increase with frequency as well.
        Subsequently, the speed of light in air was measured, delivering the value $c=\SpeedOfLight$.
        This value shows great agreement with the theoretical value $c=...$.
        \\
        \\
        Measuerement made: 19. September 2023\\       %Datum ändern!
        Submitted: 26. September 2023                %Datum ändern!
        \\
        \\
    \end{abstract}
    }]
    \section{Introduction}\label{sec:introduction}
    Transmitting signals without losses is very important in all branches of everyday life.
    Coaxial cables have been known for their wide usability in signal transmission.
    Therefore, a better understanding of their transmittance properties as well as possible limitations is of high importance.

    Coaxial cables are a type of electrical cable consisting of an inner conductor surrounded by a cylindrical conducting shield(usually copper) seperated by a dielectric. 
    In an ideal coaxial cable, the electromagnetic field carrying the signals (electromagnetic waves) only exist in the space between the conductor and the shield allowing for almost no power losses making it superior to conventional
    transmission lines.
    In this paper we want to examine, the transmittance properties of coaxial cables while emphasizing the possible problems and limitations.
    
    \section{Wave propagation in cables - The coaxial cable}
    The transmission properties of a coaxial cable are characterized by the capacitance between the two conductors per unit length $C^*$ and inductance per unit length $L^*$
    Form maxwells equations the voltage and current in the cable are given by:
    \begin{align*}
        \frac{\partial^2U}{\partial x^2}=L^*C^*\frac{\partial^2U}{\partial t^2}\\
        \frac{\partial^2I}{\partial x^2}=L^*C^*\frac{\partial^2I}{\partial t^2}
    \end{align*}
    These are wave equations with the wave propagation velocity
    \begin{align}
        \label{coaxial:vel}
        v = \frac{1}{\sqrt{L^*C^*}}.
    \end{align}
    Since we know the geometry, we can calculate the inductance as well as the capacitance of the coax cable.
    \begin{align}
        C^* = 2\pi\epsilon_0\epsilon_r (\ln(\frac{d_a}{di}))^{-1}\endline
        L^* = \frac{\mu_r\mu_0}{2\pi}\ln(\frac{d_a}{d_i})
        \end{align}
    By inserting the expressions for L* and C* for a coaxial cable with relative permittivity $\epsilon_r$ and relative magnetic permeability $\mu_r$ we obtain
    \begin{align}
        \label{eq:vavePropagationVelocity}
        v = \frac{c}{\sqrt{\mu_r\epsilon_r}},
    \end{align}
    where c is the speed of light in vacuum.
    Though there is negligible ohmic resistance, coaxial cables still have an impedance of
    \begin{align}
        Z=\sqrt{\frac{L^*}{C^*}}
    \end{align}
    and the reflection coefficient $\rho$, which describes the ratio between the amplitude of the incident and the reflected wave, is given by
    \begin{align}
        \label{eq:coaxRef}
        \rho=\frac{R-Z}{R+Z}
    \end{align}
    where $R$ is a resistor at the end of the cable.
    \section{Wave reflection properties in the coaxial cable}
    \label{sec:reflectionProperties}
    \begin{figure}[htbp]                                 %So bindet man Graphiken ein
        \begin{center}                                       %zentriert die Graphik
            \includegraphics[width=1.3\linewidth,scale=1.5]{pictures/Construction}      %das eingentliche Einbinden; "schwarz" ist der Dateiname ohne Endung
            \caption[]{The experimentall setup showing the used pulse generator used to generate $30\,$ns pulses. 
            The two channels of the oscilloscope are connected to the two Resistors $R_{\text{Near/Far}}$ which are on the end of the cable.
            The coaxial cable has a lenght of $l=50\text{m}$ and wave impedance of $50\,\Omega$ per factory specification.}   %Beschriftung der Graphik
            \label{fig:Construction}                                      %das wird zu Zitiern im Text gebraucht
        \end{center}
    \end{figure}
    In order to examine the reflection properties of waves inside the cable we made use of \autoref{eq:coaxRef} which suggested, that the reflection
    coefficient $\rho$ depends on the resistance at the end of the cable.
    For this experiment we used $30\,$ns pulses generated by the frequency generator and connected both ends of the cable to an oscilloscope.
    The experimental setup can be seen in \autoref{fig:Construction}.
    The pulses were then recorded with the oscilloscope as the resistance at the far end of the cable was varied to simulate an open or fixed end.
    The used resistors varied form from $R=0\,\Omega $ realized by a short circuit, to $R=50\,\Omega$ and $R=\infty$ realized by leaving the end of the cable open.
    Furthermore, we set up a runtime measurement experiment in which we measure the time it takes a signal to travel the lenght of the cable two times.
    Form that measured time, we calculate the propagation velocity of the pulses in the cable.
    
    \subsection{Reflected pulses}\label{subsec:WaveProperitesInTheCoax}
    We use the setup form \autoref{fig:Construction} and took pictures of the voltages shown on the oscilloscope for all of the possible resistor combinations. $R_{\text{Near}}, R_{\text{Far}} \in \{ 0,50,\infty\}$
    As we know from analogous phenomena (ex. waves on a string), waves behave differently depending on if the end of the string is fixed or not.
    From \autoref{eq:coaxRef} we know that the reflection coefficient is 0 for R = Z and $\rho = 1 $ for $R \to \infty$ meaning that the different resistors simulate fixed and loose ends respectively.
    To show this experimentally, we start with no resistor at the generator far end, $R_{far} = \infty$.
    From \autoref{eq:coaxRef} we expect the complete reflection of the Signal or $\rho = 1$ and a doubling in amplitude at the generator far end.
    In order to minimize reflections on the generator near end, since we want to examine the influence of the resistor at the far end, we set $R_{near}$ to 50$\,\Omega$.
    This can be seen in \autoref{fig:NoResistorFarEnd}.
    \begin{figure}[htbp]                                 %So bindet man Graphiken ein
        \begin{center}                                       %zentriert die Graphik
            \includegraphics[angle=90,width=0.9\linewidth]{pictures/Abb3_NoResistorFarEnd (2)}      %das eingentliche Einbinden; "schwarz" ist der Dateiname ohne Endung
            \caption[]{Observation of a $30\,$ns pulse with initial amplitude $10\,$mV and its reflection in a $50\,$m long coaxial cable with a wave impedance of $50\,\Omega$ per factory specification.
            The used frequency was $f=200\,$kHz and the resistor values $R_{\text{Near}}=50\,\Omega$ and $R_{\text{Far}} = \infty$.
            
                CH 1: The near end of the cable. The amplitude of the reflected signal allows for approximation of the damping coming out to be... 
                
                CH 2: The Amplitude at the far end of the cable not showing the expected doubling due to damping, but an amplitude consistent with the aprroximated damping of around... }   %Beschriftung der Graphik
            \label{fig:NoResistorFarEnd}                                      %das wird zu Zitiern im Text gebraucht
        \end{center}
    \end{figure}
    The doubling in amplitude can not be exactly seen due to damping inside of the cable which follows an exponential decay.
    Nevertheless, comparing to Fig.\autoref{fig:VanishingReflectionFarEnd} it is clear that the amplitude is certainly not just the incoming one.
    The small deformations in signal which can be seen in all following measurements, are caused by the pulse generator. 
    By setting $R_{\text{Far}} = 0\,\Omega$ i.e by a short circuit plug, we continue our observations.
    From \autoref{eq:coaxRef} we expect a reflection coefficient of $\rho = -1$ and thereby a vanishing amplitude at the far end of the cable.
    This is analogous to a fixed end of a string which forces a node at that point.
    This is shown in \autoref{fig:ShortCircuitFarEnd}.
    \begin{figure}[htbp]                                 %So bindet man Graphiken ein
        \begin{center}                                       %zentriert die Graphik
            \includegraphics[angle=90,width=0.9\linewidth]{pictures/Abb4_ShortCircoutFarEnd}      %das eingentliche Einbinden; "schwarz" ist der Dateiname ohne Endung
            \caption[]{Observation of a $30\,$ns pulse with initial amplitude $10\,$mV and its reflection in a $50\,$m long coaxial cable with a wave impedance of $50\,\Omega$ per factory specification.
            The used frequency was $f=200\,$kHz and the resistor values $R_{\text{Near}}=50\,\Omega$ and $ R_{\text{Far}} = 0\,\Omega.$
                
                CH 1: Generator near end of the cable. The reflected pulse exhibits a negative sign which is consistent with the reflection coefficient of $\rho = -1$.
                The obvious damping is due to the nonzero ohmic resistance.
                
                CH 2: The amplitude at the far end of the generator showing zero amplitude. This is consistent with the calculated reflection coefficient, causing perfect anihilation of the incoming wave with the reflected one.}   %Beschriftung der Graphik
            \label{fig:ShortCircuitFarEnd}                                      %das wird zu Zitiern im Text gebraucht
        \end{center}
    \end{figure}
    
    As we can see form this experiment the reflections of waves on the resistors can pose problems since they can interfere with the incoming signals or even be mistaken with them.
    \autoref{eq:coaxRef} predicts a vanishing reflection at exactly $R=Z$ or in this case $R \approx 50\,\Omega$.
    This would also mean that the signal at the generator far end would be just the incoming signal since there is nothing to interfere with it.
    \autoref{fig:VanishingReflectionFarEnd} shows exactly that behaviour, besides the damping due to ohmic resistance which is omnipresent in these measurements.
    \begin{figure}[htbp]                                 
        \begin{center}                                       
            \includegraphics[angle=90,width=0.9\linewidth]{pictures/Abb5_VanishingReflectionFarEnd (2)}      
            \caption[]{Observation of a $30\,$ns pulse with initial amplitude $10\,$mV and its reflection in a $50\,$m long coaxial cable with a wave impedance of $50\,\Omega$ per factory specification.
            The used frequency was $f=200\,$kHz and the resistor values $R_{\text{Near}}=50\,\Omega$ and $ R_{\text{Far}} = 50\,\Omega.$
            
                CH 1: The signal at the generator near end showing only the incomming signal but no reflection. This was expected since the reflection coefficient is calcualted to be $\rho = 0$. 
                
                CH 2: The amplitude at the far end of the generator showing a damped amplitude. This was expected, since there is no reflected wave. The damping is once more
            caused by the nonzero ohmic resistance and is consistent with the findings from \autoref{fig:NoResistorFarEnd} with a damping of about 60\%\)
            }   %Beschriftung der Graphik
            \label{fig:VanishingReflectionFarEnd}                                      
        \end{center}
    \end{figure}
    \subsection{Wave impedance in coaxial cables}
    As mentioned before, \autoref{eq:coaxRef} predicts a vanishing reflection at exactly $R=Z$, and since we want to measure the impedance of the cable ourselves, we vary $R_{\text{Far}}$ until the reflected pulse is minimal.
    We then measure the resistance of $R_{\text{Far}}$ afterwards.
    Since $R_{\text{Near}}$ is not important for this measurement, we set it to $R_{\text{Near}}=50\,\Omega$ as this would also minimize unwanted reflections.
    This was done for 4 different types of resistors, and the results are given in Tab. \autoref{tab:impedance}
    \begin{table}[htbp]          %so funktionieren die Tabellen in LaTeX
        \centering
        \begin{tabular*}{\linewidth}{@{\extracolsep{\fill}}cc}
            \hline
            \hline
            \rule[-7pt]{0pt}{23pt}  resistor  &  resistance $R(\rho=0)$ [\si\ohm]  	 \\
            \hline
            \rule[-5pt]{0pt}{23pt}   carbon film potentiometer   &   $45.2 \pm 1.1$  	 \\
            \rule[-5pt]{0pt}{23pt}   wire wound potentiometer   &   $44.5 \pm 1.0$  	 \\
            \rule[-5pt]{0pt}{23pt}   resistance decade   &   $50.0 \pm 1,1$  	 \\
            \rule[-5pt]{0pt}{23pt}   $50\,\si\ohm$ end plug   &   $52.1 \pm 1.1$  	 \\
            \hline
            \hline
        \end{tabular*}
        \normalsize
        \caption[]{Impedance of a $50\,$m coaxial cable measured with different types of resistors by method of impedance matching.}  %siehe Graphik: Beschriftung
        \label{tab:impedance}                             %siehe Graphik: zum Zitieren
    \end{table}
    
    As can be seen from the data, the measured values do not coincide within their errors.
    The errors were approximated as it was not possible to reach completely vanishing reflections making the measurements uncertain.
    We assume that \autoref{eq:coaxRef} does not hold here due to the internal construction of the two potentiometers which possibly introduces an extra inductive impedance.
    The final value for the impedance of the cable is calculated as the mean of these values giving Z=$\resistanceMean \Omega$;
    \section{Propagation velocities of waves in coaxiall cables}
    \subsection{Runtime measurement}
    \label{subsec:runtimeMeasurement}
    For this measurement we used the setup from Fig \ref{fig:Aufbau} with $R_{near}=50\Omega$ and $R_{far}=\infty \Omega $.
    The open end was used to reflect a signal and measure the time it takes the signal to travel the length of the cable two times.
    The time difference was measured with the cursors of the oscilloscope.
    This was done for both the coaxial cable as well as the retardation cable we had available.
    Since the length of the  coaxial cable was given to be 50m we used the following formula to determine the velocity
    \begin{align}
        \label{eq:runtimeVelocity}
        v=\frac{2l}{\Delta t}
        \end{align}
    which came out to be v=$\velocityRuntime$ at a frequency of 200KHz.
    According to Eq \ref{eq:vavePropagationVelocity} and assuming $\mu_r = 1$, this velocity corresponds to a relative permittivity $\epsilon_R$ = $\epsilonRRuntime$.
    \endline
    For the retardation cable, the measured length was l = $\lengthRetardation$ which corresponds to a velocity of $\vRetRuntime$ at a frequency of f $\approx$ 4KHz.
    This can be explained by the fact that the retardation cable contains materials with high magnetic permeability $\mu_r$ compared to the $\mu_r \approx 1 $ of common coaxial cables, which results in a lower velocity as can be seen in Eq. \ref{eq:vavePropagationVelocity}.
    \subsection{Standing wave measurement}
    \label{subsec:standingWaveMeasurement}
    The Understanding of standing waves will offer an alternative method of  calculating the propagation speed of the waves in the cable.
    The propagation velocity of waves can be calculated via:
    \begin{align}
        \label{eq:velocityStanding}
        v = f\cdot\lambda
        \end{align}
    with f being the frequency of the wave and $\lambda$ being the wavelength.
    Since the frequency can be controlled with the function generator only the wavelength needs to be determined.
    To determine the wavelength, we make use of the properties of standing waves. 
    \subsubsection{Model experiment}
    The two outputs of the Generator (41-16) are connected to the two channels of the oscilloscope and using math mode internally added.
    As shown in Appendix \ref{sec:appendix}, superposing two opposing sine waves of the same frequency and amplitude results in a standing wave.
    Since the Amplitudes of the initial waves are equal, the standing wave has nodes with amplitude zero, otherwise some amplitude always remains.
    This behaviour was observed and an example of the visualized image can be seen in Fig \ref{fig:standingWaves}
    \begin{figure}
        \begin{center}
            \includegraphics[angle=90,width=0.9\linewidth]{pictures/Abb6_StandingWaves}
            \caption[]{Superposition of two opposing sine waves with same frequency and amplitude. The amplitude at the nodes is $|U_1 - U_2|$ which is 0 in this case.}   %Beschriftung der Graphik
            \label{fig:standingWaves}
        \end{center}
    \end{figure}
    \subsection{Standing waves in the coaxial cable}
    \label{subsec:standingWavesCoax}
    These waves can be created in a coaxial cable by sending a sine signal and letting it be reflected on the far end of the cable.
    Depending on the resistance at the end of the cable, a node at the near end of the cable can only be created if the frequency is an integer multiple of $\frac{\lambda}{4}$.
    For a fixed end these integers are even while for open ends they are uneven.
    The measurements for $\frac{\lambda}{4},\frac{\lambda}{2}$ and $\frac{3\lambda}{4}$ respectively can be seen in Tab. \ref{tab:standingWaveData}.
    Analogous to the runtime measurement, the velocities and the relative permittivities can be calculated and can be seen Tab. in \ref{tab:standingWaveData} as well.
    As can be seen from the data, the velocity depends on the frequency of the pulse.
    \begin{table}[htbp]          %so funktionieren die Tabellen in LaTeX
        \centering
        \fontsize{8pt}{8pt}
        \begin{tabular*}{\linewidth}{@{\extracolsep{\fill}}ccc}
            \hline
            \hline
            \rule[-5pt]{0pt}{23pt}  frequency [Hz]  & velocity$ \frac{m}{s}$ & $\epsilon_r$   	 \\
            \hline
            \rule[-5pt]{0pt}{23pt}  \tiny $ \frequencyFirst$ & \tiny$ \vStandingFirst$ &   \tiny$ \epsilonRFirst$  	 \\
            \rule[-5pt]{0pt}{23pt}    \tiny$\frequencySecond$ & \tiny$ \vStandingSecond$ &   \tiny$ \epsilonRSecond$  	 \\
            \rule[-5pt]{0pt}{23pt}   \tiny$ \frequencyThird$ & \tiny$ \vStandingThird$ &   \tiny$ \epsilonRThird$  	 \\
            \hline
            \hline
        \end{tabular*}
        \caption[]{Calculated values for propagation velocity and relative permittivity obtained by measuring the wavelegth of standing waves inside a coaxiall cable}  %siehe Graphik: Beschriftung
        \label{tab:standingWaveData}                             %siehe Graphik: zum Zitieren
    \end{table} 
    
    \section{Damping coefficient of coaxial cables}\label{sec:dampingCoefficient}
    Contrary to the assumptions from section 1, physical coaxial cables do have a nonzero ohmic resistance.
    As we have already seen in the oscillator data from Fig. 2-4, the resistance leads to an exponential decay of the signal through the cable.
    The amount of damping is characterised by the damping coefficient
    \begin{align}
       D^* = 20\cdot \frac{1}{x}\log(\frac{U_0}{U_x}) dB 
    \end{align}
    In this experiment we examined the frequency dependence of the damping coefficient.
    The setup from \ref{subsec:standingWaveMeasurement} was used to measure the minimal amplitude of the signal.
    As we have a superposition of two waves with different amplitudes, we know that the amplitude at the nodes is | $U(0) - U(2l)$ |, and since the amplitude on the near end of the generator U(0) is known
    we can determine U(2l) allowing us to calculate $D^*$.
    The calculated values are given in Tab. \ref{tab:dampingData}
    \begin{table}
            \centering
            \caption{}
            \label{tab:dampingData}
            \begin{tabular*}{\linewidth}{@{\extracolsep{\fill}}cc}
    \hline  
    \hline
    \rule[-7pt]{0pt}{23pt}  frequency [Hz]& damping coefficient dB/m \\
    \hline
    \rule[-5pt]{0pt}{23pt}  $\frequencyFirst$ & $ \dampingFirst $ \\
    \rule[-5pt]{0pt}{23pt}  $\frequencySecond$ & $ \dampingSecond $ \\
    \rule[-5pt]{0pt}{23pt}  $\frequencyThird$ & $\dampingThird $ \\
    \hline
    \hline   
            \end{tabular*}
        \end{table}
    The data shows that the frequency dependence of the damping is inversely proportional meaning that higher frequencies get damped more.
    The cable acts as a low pass filter.
    This is visualised in Fig. \ref{fig:damoingFigure}
    \begin{figure}\label{fig:damoingFigure}
    \begin{center}
        \includegraphics[width=0.9\linewidth]{Generated/dampingPlot}
        \caption[]{Visualising the frequency dependence of the damping coefficient form the data in Tab. \ref{tab:dampingData}}   %Beschriftung der Graphik
    \end{center}
        \end{figure}
    \section{Measurement of the speed of light}
    \label{sec:speedOfLightMeasurement}
    In this experiment the speed of light was to be determined using a special LED connected to an oscilloscope.
    The 20ns wide light pulses were reflected of a mirror placed at a known distance from the diode and the time it took it to travel the distance was measured.
    From a total of 9 different measured distances, the speed of light was calculated to:
    \begin{align}
        c = \SpeedOfLight \frac{m}{s}
    \end{align}
    The obtained value coincides with the literature value\footnote{$c=2.99742458\cdot10^8 \frac{m}{s} $ 2018 CODATA} within one percent which was expected since air has a small refractive index.
    The uncertainty is largely due to the uncertainty in the time measurement since the significant raise in signal from the noise was decided subjectively.
    
    \section{Conclusion}\label{sec:Conclusion}
    In this paper we examined the behaviour of electromagnetic waves in coaxial cables.
    Analogous to reflection phenomena of waves on a string, we were able to recreate open/fixed ends with different resistors at the far end of the cable.
    The observed behaviour in the cable was in very good agreement with the theory from \ref{subsec:WaveProperitesInTheCoax}.
    The wave impedance of the cable was calculate using the method of vanishing reflection.
    The obtained value of Z=$\resistanceMean$ is in agreement with the factory specification of the cable.
    The rather large error is due to the internal construction of the used resistors for some of whom Eq. \ref{eq:coaxRef} probably does not hold.
    Calculating the damping coefficient was done using standing by measuring the amplitude of the node at the near end of the cable.
    This was done for different frequencies allowing the trend of increase with frequency to be observed.
    The propagation velocity of the signals through the cable was measured using two different methods. 
    The runtime measurement from \ref{subsec:runtimeMeasurement} and the standing wave measurement from \ref{subsec:standingWaveMeasurement} show great agreement within the error bars.
    The standing wave measurement also allowed a dispersion relation to be observed, since an increase in propagation velocity with frequency is obvious.
    From calculating the relative permittivity data we assume the dielectric to be polyethylene which would be consistent with specifications of most coaxial cables.
    Afterwards, the speed of light in air was measured using equipment detailed in \ref{sec:speedOfLightMeasurement}.
    The obtained value was in great agreement with literature.
    \begin{thebibliography}{}    %so wird das Literaturverzeichnis erstellt
        \bibitem{instr} Physikalisches Grundpraktikum, Universität Würzburg, Modul C2, Versuch 41, \grqq Messung der Ausbreitungsgeschwindigkeit
        von elektromagnetischen Wellen auf Kabeln\grqq, 2021
        \bibitem{gerth} Meschede, Dieter, Gerthsen Physik, 25. Auflage, Springer-Verlag, Berlin, 2015
        \bibitem{codata} E. Tiesinga \textit{et al.}, \grqq CODATA
        recommended values of the fundamental physical constants: 2018\grqq, Rev. Mod. Phys. 93, 025010 (2021)
        
    \end{thebibliography}
    \newpage
    \section{Appendix - Exercises from the instruction}\label{sec:appendix}
    \subsubsection{Numerical example of wave propagation in a cable}
    In this section we will look at a $50\,\si{\m}$ coaxial cable with a capacitance of $C^*=100\,\frac{\si{\pF}}{\si{\m}}$ and an inductance of $L^*=15\cdot 10^{-8}\,\frac{\si{\H}}{\si{\m}}$, where the outer diameter is linked to the inner diameter by $d_a=3d_i$.
    \\\\Following the equations in the instructions \cite{instr}, we find for the time it takes a signal to travel through our cable
    \begin{align*}
        \Delta t=\frac{\Delta x}{v}=\sqrt{L^*C^*}\Delta x = 0,25\,\si{\ms}.
    \end{align*}
    For the relative permittivity we find
    \begin{align*}
        \epsilon_r=L^*C^*c^2= 2,25.
    \end{align*}
    With this result and the assumption that $\mu_r=1$, we can calculate the characteristic impedance
    \begin{align*}
        Z=\sqrt{\frac{\mu_r}{\epsilon_r}\frac{\mu_0}{\epsilon_0}}\frac{1}{2\pi}\ln{\frac{d_a}{d_i}}=44,0\,\si{\Omega}
    \end{align*}
    and solving the same equation for $\frac{d_a}{d_i}$ to produce a cable with an impedance of $Z=50\,\si{\Omega}$ results in
    \begin{align*}
        \frac{d_a}{d_i}\big( Z=50\,\si{\Omega} \big ) = 3,54.
    \end{align*}
    \subsubsection{Mathematical description of waves}
    The most simple form of a wave is the monochromatic harmonic real valued wave given by
    \begin{align}\label{eq:simplewave}
    y(x,t)=a\cdot\sin{2\pi\nu\Big(t-\frac{x}{c}\Big)}.
    \end{align}
    The wavelength $\lambda$ is defined as the smallest length satisfying $y(x,t)={y(x+\lambda,t), \forall x,t}$. Plugging this into the wave equation results in the important relation
    \begin{align*}
        c=\lambda\nu,
    \end{align*}
    where c is the propagation velocity of the wave.
    \\\\Adding up two waves of the form in eq.~(\ref{eq:simplewave}), we can first look at the special case where the amplitudes $a_1=a_2=:a$ are the same and only the frequencies differ
    \begin{align}
        y&=y_1+y_2 \nonumber\\&=2a\cos{\Big(2\pi\nu^*\big(t-\frac{x}{c}\big)\Big)}\sin{\Big(2\pi\nu\big(t-\frac{x}{c}\big)\Big)}
        \label{eq:superpos_freq}
    \end{align}
    with $\nu^*:=\frac{\nu_2-\nu_1}{2}$ and $\nu:=\frac{\nu_2+\nu_1}{2}$. This describes a wave with amplitude $2a$, frequency $\nu$ and an envelope of the frequency $\nu^*$, where at the nodes of the envelope, the amplitude vanishes completely. Considering the superposition of two waves of the form in eq. \ref{eq:simplewave} with the same frequency $\nu$ but different amplitudes $a_1$ and $a_2$, we find
    \begin{align}
        y&=y_1+y_2\nonumber\\&=(a_1+a_2)\cos{(kx)}\sin{(\omega t)}\nonumber\\&+ (a_1-a_2)\sin{(kx)}\cos{(\omega t)}
        \label{eq:superpos_amp}
    \end{align}
    where $k=\frac{2\pi\nu}{c}$ and $\omega=2\pi\nu$. This is a superposition of standing waves with amplitudes $a_1+a_2$ and $|a_1-a_2|$ and a separation in phase of $\frac{\pi}{2}$. Thus, the envelope oscillates between the amplitudes $a_1+a_2$ and $|a_1-a_2|$.\\
    Up to this point, we only considered waves with a linear dispersion and thus constant propagation speed. But in general, the propagation velocity is a function of wavelength $c=c(\lambda)$. Implementing different velocities $c_1=c(\lambda_1)$ and $c_2=c(\lambda_2)$ in eq.~\ref{eq:superpos_freq} and replacing $c_2=c_1-\Delta c$ as well as $\lambda_2=\lambda_1-\Delta\lambda$, where we look at the case of very similar propagation media such that $\Delta\lambda \ll \lambda_1$ and $\Delta c \ll c_1$, we find
    \begin{align}
        y=&2a\cos{\Big(\pi\frac{(\lambda_1\Delta c - \Delta\lambda c_1)\cdot t + \Delta\lambda\cdot x}{\lambda_1^2}\Big)}\nonumber\\&\cdot\sin{\Big(2\pi\frac{c_1\cdot t - x}{\lambda_1}\Big)}.
    \end{align}
    To compare this result with eq.~\ref{eq:superpos_freq}, we can take the limit $\Delta c \to 0$ and approximate $\lambda_1\lambda_2\approx\lambda_1^2$. Inserting $\nu_i=\frac{c_1}{\lambda_i}$, we receive exactly eq.~\ref{eq:superpos_freq}.
\end{document}