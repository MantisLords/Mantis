%! Author = User
%! Date = 13.09.2023

% Preamble
\documentclass[a4paper,10pt,twocolumn]{article}

% Packages 
\usepackage[utf8]{inputenc}  %man kann Sonderzeiche wie ü,ö usw direkt eingeben
\usepackage{amsmath}           %macht
\usepackage{amsfonts}          %       Mathe
\usepackage{amssymb}           %              mächtiger
\usepackage{graphicx}          %erlaubt Graphiken einzubinden (.eps für dvi und ps sowie .jpg für pdf)
\usepackage[T1]{fontenc}       %Zeichenbelegung der verwendeten Schrift
\usepackage{ae}                %macht schöneres ß
\usepackage{typearea}
\usepackage{amstex}
\usepackage{siunitx}	         %ermöglicht änderung des Seitenspiegels



\pagestyle{scrheadings}        %sagt Koma-Skript, dass selbstdefiniers Kopfzeilen verwendet werden
\typearea{16}                  %stellt Seitenspiegel ein
\columnsep25pt								 %definiert Breite zwischen den zwei Spalten von \twocolumns

\renewcommand{\pnumfont}{%     %ändert die Schriftart der Seitennummerierung
    \normalfont\rmfamily\slshape}  %ändert die Schriftart der Seitennummerierung 



\begin{document}
    \twocolumn[{\csname @twocolumnfalse\endcsname                %erlaubt "Abstrakt" über beide Spalten
    \titlehead{                                                  %Kopfzeile
        \begin{tabular*}{\textwidth}[]{@{\extracolsep{\fill}}lr}   %Kopfzeile
            Betreuer: Ronny Thomale & \today\\                          %Kopfzeile - Betreuer
        \end{tabular*}                                             %Kopfzeile
    }
    \title{Examinig Propagation Properties of Electromagnetic Waves in Coaxial Cables}  %Titel der Versuchs
    \author{Salahudin Smailagić and Thomas Karb}                     %Namen der Studenten
    \date{}                                                         %benötigt um automatisches Datum auszuschalten
    \maketitle                                                      %erzeugt Titelseite
    \vspace{-5ex}                                                   %verringert Abstand zur Überschrift
    \begin{abstract}                                                %Beginn des Abstracts
        This is very abstracvt.
        In fact it is so abstract, that I am scared.
        \\
        Measuerement made: 19. September 2023\\       %Datum ändern!
        Submitted: 26. September 2023                %Datum ändern!
        \\
        \\
    \end{abstract}
    }]
    \section{Introduction}
    Transmitting signals without losses is very important in all branches of everyday life.
    Thus, a better understanding of the established methods of conductance is a must.
    A commonly used mean of signal transmission is the coaxial cable, which is easy to manufacture and due to its construction minimizes energy loss and environmental interference.
    It can be found everywhere, form telephone lines through cable television signals to broadband internet.
    In this paper, we want to examine the propagation properies of singlas - electromagnetic waves, through these cables.
    \section{Theory}
    \subsection{Wave propagation in cables - The coaxial cable}
    Coaxial cables are a type of electrical cable consisting of an inner conductor surrounded by a cylindrical conducting shield(usually copper) seperated by a dielectric. In an ideal coaxial
    cable the electromagnetic field carrying the signals (electromagnetic waves) only exist in the space between the conductor and the shield allowing for almost no power losses making it superior to conventional
    transmission lines. 
    Further, electromagnetic field inside is shielded form fields outside making the cable ideal for carrying weak signals.
    These properties explain the widespread usage of coax cables for both carrying weak signals that must not interfere with the environment as well as strong signals to minimize losses.
    The transmission properties of a coaxial cable are characterized by the capacitance between the two conductors per unit length $C^*$ and inductance per unit length $L^*$
    Form maxwells equations the voltage and current in the cable are given by:
    \begin{align*}
        \frac{\partial^2U}{\partial x^2}=L^*C^*\frac{\partial^2U}{\partial t^2}\\
        \frac{\partial^2I}{\partial x^2}=L^*C^*\frac{\partial^2I}{\partial t^2}
    \end{align*}
    These are wave equations with the wave propagation velocity
    \begin{align}
        \label{coaxial:vel}
        v = \frac{1}{\sqrt{L^*C^*}}.
    \end{align}
    Since we know the geometry, we can calculate the inductance as well as the capacitance of the coax cable.
    \begin{align}
        C^* = 2\pi\epsilon_0\epsilon_r (\ln(\frac{d_a}{di}))^{-1}\endline
        L^* = \frac{\mu_r\mu_0}{2\pi}\ln(\frac{d_a}{d_i})
        \end{align}
    By inserting the expressions for L* and C* for a coaxial cable with relative permittivity $\epsilon_r$ and relative magnetic permeability $\mu_r$ we obtain
    \begin{align}
        \label{eq:vel_mat}
        v = \frac{c}{\sqrt{\mu_r\epsilon_r}},
    \end{align}
    where c is the speed of light in vacuum.
    Though there is negligible ohmic resistance, coaxial cables still have an impedance of
    \begin{align}
        Z=\sqrt{\frac{L^*}{C^*}}
    \end{align}
    and the reflection coefficient $\rho$, which describes the ratio between the amplitude of the incident and the reflected wave, is given by
    \begin{align}
        \label{eq:coaxRef}
        \rho=\frac{R-Z}{R+Z}
    \end{align}
    where $R$ is a resistor at the end of the cable.
    \section{Experiment and setup}
    The broad experimental setup can be seen in Fig.1 
    \begin{figure}[htbp]                                 %So bindet man Graphiken ein
        \begin{center}                                       %zentriert die Graphik
            \includegraphics[width=0.9\linewidth]{pictures/Abb2_Versuchsaufbau1}      %das eingentliche Einbinden; "schwarz" ist der Dateiname ohne Endung
            \caption[]{Sed fermentum vestibulum wisi. Nunc dictum ligula at ipsum. Integer vulputate elit sed enim. Sed pede dolor, convallis quis, rhoncus ut, aliquet ut, urna.}   %Beschriftung der Graphik
            \label{fig:Aufbau}                                      %das wird zu Zitiern im Text gebraucht
        \end{center}
    \end{figure}
    \section{Wave properties in the coax cable}
    In this section we examine the shape of electromagnetic pulses and their reflections on resistors on the far end of the generator.
    Furthermore we examine the properties of these pulses, measuring the time it takes a signal to reflect on the far end of the cable and come back as well as creating a standing wave and counting the wavelengths in the cable to determine the propagation velocity.
    
    \subsection{Reflected pulses}\label{subsec:wave-impedance-in-the-coax}
    We use the setup form Fig.1 and take pictures of the voltages shown on the oscilloscope for all of the possible resistor combinations. $R_{near}, R_{far} \in \{ 0,50,\infty\} \Omega$
    As we know from analogous phenomena (ex. waves on a string), waves behave differently depending on if the end of the string is fixed or not.
    From Eq. \ref{eq:coaxRef} we know that the reflection coefficient is 0 for R = Z and 1 for $\lim_{R \to \infty}\rho$ meaning that the different resistors simulate fixed and loose ends respectively.
    To show this experimentally, we start with no resistor at the generator far end, $R_{far} = \infty$.
    From \ref{eq:coaxRef} we expect the complete reflection of the Signal or $\rho = 1$ and a doubling in amplitude.
    To minimize reflections on the generator near end, since we want to examine the influence of the resistor at the far end, we set $R_{near}$ to 50$\Omega$.
    This can be seen in \ref{fig:NoResistorFarEnd}
    \begin{figure}[htbp]                                 %So bindet man Graphiken ein
        \begin{center}                                       %zentriert die Graphik
            \includegraphics[width=0.9\linewidth]{pictures/Abb3_NoResistorFarEnd}      %das eingentliche Einbinden; "schwarz" ist der Dateiname ohne Endung
            \caption[]{"Meine Schokoladenseite" - Ronny Thomale}   %Beschriftung der Graphik
            \label{fig:NoResistorFarEnd}                                      %das wird zu Zitiern im Text gebraucht
        \end{center}
    \end{figure}
    \textit{The figure does not show that due to damping effects. The general trend however is still visible. Some deformation can also be seen, that can be explained by taking the capacitance and inductance of the used cables into account}
    
    Now we use the short circuit plug at the generator far end and expect From Eq. \ref{eq:coaxRef} a reflection coefficient of -1 and a vanishing amplitude at the generator far end.
    This is analogous to a fixed end which forces a node.
    This is shown in \ref{fig:ShortCircuitFarEnd}, with the damped amplitude.
    \begin{figure}[htbp]                                 %So bindet man Graphiken ein
        \begin{center}                                       %zentriert die Graphik
            \includegraphics[width=0.9\linewidth]{pictures/Abb4_ShortCircuitFaeEnd}      %das eingentliche Einbinden; "schwarz" ist der Dateiname ohne Endung
            \caption[]{Ich schaue in deine Seele}   %Beschriftung der Graphik
            \label{fig:ShortCircuitFarEnd}                                      %das wird zu Zitiern im Text gebraucht
        \end{center}
    \end{figure}
    
    As we can see form this experiment the reflections of waves on the resistors can pose problems since they can interfere with the incoming signals or even be mistaken with them.
    Eq. \ref{eq:coaxRef} predicts a vanishing reflection at exactly $R=Z$ or in this case R$\approx 50\Omega$, this would also mean that the signal at the generator far end would be just the incoming signal since there is nothing to interfere with it.
    Fig \ref{fig:VanishingReflectionFarEnd} shows exactly that behaviour, besides the damping which is omnipresent in these measurements.
    \begin{figure}[htbp]                                 
        \begin{center}                                       
            \includegraphics[width=0.9\linewidth]{pictures/Abb5_VanishingReflectionFarEnd}      
            \caption[]{Rhommy}   %Beschriftung der Graphik
            \label{fig:VanishingReflectionFarEnd}                                      
        \end{center}
    \end{figure}
    
    noch zu R_{near}
    
    As mentioned before, Eq. \ref{eq:coaxRef} predicts a vanishing reflection at exactly R=Z, and since we want to measure the impedance of the cable ourselves, we vary $R_{far}$ until the reflected pulse is minimal and measure the resistance of $R_{far}$ afterwards.
    This was done for 4 different types of resistors, and the results are given in Tab. \ref{tab:impedance}
    \begin{table}[htbp]          %so funktionieren die Tabellen in LaTeX
        \centering
        \begin{tabular*}{\linewidth}{@{\extracolsep{\fill}}cc}
            \hline
            \hline
            \rule[-7pt]{0pt}{23pt}  resistor  &  resistance $R(\rho=0)$ [\si\ohm]  	 \\
            \hline
            \rule[-5pt]{0pt}{23pt}   carbon film potentiometer   &   $50,4 \pm 1,1$  	 \\
            \rule[-5pt]{0pt}{23pt}   wire wound potentiometer   &   $43,3 \pm 1,0$  	 \\
            \rule[-5pt]{0pt}{23pt}   resistance decade   &   $50,0 \pm 1,1$  	 \\
            \rule[-5pt]{0pt}{23pt}   short circuit plug   &   -  	 \\
            \rule[-5pt]{0pt}{23pt}   $50\,\si\ohm$ end plug   &   $48,2 \pm 1,1$  	 \\
            \hline
            \hline
        \end{tabular*}
        \normalsize
        \caption[]{Impedance of a coaxial cable measured with different types of resistors by impedance matching.}  %siehe Graphik: Beschriftung
        \label{tab:impedance}                             %siehe Graphik: zum Zitieren
    \end{table}
    
    diskussion zu den berechneten werten.... wollte nicht ganz passen - wire wound potentiometer.
    
    \subsection{Standing waves}
    The Understanding of standing waves will offer an alternative method of  calculating the propagation speed of the waves in the cable.
    Here follows a small introduction to the experimentally setup as well as an explanation of the wave behaviour.
    \subsection{setup}
    The two outputs of the Generator (41-16) are connected to the two channels of the oscilloscope and using math mode internally added.
    As shown in Appendix \ref{}, superposing two waves of the same frequency and amplitude shifted by $\pi$ results in a standing wave.
    Depending on if the ends of...
    
    
\end{document}