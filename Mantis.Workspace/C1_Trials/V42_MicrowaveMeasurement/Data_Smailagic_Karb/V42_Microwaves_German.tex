%! Author = User
%! Date = 13.09.2023

% Preamble
\documentclass[a4paper,10pt,twocolumn]{article}

% Packages 
\usepackage[utf8]{inputenc}  %man kann Sonderzeiche wie ü,ö usw direkt eingeben
\usepackage{amsmath}           %macht
\usepackage{amsfonts}          %       Mathe
\usepackage{amssymb}           %              mächtiger
\usepackage{graphicx}          %erlaubt Graphiken einzubinden (.eps für dvi und ps sowie .jpg für pdf)
\usepackage[T1]{fontenc}       %Zeichenbelegung der verwendeten Schrift
\usepackage{ae}                %macht schöneres ß
\usepackage{typearea}
\usepackage{amstex}
\usepackage{siunitx}
\usepackage{hyperref}	         %ermöglicht änderung des Seitenspiegels
\usepackage{subcaption}


\usepackage{amsmath}
\usepackage{tikz}
\usepackage{pgfplots}

\newcommand{\TemperatureHumidityCovariance}{102.3\ ^{\circ} C g / m^3}
\newcommand{\TemperatureHumidityCorrelation}{0.8571}
\newcommand{\RegressionOffset}{(1.8 \pm 1.2)\  g / m^3}
\newcommand{\RegressionSlope}{(0.343 \pm 0.057)\ g / m^3 / ^{\circ} C}



\pagestyle{scrheadings}        %sagt Koma-Skript, dass selbstdefiniers Kopfzeilen verwendet werden
\typearea{16}                  %stellt Seitenspiegel ein
\columnsep25pt								 %definiert Breite zwischen den zwei Spalten von \twocolumns

\renewcommand{\pnumfont}{%     %ändert die Schriftart der Seitennummerierung
    \normalfont\rmfamily\slshape}  %ändert die Schriftart der Seitennummerierung 



\begin{document}
    \twocolumn[{\csname @twocolumnfalse\endcsname                %erlaubt "Abstrakt" über beide Spalten
    \titlehead{                                                  %Kopfzeile
        \begin{tabular*}{\textwidth}[]{@{\extracolsep{\fill}}lr}   %Kopfzeile
            Tutor: Jochen Kaupp & \today\\                          %Kopfzeile - Betreuer
        \end{tabular*}                                             %Kopfzeile
    }
    \title{Experimental methods for examining properties of classical light with microwaves: Polarisation, Total Reflaction
    and tunneling; Using crystal model to demonstrate bragg reflection}  %Titel der Versuchs
    \author{Salahudin Smailagić and Thomas Karb}                     %Namen der Studenten
    \date{}                                                         %benötigt um automatisches Datum auszuschalten
    \maketitle                                                      %erzeugt Titelseite
    \vspace{-5ex}                                                   %verringert Abstand zur Überschrift
    \begin{abstract}                                                %Beginn des Abstracts
%        We use microwaves to study the wave character of light.
%        Thier wavelength in cm range make it possible to reveil properties of EM-waves, which would
%        for visible light require significantly more complex setups.
%        Firstly the wavelength can be easely measured by counting nodes.
%        Herefor we use a setup with standing waves and a michelson interferometer.
%        We verify polarisation and Malus'law for linear-polarisation-filters.
%        Furthermore we show total reflection and the tunnel effect, which only can be explained
%        by the wave character of the used microwaves.
%        These effects are then applied for bragg diffraction.
%        We use an exemplary crystal model to demonstrate the principle.
%        At last we show qualitatively the usage of a metal tube as waveguide.
        
        Die Maxwell-Gleichungen beschreiben Licht als elektromagnetische Welle.
        Ziel dieses Papers ist der Nachweis der Welleneigenschaften.
        Wir nutzen hierzu eine Gunn-Diode, welche Microwellen emittiert.
        Aus der Wellenoptik ergibt sich, dass vor einem Spiegel sich stehende Wellen ausbilden.
        Wir nutzen diese Eigenschaft, um die Wellenlänge der Diode zu bestimmen, indem wir die Knoten der sthenden
        Welle zählen: $\lambda = \StandingWavesWaveLength$.
        Ferner interferieren zwei überlagerte Lichtstrahlen.
        Mit Hilfe eines Micealson-Interferometers nutzen wir dies, für eine zweite Bestimmung der Wellenlänge 
        $\lambda = \InterferometerWaveLength$.
        Elektro-magnetische Wellen sind keine skalaren Wellen, sondern sind polarisiert.
        Dies weisen wir nach, indem wir einen linearen Polfilter verwenden.
        Wir können das Gesetz von Malus, welches die Durchlass-Intensität des Polfilters in Abhängigkeit des Winkels
        beschreibt, zu mindest qualitativ bestätigen.
        Hinter einer Fläche, an der die Lichtstrahlen totalreflektiert werden, bildet sich laut der Theorie
        evaleszente Wellen aus. % TODO Rechtschreibung
        Wir weisen diese nach unter Verwendung zweier Wachsprismen.
        Die Intensität der evaleszenten Welle nimmt exponentiell ab, was wir bestätigen können.
        Die Reflexions- und Interferenzeigenschaften macht sich die Methode der Bragg-Reflexion zu nutze,
        um Gitterkonstanten zu bestimmen.
        Wir demonstrieren das Prinzip anhand eines Kristallmodels.
        Zu letzt zeigen wir qualitativ, dass ein Metalschlauch als Wellenleiter verwendet werden kann.
        
        \\
        Measuerement made: 21. September 2023\\       %Datum ändern!
        Submitted: 28. September 2023                %Datum ändern!
        \\
        \\
    \end{abstract}
    }] 
    \section{Introduction}
    To study wave-properties of light we use microwaves.
    They have a large wavelength in the cm-range.
    Effects like interference and tunneling occur on a macroscopic scale, which makes it easy to show them
    in an experiment.
    Because all properties shown here apply also for light in the visible spectrum, you can 
    adopt the presented methods on a smaller lengthscale.
    
    \section{Experimenteller Aufbau}
    Für unsere Experimente benutzen wir das `PASCO Microwave Optics System WA-9316A'.
    \subsection{Empfänger}
    
    Wir benutzten den im `PASCO'-System enthaltenen Empfänger ´WA-9802'.
    Er ist eine Scotty-Diode, welche nur linear polarisiertes Licht detektiert.
    Zur Fokussierung ist eine Horn-Antenne auf dem Empfänger montiert.
    Der Empfänger gibt eine Spannung aus, welche wir dann mit dem Voltmeter Agilent 34405A messen.
    Es ist anzumerken die ausgegebene Spannung ist weder proportional zur Intensität $I$ der Mikrowellen oder zum E-Feld $E$.
    Stattdessen ist es ein Zwischenwert~\cite{pasco}.
    Aus diesem Grund können wir nicht von der Spannung auf die Intensität zurück rechnen.
    Deswegen sind in all unseren Daten nur die gemessenen Spannungen angegeben.
    Ferner gibt der Empfänger eine Spannung aus, auch wenn keine Mikrowellen vorhanden sind.
    Diese Nullpunktverschiebung ist aus den angegebenen Daten korrigiert worden.
    
    \subsection{Transmitter}

%    \figAngleDispersion{Angle dispersion of the Gunn diode used in our setup.
%    We fit a gauss curve to get an approximation of the variance.
%        $fwhm = \AngleDispersionGaussFWHM$}
    \figAngleDispersion{
        Die Winkeldispersion der in unserem Aufbau verwendeten Gunn-Diode.
        Der Empfänger wurde um denn Emitter geschwängt, und abhängig vom Winkel die Intensität zu messen.
        Wir passen eine generische Gaußkurve and die Daten an, um eine grobe Abschätzung der Streubreite zu erhalten:
        $fwhm = \AngleDispersionGaussFWHM$ . % TODO Graphiken
    }
    

%    A Gunn diode was used to create the microwaves in our experiments. 
%    It has a horn antenna for focusing the light.
%    We measured the angle-intensity-relation and made a basic gauss-fit to get a rough estimate of the light dispersion (cf.~\ref{fig:AngleDispersion}).
%    The full-width-half-maximum of the used emitter is

    Wir benutzen den Transmitter `WA-9801' aus dem `PASCO'-System.
    Dieser ist eine Gunn-Diode auf der ebenfalls eine Horn-Antenne montiert ist.
    Diese emittiert Mikrowellen mit einer Wellenlänge von $\lambda = \OfficialWaveLength$ laut Hersteller.
    
    Zu nächst wollen wir die Winkeldispersion der Gunn-Diode bestimmen, um etwaige Fehler durch eine
    große Streuung abschätzen zu können.
    In einem Bereich von $\pm 90 ^{\circ}$ messen wir die Intensität in Abhängigkeit des Winkels.
    Die Daten sind in Abbildung~\ref{fig:AngleDispersion} aufgetragen.
    Wir führen einen Gauß-Fit durch, um eine grobe Abschätzung der Lichtstreuung zu erhalten.
    Auf den tatsächlichen funktionellen Zusammenhang können wir nicht zurück schließen, da die gemessene Spannung
    nicht proportional zur Intensität ist.
    Das Full-width-half-maximum der verwendeten Diode ist
    
    \begin{align*}
        fwhm = \AngleDispersionGaussFWHM
    \end{align*}
    
    Der angegebene Fehler wurde über die Parameterfehlern des Fits bestimmt.
    Er bestimmt sich aus der Streuung der Datenpunkte.
    Der systematische Fehler, welcher sich aus der auf Grund der unbekannten Spannung-Intensität-Beziehung ergibt, kann
    nicht abgeschätzt werden.
    
    \subsection{Linse}

%    \figimageDistance{We measure the signal depending on the reciever-lens distance $b$.
%    The intensity fluctuates since a stationary wave overlaps.
%    We apply a gauss-fit to get the maximum image distance.
%    This can be used to calculate the focal length after the lens formula ~\eqref{eq:LensFormula}}
    \figimageDistance{Wir messen die Intesität in Abhängigkeit der Bildweite $b$.
    Die Intesität schwankt stark, da sich eine stehende Welle überlagert.
    Wir verwenden eine generische Gauß-Anpassung, um eine Abschätzung des Maximums zu erhalten:
    $b = \ImageDistance$.
    Mit der Abbildungsgleichung~\ref{eq:LensFormula} kann die Brennweite bestimmt werden:
    $f = \FocalLength$.
    }
    
%    For the experiments with bragg diffraction and tunnel effect we need parallel light rays.
%    We used a wax lens for focusing the emitted microwaves.
%    To determine the focal length, the lens was placed away from the emitting diode.
%    The object-distance\footnote{emitter-lense distance after ~\cite{pasco}} in our setup was $g = \ObjectDistance$
%    Then by varying the receiver positions the intensity was measured as shown in figure ~\ref{fig:imageDistance}. 
%    Because the wavelength is in the cm range, the diameter of the resulting airy disk is also in the cm range.
%    This means there is no distance were the object (emitter) is completely focused as you would expect from geometric optics.
%    So to get the actual image-distance you need to find the maximum.
%    Since we have few data points with much noise we use a gauss curve to find an approximation of the maximum.
%    Though the actual theoretical intensity-distribution is more complicated.

    Für die Experimente mit Bragg-Beugung und Tunneleffekt benötigen wir parallele Lichtstrahlen. 
    Wir verwendeten eine Wachslinse zur Fokussierung der emittierten Mikrowellen. 
    Um die Brennweite zu bestimmen, wird die Linse ca.\ $80 \mathrm{cm}$ entfernt von der emittierenden Diode platziert. 
    Die Gegenstandweite zwischen Linse und Transmitter betrug in unserem Aufbau $g = \ObjectDistance$.
    Hier zu wurde der Abstand zwischen der Mitte der Linse und der effektiven Position des Senders nach~\cite{pasco} gemessen.
    
    Nun wird der Empfänger auf der anderen Seite positioniert.
    Für verschiedene Abstände zur Linse wir die Intensität gemessen.
    Es muss auch hier die effektive Position des Empfängers nach~\cite{pasco} berücksichtigt werden.
    Der Intensitätsverlauf ist in Abbildung~\ref{fig:imageDistance} dargestellt.
    
    Da die Wellenlänge der Mikrowellen im cm-Bereich liegt, liegt auch der Durchmesser 
    des resultierenden Airy-Scheibchens hinter der Linse auch im cm-Bereich.
    Das bedeutet, dass es keine Entfernung gibt, in der das Objekt vollständig fokussiert ist, wie man es in der
    geometrischen Optik erwarten würde. 
    Um die tatsächliche Bildentfernung zu ermitteln, muss man also das Maximum finden.
    
    Wir verwenden auch hier wieder eine generische Gauss-Kurve, um dass Maximum abzuschätzen.
    Denn neben der oben beschrieben Verzerrung der Daten durch die Scotty-Diode, ist in dieser Messung ein
    starkes Rauschen vorhanden.
    Hinter der Linse bildet sich eine stehende Welle aus, welche den Verlauf überlagert.
    
%    Here the image distance is 
    
    Die über den Fit bestimmte Bildweite beträgt:
    
    \begin{align*}
        b = \ImageDistance
    \end{align*}

    Auch hier ist der Fehler aus der statistischen Streuung der Datenpunkte um den Fit bestimmt worden.
    Er spiegelt nicht die systematischen Fehler wieder.
%    Now we can calculate the focal length after the lens formula
    Mit der Bildweite können wir nun die Brennweite berechnen nach der Abbildungsgleichung:
    
    \begin{align}
        \label{eq:LensFormula}
        \frac{1}{f} = \frac{1}{b} + \frac{1}{g}
    \end{align}
%    We have determined the focal length of the lens in our setup as:
    Die Brennweite der von uns genutzten Linse ist:
    \begin{align*}
        f = \FocalLength
    \end{align*}
    
    \section{Wellenlängen Messung}
    \subsection{Stehende Welle}
%    For the first approach we create stationary waves.
%    We position a mirror opposite of the emitter in a distance of $1m$.
%    Then we move the receiver between mirror and diode and count the nodes of the resulting standing wave.
%    We then measure the distance around 40 nodes and can calculate the wavelength:
    
    
    
    Für den ersten Ansatz erzeugen wir stehende Wellen. 
    Wir positionieren einen Spiegel gegenüber dem Sender in einem Abstand von $1\, \mathrm{m}$. 
    Dann bewegen wir den Empfänger zwischen Spiegel und Diode und zählen die Knotenpunkte der entstehenden stehenden Welle. 
    Wir messen dann den Abstand um 40 Knoten und können die Wellenlänge berechnen:
    \begin{align*}
        \lambda = \StandingWavesWaveLength
    \end{align*}
    \subsection{Michelson interferometer}

    \begin{figure}[htbp]
        \includegraphics[width=0.9\linewidth]{Interferometer}
        \center
        \caption{By varing the optical path length of the rays you see constructive or destructive interference on the
        detector. This depends on the wavelength of the ligth, which enables you to measure it.
        Graphic from~\cite{imageMichelsonInterferometerWiki}}
        \label{fig:Interferometer}
    \end{figure}
    
    As the second approach we use an interferometer.
    The emitted light is split by a fiberboard as semi-transparent mirror in two perpendicular rays.
    They are then reflected by two metal sheets and reunited again by the fiberboard. 
    The resulting ray interferes constructively or destructively depending on the difference between the two optical paths .
    We move the position of one mirror to change the optical path.
    After seeing 15 maxima and minima on the detector we read the changed
    path length.
    With our setup we have determined the wavelength as:
    \begin{align*}
        \lambda = \InterferometerWaveLength
    \end{align*}
    \section{Polarisation}

    \figPolarisation{Transmitted intensity through a polarisation filter.
    Though the shape of the curve fits, there is a strong deviation from the theory by Malus ~\eqref{eq:cos4}}
    
    The Gunn diode with the horn antenna in our setup emits vertical polarized light.
    Also the receiver detects in default configuration only vertical polarized light.
    This can be easily shown by rotating the receiver about $ 90\degree$.
    Now no signal is being detected.
    If you then insert a linear polarisation filter in a $ 45\degree$ angle you will detect a signal.
    This follows because the vertical polarized E-field is projected on the polarisation filter.
    For the polarisation filter a simple metal grid was used with 22 lattice bars.

    As a quantitative assertion we arranged the receiver and emitter in vertical position and rotated the polarisation
    filter in an $ 180\degree $-arc and measured the angle-intensity distribution.
    Following Malus's law~\cite{gerth} the intensity after the filter is:
    \begin{align}
        I = I_0 \cos^2(\phi)
    \end{align}
    Since the metal grid and the receiver act as two polarisation filter our signal should follow the formula:
    \begin{align}
        \label{eq:cos4}
        I = I_0 \cos^4( \phi )
    \end{align}
    If we compare the theoretical function to the measured data as in ~\ref{fig:Polarisation} we see a strong deviation.
    Multiple factors could be responsible for this result.
    Firstly the polarisation filter used, is a lattice consisting of 22 metal bars. 
    The relatively low count of bars makes it not an ideal linear polarisation filter.
    Furthermore signal loss over the distance may cause the deviation.
    Lastly it could be because of the cutoff from the receiver.
    Maybe too weak signals are being cutoff and so the curve is deformed.
    To verify these hypotheses more measurements are required.
    
    \section{Total inner reflection and tunnel effect}
    
    \begin{figure}[htbp]
        \includegraphics[width=0.9\linewidth]{setup-total-reflection}
        \centering
        \caption{Experimental setup for showing total reflection and tunnel effect.
        In the gap between the the double prism a evanescent wave is formed.
        The itensity declines exponentaly according to equation ~\eqref{eq:ExpTransDecline}}
        \label{fig:SetupTotalReflection}
    \end{figure}

    \figTotalReflection{We measure the transmitted and reflected signal at a double prism.
    The transmitted signal declines exponentially. This can be explained by the tunnel effect.}
    
    An important property of light which we were able to show, is total reflection and evanescent waves.
    For the experiment we emitted microwaves on a paraffin wax prism. 
    On the $45\degree$ paraffin-air boundary the rays are totally reflected and we detect them in an $90\degree$ rotated
    angle. 
    After the paraffin-air boundary an evanescent wave is formed.
    Its intensity declines exponentially.
    \begin{align}
        \label{eq:ExpTransDecline}
        I_{trans} = I_0 e^{\frac{-2x}{s}} 
    \end{align}
    Here $x$ is the perpendicular distance and $s$ is the declination rate which depends on the wavelength, 
    the angle of reflection and the refractive index.
    Because of energy conservation, the reflected intensity is:
    \begin{align*}
        I_{ref} = I_0 (1 - e^{\frac{-2x}{s}})
    \end{align*}
    

    You can detect the evanescent wave by a adding a second prism after the $ 45\degree$ boundary parallel to the first one
    in the distance $x$.
    Now analogous to the quantenmechanical tunnel effect\cite{gerth} the light can propagate through the air gap and a signal can be
    detected.
    We have measured the reflected and transmitted microwaves for different gap sizes $x$ and then applied an exponential 
    fit as you can see in~\ref{fig:TotalReflection}
    

    \section{Bragg Reflection}
    If you have monochromatic waves with a wavelength around the atomic scale (x-ray, neutrons or electrons) you can
    determine the lattice constant $d$ of a cubic crystal via bragg diffraction. 
    In Bragg's model the crystal is a set of discrete planes, where waves are being reflected.
    If the waves have the correct wavelengths and hit the planes in the correct angle, the reflected waves interfere 
    constructively and you see a bragg peak as a result~\cite{gerth}.
    
    The crystal planes can be indexed via the parameters $(h,\,k,\,l)$ .
    \begin{figure}[htbp]
        \centering
        \begin{subfigure}{0.49\linewidth}
            \centering
            \includegraphics[width = \linewidth]{Miller_Indizes_Ebenen_100}
        \end{subfigure}
        \begin{subfigure}{0.49\linewidth}
            \centering
            \includegraphics[width = \linewidth]{Miller_Indizes_Ebenen_110}
        \end{subfigure}
        \caption{Bragg-planes for a cubic crystal model with indices $(h,\,k,\,l)$.
        Images from ~\cite{miller}}
        \label{fig:BraggPlanesIndices}
    \end{figure}

    \figBraggReflectionOneZeroZero{Bragg diffraction on the 100-plane of a cubic lattice model.
    A peak was detected at the first bragg-maxima}
    \figBraggReflectionOneOneZero{Bragg diffraction on the 110-plane of a cubic lattice model.
    Here the bragg-peak occurs at a greater angle compared to the 100-palne}

    For the angle $\Theta$, for which a peak occurs, the following relation applies:
    \begin{align}
        \label{eq:BraggAngle}
        \frac{d}{\sqrt{h^2+k^2+l^2}} = \frac{\lambda}{2 \sin(\Theta)}
    \end{align}
    
    To demonstrate the principle, we use a basic crystal model, consisting of metal bars arranged in a grid structure.
    For examining real crystals, it would require electromagnetic waves with much smaller wavelengths.
    We realize the experiment by placing the emitter in the focal point of the wax lens, so the crystal is hit by
    parallel light-rays.
    The crystal model is then rotated about $\Theta$ and the detector about $2\Theta$, to actualize the Bragg condition.
    
    We have done two measurement-series: For the plane-orientation $(h=1,\,k=0,\,l=0)$ and for $(h=1,\,k=1,\,l=0)$.
    As you can see in figure~\ref{fig:BraggReflectionOneZeroZero} and figure~\ref{fig:BraggReflectionOneOneZero}.
    If we use equation~\eqref{eq:BraggAngle} to calculate the lattice constant from our measurement-series, we get:
    \begin{align*}
        d_{100} = \CristalconstantOneZeroZero \\
        d_{110} = \CristalconstantOneOneZero
    \end{align*}
    
    \section{Waveguide}
    
    Lastly we examined a simple waveguide for microwaves.
    We used a basic metal tube.
    The waves are being reflected in its interior and thus guided by the tube.
    We placed the emitter at one end of the tube and the receiver at the other.
    A signal is being detected.
    If you plug one end, no signal is being detected.
    This proves, the waves are guided inside of the tube and not on its surface.
    Also the tube preserves the polarisation of the emitted light.
    This was shown by simply rotating the receiver, since it only detects v-polarized light.

    \section{Summary}
    
    At first we tested our experimental setup.
    The Gunn diode has a gaussian angle-intensity distribution with a variance $fwdh = \AngleDispersionGaussFWHM$.
    To focus the microwaves we used a wax lens with a focal length $f = \FocalLength$.
    Then we used two approaches to measure the wavelength. 
    Via stationary waves we get a wavelength $\lambda = \StandingWavesWaveLength$.
    The michelson interferometer produces a similar result $\lambda = \InterferometerWaveLength $.
    We could show the polarization of the emitted light and at least qualitatively verify Malus' law. 
    The effect of total reflection and tunnel effect was demonstrated by using a double prism.
    The intensity of the evanescent waves declines exponentially.
    We used a crystal model to show the method of bragg diffraction.
    The lattice constant was determined to be $d_{100} = \CristalconstantOneZeroZero$.
    At last we used a metal tube as waveguide.
    We demonstrated qualitatively it preserves the polarisation of the guided light.
    
    
    %FF: Angabe der verwendeten Literatur mit Quellennachweis.
    \begin{thebibliography}{}    %so wird das Literaturverzeichnis erstellt
        \bibitem{instr} Physikalisches Grundpraktikum, Universitüt Würzburg, Modul C1, Versuch 42, Versuche mit Mikrowellen - Kristallinterferenzen mit Mikrowellen, 2021
        \bibitem{gerth} Meschede, Dieter, Gerthsen Physik, 25. Auflage, Springer-Verlag, Berlin, 2015
        \bibitem{codata} P. J. Mohr, D. B. Newell, and B. N. Taylor: \grqq CODATA
        recommended values of the fundamental physical constants: 2014\grqq , Rev. Mod. Phys.
        88, 035009 (2016))
        \bibitem{miller} \url{https://de.wikipedia.org/wiki/Datei:Miller_Indizes_Ebenen.png}, zuletzt aufgerufen am 14.09.2021
        \bibitem{pasco} Pasco Microwave Optics System (WA-9314C) Instruction Manual
        %\bibitem{missing} \url{https://en.wikipedia.org/wiki/Transverse_mode}, zuletzt aufgerufen am 8.10.2021
        \bibitem{imageMichelsonInterferometerWiki} \url{https://en.wikipedia.org/wiki/Michelson_interferometer#/media/File:Michelson_interferometer_with_labels.svg}, last visit 23.09.23
    \end{thebibliography}
    
\end{document}