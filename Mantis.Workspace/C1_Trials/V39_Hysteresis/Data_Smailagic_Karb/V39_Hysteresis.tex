%! Author = User
%! Date = 13.09.2023

% Preamble
\documentclass[a4paper,10pt,twocolumn]{article}

% Packages 
\usepackage[utf8]{inputenc}  %man kann Sonderzeiche wie ü,ö usw direkt eingeben
\usepackage{amsmath}           %macht
\usepackage{amsfonts}          %       Mathe
\usepackage{amssymb}           %              mächtiger
\usepackage{graphicx}          %erlaubt Graphiken einzubinden (.eps für dvi und ps sowie .jpg für pdf)
\usepackage[T1]{fontenc}       %Zeichenbelegung der verwendeten Schrift
\usepackage{ae}                %macht schöneres ß
\usepackage{typearea}
\usepackage{amstex}
\usepackage{siunitx}
\usepackage{hyperref}	         %ermöglicht änderung des Seitenspiegels
\usepackage{subcaption}


\usepackage{amsmath}
\usepackage{tikz}
\usepackage{pgfplots}

\newcommand{\alphaNoError}{(4.047 \pm 0.036)}
\newcommand{\betaNoError}{(-4.73 \pm 0.29) \cdot 10^{-3}}
\newcommand{\halfTimeNoError}{(146.5 \pm 9.1)\ s}
\newcommand{\alphaGauss}{(4.04 \pm 0.10)}
\newcommand{\betaGauss}{(-4.62 \pm 0.95) \cdot 10^{-3}}
\newcommand{\halfTimeGauss}{(150 \pm 31)\ s}
\newcommand{\alphaPoisson}{(4.05 \pm 0.10)}
\newcommand{\betaPoisson}{(-4.75 \pm 0.95) \cdot 10^{-3}}
\newcommand{\halfTimePoisson}{(146 \pm 29)\ s}
\newcommand{\symN}{\delta N}



\pagestyle{scrheadings}        %sagt Koma-Skript, dass selbstdefiniers Kopfzeilen verwendet werden
\typearea{16}                  %stellt Seitenspiegel ein
\columnsep25pt								 %definiert Breite zwischen den zwei Spalten von \twocolumns

\renewcommand{\pnumfont}{%     %ändert die Schriftart der Seitennummerierung
    \normalfont\rmfamily\slshape}  %ändert die Schriftart der Seitennummerierung 



\begin{document}
    \twocolumn[{\csname @twocolumnfalse\endcsname                %erlaubt "Abstrakt" über beide Spalten
    \titlehead{                                                  %Kopfzeile
        \begin{tabular*}{\textwidth}[]{@{\extracolsep{\fill}}lr}   %Kopfzeile
            Tutor: ? & \today\\                          %Kopfzeile - Betreuer
        \end{tabular*}                                             %Kopfzeile
    }
    \title{Magnetic hysteresis curves and characteristic magnetic properties of Fe-Ni-Alloy and Steel}  %Titel der Versuchs
    \author{Salahudin Smailagić and Thomas Karb}                     %Namen der Studenten
    \date{}                                                         %benötigt um automatisches Datum auszuschalten
    \maketitle                                                      %erzeugt Titelseite
    \vspace{-5ex}                                                   %verringert Abstand zur Überschrift
    \begin{abstract}                                                %Beginn des Abstracts
        
        
        
        \\
        Measuerement made: 21. September 2023\\       %Datum ändern!
        Submitted: 26. September 2023                %Datum ändern!
        \\
        \\
    \end{abstract}
    }] 
    \section{Introduction}
    Ferromagnetic materials have a high magnetic permeability.
    In paramagnets and diamagnets the resulting magnetic field (B-Field) directly depends on the
    extern magnetic field applied (H-Field).
    In contrast in ferromagnets the resulting B-Field is also depending on the magnetisation history of the material.
    The relation of H-field and B-field is described by the hysteresis curve.
    This phenomena enables permanent magnets.
    It enables to magnetize and demagnetize materials such as iron.
    This characteristic relation described by the hysteresis is important for applications in transformers,
    electric-motors, etc.
    In this paper we examine the hysteresis curve of two specific Fe-Ni alloys and steel Sk 732.
    
    \section{Theory}
    
    The magnetic properties of a material are caused by the spins of the electrons.
    In an ferromagnetic material the individual spins of the electrons do not cancel each other out, so every atom has an
    resulting spin.
    But every spin creates a magnetic dipole moment.
    Meaning every atom carries a magnetic dipole moment.
    In a ferromagnetic material the spin of neighboring atoms couple,
    creating domains where the atomic spins are aligned. 
    Those domains also called Weiss domains are separated by small boundaries, only a few atoms wide.
    In the 'unmagnetized' state the magnetic dipole moments of each Weiss domain, cancel each other out, resulting in
    a magnetic field of zero.
    If you now apply an external magnetic field the boundaries of the Weiss domains move, so the magnetic moments of the
    domains align with the direction of the applied H-field.
    So the H-Field gets amplified by a large amount.
    
    But the Weiss domains alone, would not explain permanent magnets and hysteresis. 
    Since in an perfect isotropic cristal, if you would remove the external H-field, the boundaries of the Weiss domains
    would move back to the initial position of lowest energy. 
    Thus the material would be 'unmagnetized' again.
    
    A real material on the other hand is anisotropic.
    It has defects on the cristal lattice.
    After removing the external H-Field, the boundary walls get pinned on those defects.
    So the Weiss domains can not go back in their initial configuration, thus the material stays magnetized.
    This effect results in the hysteresis curve.
    
    You can release the walls of their pinned state by heating the material, by causing vibrations through e.g.: hammering,  
    or by applying an oscillating external B-field as shown in section ~\ref{subsec:steel}.
    
    The hysteresis curve is characterized by a multitude of properties:
    If you apply the H-field the weiss domains will align.
    Thus amplifying the B-field. 
    By increasing the H-field even more, all dipole moments will be aligned and there is no amplification anymore.
    This saturation state is characterized by the threshold $B_{sat}$.
    After wards the B-field only increases with the vacuum permeability $\mu_0$.
    Now after removing the external H-field, the remaining B-field is called remanence $B_{R}$.
    The coercivity $H_C$ is the magnetic force (H-field) the material can withstand, without being demagnetized.
    
    Moving the weiss domains, costs energy.
    This energy loss, when cycling the hysteresis curve, can be calculated with the material density $\rho$:
    
    \begin{align}
        \label{eq:EnergyLoss}
        \xi = \frac{E}{m} = \frac{1}{\rho} \oint{B dH}
    \end{align}
    
    This is the area under the hysteresis curve.
    
    \section{Experimental setup}
    
    \section{Hysteresis curves}
    \subsection{Steel ST37K}
    \label{subsec:steel}
    \subsection{PERMENORM 5000 H2 Fe-Ni-Aloy}
    \subsection{PERMENORM 5000 Z Fe-Ni-Aloy}
    \subsection{Wood as reference}
    \section{summary}
    
    
    %FF: Angabe der verwendeten Literatur mit Quellennachweis.
    \begin{thebibliography}{}    %so wird das Literaturverzeichnis erstellt
        
    \end{thebibliography}
    
\end{document}