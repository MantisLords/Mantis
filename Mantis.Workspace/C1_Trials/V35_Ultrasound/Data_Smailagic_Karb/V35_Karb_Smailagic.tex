%! Author = User
%! Date = 13.09.2023

% Preamble
\documentclass[a4paper,10pt,twocolumn]{article}

% Packages 
\usepackage[utf8]{inputenc}  %man kann Sonderzeiche wie ü,ö usw direkt eingeben
\usepackage{amsmath}           %macht
\usepackage{amsfonts}          %       Mathe
\usepackage{amssymb}           %              mächtiger
\usepackage{graphicx}          %erlaubt Graphiken einzubinden (.eps für dvi und ps sowie .jpg für pdf)
\usepackage[T1]{fontenc}       %Zeichenbelegung der verwendeten Schrift
\usepackage{ae}                %macht schöneres ß
\usepackage{typearea}
\usepackage{siunitx}
\usepackage{mathtools}
\usepackage{hyperref}
\usepackage{hhline}
\usepackage{caption}
\usepackage{biblatex}


\captionsetup{font=footnotesize}



\usepackage{amsmath}
\usepackage{tikz}
\usepackage{pgfplots}

\newcommand{\alphaNoError}{(4.047 \pm 0.036)}
\newcommand{\betaNoError}{(-4.73 \pm 0.29) \cdot 10^{-3}}
\newcommand{\halfTimeNoError}{(146.5 \pm 9.1)\ s}
\newcommand{\alphaGauss}{(4.04 \pm 0.10)}
\newcommand{\betaGauss}{(-4.62 \pm 0.95) \cdot 10^{-3}}
\newcommand{\halfTimeGauss}{(150 \pm 31)\ s}
\newcommand{\alphaPoisson}{(4.05 \pm 0.10)}
\newcommand{\betaPoisson}{(-4.75 \pm 0.95) \cdot 10^{-3}}
\newcommand{\halfTimePoisson}{(146 \pm 29)\ s}
\newcommand{\symN}{\delta N}


\pagestyle{scrheadings}        %sagt Koma-Skript, dass selbstdefiniers Kopfzeilen verwendet werden
\typearea{16}                  %stellt Seitenspiegel ein
\columnsep25pt								 %definiert Breite zwischen den zwei Spalten von \twocolumns

\renewcommand{\pnumfont}{%     %ändert die Schriftart der Seitennummerierung
    \normalfont\rmfamily\slshape}  %ändert die Schriftart der Seitennummerierung 



\begin{document}
    \twocolumn[{\csname @twocolumnfalse\endcsname                %erlaubt "Abstrakt" über beide Spalten
    \titlehead{                                                  %Kopfzeile
        \begin{tabular*}{\textwidth}[]{@{\extracolsep{\fill}}lr}   %Kopfzeile
            Tutor: Dr. Tobias Kiessling & \today\\                          %Kopfzeile - Betreuer
        \end{tabular*}                                             %Kopfzeile
    }
    \title{Non invasive material analysis and ultrasound properties in different materials}  %Titel der Versuchs
    \author{Thomas Karb and Salahudin Smailagić}                     %Namen der Studenten
    \date{}                                                         %benötigt um automatisches Datum auszuschalten
    \maketitle                                                      %erzeugt Titelseite
    \vspace{-5ex}                                                   %verringert Abstand zur Überschrift
    \begin{abstract}%Beginn des Abstracts
        
        
        \\

        Measurement made: 14th March 2024 \\
        Submitted: 21th March 2024  \\

        \\
    \end{abstract}
    }]
    
    
    
    \section{Introduction}\label{sec:introdction}
    Ultrasound are mechanical waves with a frequency beyond $f = 20\,$kHz.
    Since the coupling between the atoms and molecules is highly dependent on the type of material, their propagation speed varies in different materials.
    The speed of the waves can also depend on the orientation of the crystal structure of the material, since the underlying symmetry also plays a role in the wave dispersion properties.
    On top of that, they can form longitudinal and transverse waves in the material, which generally have different dispersion properties.
    The goal of this paper is to understand the properties of ultrasound waves in different materials, while trying to see where the limitations of this method lie.
    
    
    
    \section{Basic functions of the echoscope}\label{sec:BasicFunctions}
    An echoscope is a ultrasonic measurement system which uses ultrasound probes, commonly made of a piezocrystal, connected to a computer or oscilloscope.
    The piezocrytal converts a voltage into mechanical vibration and vice versa.
    Therefore the probe can be used as an emitter and receiver of ultrasound.
    Naturally, there are two modes in which the echoscope can work.
    Either in reflection mode, where a single probe functions as an emitter and a receiver, or in transmission mode, with two different probes at either side of the examined material.
    
    \subsection{Runtime measurement}\label{subsec:RuntimeMeasurement}
    To determine the propagation velocities of ultrasound in different materials, we use the echoscope in reflection mode and record the time difference between the 
    rising edges of the initial and the reflected pulse.
    Using that time difference and the measured length of the used materials, the velocity can be determined.
    \begin{equation}
        v_s = \frac{2d}{t}
    \end{equation}
    This was done for example cylinders of different lengths made of polyacryl and PVC .
    The frequency of the ultrasound was varied, in order to test the materials on the frequency dependence of the propagation velocity.

    \begin{table}[htbp]          %so funktionieren die Tabellen in LaTeX
        \centering
        \begin{tabular*}{0.9\linewidth}{@{\extracolsep{\fill}}ccc}
            \hline
            \hline
            \rule[-7pt]{0pt}{23pt} Medium & Frequency  &  v_s 	 \\
            \hline
            \rule[-5pt]{0pt}{23pt}   Polyacryl & 1   &  \polyVelocityRuntimeMeasurementsOneMHz   	 \\
            \rule[-5pt]{0pt}{23pt}   Polyacryl & 2   & \polyVelocityRuntimeMeasurementsTwoMHz    	 \\
            \rule[-5pt]{0pt}{23pt}   PVC & 1  &   \PVCVelocityRuntimeMeasurementsOneMHz  	 \\
            \rule[-5pt]{0pt}{23pt}   PVC  & 2 &   \PVCVelocityRuntimeMeasurementsTwoMHz 	 \\
            \hline
            \hline
        \end{tabular*}
        \normalsize
        \caption[]{Propagation velocities of sound in different materials and different frequencies.
        The \textit{Echoskop GS200} was used in reflection mode for the measurements. A comparison with literature values is unreasonable, since
        the exact composition of the polyacryl as well as the PVC is not known.}  %siehe Graphik: Beschriftung
        \label{tab:FittedExponent}                             %siehe Graphik: zum Zitieren
    \end{table}
    From the determined velocities we are not able to say with certainty, if the frequency dependence exists.
    This is due to the fact, that the values coincide within the errors.
    For a more sophisticated statement, one would need to a broader frequency range.
    Comparing the determined velocities with literature values is unreasonable, since the exact composition of the used materials is not known.
    
    \subsection{Spectral measurement techniques}\label{subsec:Cepstrum}
    Having to determine where exactly the rising edge of a peak lies introduces some errors in the measurements.
    Having a measurement technique which does not rely on determining where the rising edge of a peak is, would generally be more accurate.
    One of these methods is the Spectrum function on the \textit{Echoskop GS200}, which implements a fast fourier transform (FFT) of the signal.
    From the FFT, the periodic reflections of the signal can easily be determined.
    \\
    The second method - the Cepstrum which also makes use of the fourier transform to determine periodicities in the signal.
    Both these methods are much more accurate than a simple runtime measurement, since they take more data-points into account.
    In order to quantitatively compare these measurement techniques, we measured the width of a polyacryl block.
    The results can be seen in \autoref{tab:SpectralMethods}.
    \begin{table}[htbp]          %so funktionieren die Tabellen in LaTeX
        \centering
        \begin{tabular*}{0.9\linewidth}{@{\extracolsep{\fill}}cc}
            \hline
            \hline
            \rule[-7pt]{0pt}{23pt} Method & Width [mm]  	 \\
            \hline
            \rule[-5pt]{0pt}{23pt}   Ruler & $10.00 \pm 0.03$    \\
            \rule[-5pt]{0pt}{23pt}   Runtime & $10.16 \pm 0.25$  \\
            \rule[-5pt]{0pt}{23pt}   Spectrum & $9.98 \pm 0.21$  \\
            \rule[-5pt]{0pt}{23pt}   Cepstrum  & $9.97 \pm 0.32$ \\
            \hline
            \hline
        \end{tabular*}
        \normalsize
        \caption[]{Comparison of different measurement techniques used to determine the width of a polyacryl cylinder.
        The ultrasound probe was used at $2\,$MHz and the signal was recorded on a \textit{Echoskop GS200}.}  %siehe Graphik: Beschriftung
        \label{tab:SpectralMethods}                             %siehe Graphik: zum Zitieren
    \end{table}
    
    \subsection{Speed of sound in water - two measurement techniques}\label{subsec:SpeedinWater}
    Using the same setup as before, we can also determine the speed of sound in water with a runtime measurement.
    For that, we use a water bath with a movable barrier placed inside just like in \autoref{fig:AufbauWasser}.
    We set the barrier (referred to as test block in the paper) perpendicular to the ultrasound waves and measure the time it takes for the reflected pulse to come back.
    This was done for multiple distances, by moving the barrier inside the water.
    The recorded distance-time data-points were then used to fit a line function through them.
    The incline of the fitted line corresponds to the velocity.
    The determined value for the propagation speed of sound waves in water is:
    \begin{equation}
        v_s = \WaterVelocity
    \end{equation}
    This value does not coincide with the literature value of $1485\,\frac{m}{s}$ at room temperature [2] within the margin of error.
    A possible explanation for this could be that chemicals dissolved in the tap water influence the sound propagation properties.
    Furthermore, the water bath was not completely full, hence the ultrasound probe could not be submerged completely.
    This may have lead to a part of the ultrasound waves forming surface waves which may have different dispersion properties.
    \\
    A much better measurement technique is the one using the \textit{Debye-Sears} effect, which works in the following way.
    If we continuously send ultrasound waves in a closed water container and let them reflect of the other edge of the container, they will form standing waves inside the water.
    The differences in density at the nodes and the maxima of the standing waves act as a periodic lattice which makes incoming light diffract.
    For the n-th diffraction order displaces by a length of $x_n$ from the 0-th order and $d>>x_n$ we get:
    \begin{equation}\label{eq:diffraction}
        sin(\alpha_n) = n \frac{\lambda_L}{\Lambda_s}
    \end{equation}
    and for $sin(\alpha_n) = \frac{x_n}{d}$ we get the propagation velocity:
    \begin{equation}\label{eq:Velocitydebye}
        v_s = n \frac{d}{x_n} f_s \lambda_L
    \end{equation}
    Doing this measurement for different laser wavelengths and different ultrasound frequencies, one can determine the propagation velocity of ultrasound in water.
    Three different lasers ($ \lambda_{Blue} = 405\,$nm ,$ \lambda_{Green} = 532\,$nm, $ \lambda_{Red} = 650\,$nm) were used for each of the four ultrasound frequencies ($f=3MHz$, $f=6MHz$, $f=9MHz$, $f=12MHz$).
    The end result is obtained as the weighted mean of all measurements.
    \begin{equation}\label{eq:VelocityDebyeValue}
        v_{Debye-Sears} = \WatervelocityDebye
    \end{equation}
    This value coincides with the literature value, and will be used in all further calculations which include the speed of sound in water.
    
    \subsection{Sound propagation velocities in metals}\label{propagationInMetals}
    \begin{figure}
        \begin{center}
        \includegraphics[width = \linewidth]{Generated/AufbauWasser}
            \caption[]{}
            \label{fig:AufbauWasser}
        \end{center}
    \end{figure}    
    Our way of doing the runtime measurement requires us being able to differentiate the two peaks of the signal.
    Naturally, we would have a hard time analyzing very thin materials since the two peaks in the signal would overlap considerably.
    On top of that, 
    In order to measure the propagation velocity for these types of materials, we use a different setup.
    This setup, which can be seen in \autoref{fig:AufbauWasser}, uses the physics of wave refraction.
    For that, a block of copper or aluminum is placed in a water bath on a rotating rig.
    The echoscope is set to transparency mode and the two probes are placed on the opposite ends of the bath.
    By varying the angle of the blocks relative to the probes, we can determine the propagation velocity of the waves in the material.
    By knowing the angle of the incoming and the refracted wave \textit{Snells} law can be used to determine the velocities.
    \begin{equation}\label{eg:SnellLaw}
        \frac{sin(\beta)}{sin(\alpha)} = \frac{v_s}{v_w}
    \end{equation}
    Where $v_w$ is the speed of sound in water and $v_s$ the speed of sound in the examined material.
    This method also allows a much more sophisticated setup for examining the dispersion properties of different directions inside the material.
    All metals have a specific periodic ordering of atoms.
    Depending on the symmetries of this ordering, as well as the types of bonds between the atoms, the crystal exhibits different propagation velocities of mechanical waves.
    By performing the measurement, we know that we expect three peaks - one for each mode.
    The first peak, corresponding to the fastest velocity, is that of the longitudinal mode.
    The second and third peak are the transverse modes.
    We know from theory, that for $\beta = 90^\circ$ there is total reflection for both the transverse and the longitudinal modes.
    So when the signal on the echoscope vanishes, we can approximate the refracted angle $\beta$ to be $90^\circ $, and use it to calculate $v_s$.
    Similarly, we know that transmission for transverse waves becomes maximal at $\beta = 45^\circ $.
    Using that, we can calculate another velocity value.
    The measured intensities for copper and aluminum are depicted in \autoref{fig:CopperPlot} and \autoref{fig:AluminumPlot} respectively.
    \begin{figure}
        \begin{center}
            \includegraphics[width = \linewidth]{Generated/LongTransPlotCopper}
            \caption[]{CopperPlot}
            \label{fig:CopperPlot}
        \end{center}
    \end{figure}
    \begin{figure}
        \begin{center}
            \includegraphics[width = \linewidth]{Generated/LongTransPlotAluminum}
            \caption[]{AluminumPlot}
            \label{fig:AluminumPlot}
        \end{center}
    \end{figure}
    From the angle values depicted in the figures, the propagation velocities of the different modes could be determined.
    It is important to mention, that we were not able to differentiate the second transverse peak from the first one.
    This is due to the fact that the peaks on the echoscope overlapped significantly, making it impossible to tell one from another.
    Nevertheless, the determined values seem to be around the same value as the theory would suggest \autoref{}.
    Keeping in mind, that the theoretically calculated values stem from the approximation that the bonds between the atoms behave according to \textit{Hooks} law.
    
    \begin{table}[htbp]          %so funktionieren die Tabellen in LaTeX
        \centering
        \begin{tabular*}{0.9\linewidth}{@{\extracolsep{\fill}}cccc}
            \hline
            \hline
            \rule[-7pt]{0pt}{23pt} Medium &Mode& $\beta^\circ$  &  $v_s$[$\frac{m}{s}$] 	 \\
            \hline
            \rule[-5pt]{0pt}{23pt}   Copper & longitudinal   &90&  \LongCopper   	 \\
            \rule[-5pt]{0pt}{23pt}   Copper & transverse & 45 & \TransCopper  	 \\
            \rule[-5pt]{0pt}{23pt}   Copper & transverse & 90 & \TransCopperOne  	 \\
            \rule[-5pt]{0pt}{23pt}   Aluminum & longitudinal & 90 & \LongAlu 	 \\
            \rule[-5pt]{0pt}{23pt}   Aluminum & transverse & 45 & \TransAlu 	 \\
            \rule[-5pt]{0pt}{23pt}   Aluminum & transverse & 90 & \TransAluOne 	 \\
            \hline
            \hline
        \end{tabular*}
        \normalsize
        \caption[]{Sound velocities for different modes determined using \textit{Snells} law and the setup from \autoref{fig:AufbauWasser}.
        The signal was recorded on the \textit{Echoskop GS200} with a ultrasound frequency of $2\,$MHz.}  %siehe Graphik: Beschriftung
        \label{tab:LongTransVelocities}                             %siehe Graphik: zum Zitieren
    \end{table}
    
    
    
    \section{Absorption of ultrasound}\label{sec:Absorbtion}
    When traveling through a material, ultrasound waves oscillate the atoms and molecules of the material.
    This oscillation leads to energy loss in the form of dissipated heat.
    So an initial ultrasound signal sent into one end of a material will get partially absorbed i.e. damped.
    In order to account for that in our measurement, we used the \textit{time gain control} (TGC) function of the echoscope,
    which amplifies signals after some time.
    This makes it possible to measure signals which would otherwise be to small while not overamplifying the visible signals.
    The dampening of the signal inside a material generally follows an exponential decay
    \begin{equation}\label{eg:Absorbtion}
        I(x) = e^{-\lambda x}
    \end{equation}
    where $\lambda$ is the material-specific dampening coefficient.
    By measuring the intensity of the ultrasound signal over a known distances x, the dampening coefficient can be calculated via
    \begin{equation}\label{eg:AbsCoeff}
        \lambda = -log(\frac{A_x}{A_0}) \frac{1}{2x}
    \end{equation}
    This mesurement was done for PVC for different frequencies. 
    The values can be seen in \autoref{tab:DampeningCoeff}.
    \begin{table}[htbp]          
        \centering
        \begin{tabular*}{0.9\linewidth}{@{\extracolsep{\fill}}cc}
            \hline
            \hline
            \rule[-7pt]{0pt}{23pt}  Frequency [MHz]  &  $\lambda \,$[cm^-1] 	 \\
            \hline
            \rule[-5pt]{0pt}{23pt}    1   &  \DampeningOneMhz 	 \\
            \rule[-5pt]{0pt}{23pt}    2   & \DampeningTwoMHz     \\
            \hline
            \hline
        \end{tabular*}
        \normalsize
        \caption[]{DampeningCoeff}  
        \label{tab:DampeningCoeff}                             
    \end{table}
    The absorption coefficient is larger for higher frequencies.
    This is due to the fact, that the higher frequency wave oscillates the atoms faster which leads to more heat radiation and ultimately more dissipation.
    The absorption for $4\,$MHz could not be determined because the reflected signal was to small to measure accurately.
    
    
    
    \section{Resolution of ultrasound imaging}\label{sec:Resolution}
    The echoscope works by sending finite length pulses through the material.
    Naturally, if these finite pulses overlapped to some extent, it would be hard to separate them in the measurement.
    This can pose a problem when trying to separate reflection sources which are to closer than the pulse width.
    A general mean of quantifying the resolution of devices is the \textit{Full width half maximum} or FWHM of the pulses.
    The criterium for distinguishing two peaks would be: \textit{If the peaks are seperated by more that the FWHM, they are distinguishable}.
    The FWHM of our used \textit{Echoskop GS200} was measured for three different frequencies and the results are depicted in \autoref{tab:FWHM}
    \begin{table}[htbp]
        \centering
        \begin{tabular*}{0.9\linewidth}{@{\extracolsep{\fill}}cc}
            \hline
            \hline
            \rule[-7pt]{0pt}{23pt}  Frequency [MHz]  &  FWHM [$\mu s$] 	 \\
            \hline
            \rule[-5pt]{0pt}{23pt}    1   &  $1.60 \pm 0.05 $	 \\
            \rule[-5pt]{0pt}{23pt}    2   &    $1.15 \pm 0.05$  \\
            \rule[-5pt]{0pt}{23pt}    4   &   $0.75 \pm 0.05$  \\
            \hline
            \hline
        \end{tabular*}
        \normalsize
        \caption[]{Full width half maximum values of ultrasound pulses at different frequencies.
        The measurement was done with the cursor difference in A-Scan mode on a \textit{Echoskop GS200}.
        The used material was a polyacryl test block which can be seen in \autoref{fig:Block}.
        It is visible from the data, that a higher frequency results in a better resolution.}
        \label{tab:FWHM}
    \end{table}
    
    As we already know from wave mechanics, a higher frequency i.e. a smaller wavelength results in a better resolution.
    So choosing a higher frequency for measurements that require a high resolution would be the right thing to do.
    However, higher frequencies also get absorbed more (\autoref{sec:Absorbtion}) leading to a less accurate measurement.
    A balance between these two effects is a must for reliable results.
    
    \section{Material analysis with ultrasound imaging}\label{sec:MaterialAnalysis}
    \begin{table}[htbp]
        \centering
        \begin{tabular*}{0.9\linewidth}{@{\extracolsep{\fill}}ccc}
            \hline
            \hline
            \rule[-7pt]{0pt}{23pt}  Bore & $d_{Ruler}$ & $d_{Ultrasound} $	 \\
            \hline
            \rule[-5pt]{0pt}{23pt}    1   &  $6.85 \pm 0.03 $	& \depthOne \\
            \rule[-5pt]{0pt}{23pt}    2   &  $14.86 \pm 0.03 $	& \depthTwo \\
            \rule[-5pt]{0pt}{23pt}    3   &  $22.75 \pm 0.03 $	& \depthThree \\
            \rule[-5pt]{0pt}{23pt}    4   &  $30.81 \pm 0.03 $	& \depthFour\\
            \rule[-5pt]{0pt}{23pt}    5   &  $38.80 \pm 0.03 $	& \depthFive \\
            \rule[-5pt]{0pt}{23pt}    6   &  $46.45 \pm 0.03 $	& \depthSix \\
            \rule[-5pt]{0pt}{23pt}    7   &  $53.79 \pm 0.03 $	& \depthSeven \\
            \rule[-5pt]{0pt}{23pt}    8   &  $61.40 \pm 0.03 $	& \depthEight \\
            \rule[-5pt]{0pt}{23pt}    9a   &  $17.40 \pm 0.03 $	& \depthNine \\
            \rule[-5pt]{0pt}{23pt}    9b   &  $19.05 \pm 0.03 $	& ($19.366 \pm 0.080$)\\
            \rule[-5pt]{0pt}{23pt}    10   &  $55.43 \pm 0.03 $	& \depthOneOne \\
            \rule[-5pt]{0pt}{23pt}    $\Delta_9$   &  $1.65 \pm 0.04 $	& $1.678 \pm 0.015$ \\
            \hline
            \hline
        \end{tabular*}
        \normalsize
        \caption[]{Depth measurement on a polyarcry testblock which can be seen in \autoref{fig:Block} at a ultrasound frequency of $2\,$MHz, compared to a measurement done with a ruler.
        The used propagation speed for polyacryl is taken from \autoref{subsec:RuntimeMeasurement} as \polyVelocityRuntimeMeasurementsOneMHz. }
        \label{tab:Depths}
    \end{table}
    Ultrasound allows us to gather information about the structure of a material without having to look inside or possibly destroy it.
    To demonstrate this in action, we use a polyacryl test-block which can be seen in \autoref{fig:Block}.
    The measurement is analogous to the runtime measurement from \autoref{subsec:RuntimeMeasurement}.
    In order to balance the effects of absorption while at the same time maintaining some good resolution, we used the echoscope at a frequency of $2\,$MHz.
    For reference, the determined depths of the bores were also measured with a ruler and the results are given in \autoref{tab:Depths}
    \begin{figure}
        \begin{center}
            \includegraphics[width = \linewidth]{Generated/Block}
            \caption[]{Schematic representation of the polyacryl testblock which is used for the measurements.
            The time it takes the wave to reflect on the boring was measured and converted into a depth.}
            \label{fig:Block}
        \end{center}
    \end{figure}
    
    The ultrasound measurement coincides with the ruler measurement within $90 \% $, except for the first boring which deviates slightly more.
    Of course, one would need to know exactly the properties of the material to make these measurements even more accurate.
    In order to determine the depths, the propagation velocity in polyacryl determined in \autoref{subsec:RuntimeMeasurement} was used.
    This may also be a source of error.
    
    \section{References}
    [1] https://www.physik.uni-wuerzburg.de/studium/ bachelor/grundpraktikum/ modulbeschreibungen/ fortgeschrittenenmodul-c1/
    \newline
    [2] Meschede, Dieter, gerthsen Physik, 25. Auflage, Springer-Verlag, Berlin 2015
    
    \section{Appendix}\label{sec:appendix}
    Mechanical waves are theoretically described by a set of differential equations:
    \begin{figure}
        \begin{center}
            \includegraphics[]{Generated/WaveEquations}
            \caption[]{}
            \label{eg:WaveEquations}
        \end{center}
    \end{figure}
    The dispersion relations are determined by plugging plane waves in the equation.
    In simple cubic crystal structures the Young modulus simplifies to three independent components.
    The Equation above delivers a set of linear equations:
    \begin{equation}
        M{\vec{A}} = 0
    \end{equation}
    Since for non-trivial Solutions, the eigenvalues vanish - we can determine the dispersion relations from the eigenvalues.
    \begin{equation}
        v_s = \omega / k
    \end{equation}
    The angle dependency of these velocities can be seen in \autoref{} and \autoref{} respectively.
    The components of the Young modulus are taken from the table given in [1].
    The anisotropy determined from [2] is:
    \begin{equation}
        
    \end{equt}
\end{document}