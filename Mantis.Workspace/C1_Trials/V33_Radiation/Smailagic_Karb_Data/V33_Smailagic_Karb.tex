%! Author = User
%! Date = 13.09.2023

% Preamble
\documentclass[a4paper,10pt,twocolumn]{article}

% Packages 
\usepackage[utf8]{inputenc}  %man kann Sonderzeiche wie ü,ö usw direkt eingeben
\usepackage{amsmath}           %macht
\usepackage{amsfonts}          %       Mathe
\usepackage{amssymb}           %              mächtiger
\usepackage{graphicx}          %erlaubt Graphiken einzubinden (.eps für dvi und ps sowie .jpg für pdf)
\usepackage[T1]{fontenc}       %Zeichenbelegung der verwendeten Schrift
\usepackage{ae}                %macht schöneres ß
\usepackage{typearea}
\usepackage{siunitx}
\usepackage{mathtools}
\usepackage{hyperref}
\usepackage{hhline}
\usepackage{caption}
\usepackage{biblatex}

\captionsetup{font=footnotesize}



\usepackage{amsmath}
\usepackage{tikz}
\usepackage{pgfplots}

\newcommand{\TemperatureHumidityCovariance}{102.3\ ^{\circ} C g / m^3}
\newcommand{\TemperatureHumidityCorrelation}{0.8571}
\newcommand{\RegressionOffset}{(1.8 \pm 1.2)\  g / m^3}
\newcommand{\RegressionSlope}{(0.343 \pm 0.057)\ g / m^3 / ^{\circ} C}


\pagestyle{scrheadings}        %sagt Koma-Skript, dass selbstdefiniers Kopfzeilen verwendet werden
\typearea{16}                  %stellt Seitenspiegel ein
\columnsep25pt								 %definiert Breite zwischen den zwei Spalten von \twocolumns

\renewcommand{\pnumfont}{%     %ändert die Schriftart der Seitennummerierung
    \normalfont\rmfamily\slshape}  %ändert die Schriftart der Seitennummerierung 



\begin{document}
    \twocolumn[{\csname @twocolumnfalse\endcsname                %erlaubt "Abstrakt" über beide Spalten
    \titlehead{                                                  %Kopfzeile
        \begin{tabular*}{\textwidth}[]{@{\extracolsep{\fill}}lr}   %Kopfzeile
            Tutor: Dr. Leonid Bovkun & \today\\                          %Kopfzeile - Betreuer
        \end{tabular*}                                             %Kopfzeile
    }
    \title{Simulating a perfect black body - verifying the Stefan Bolzmann Law}  %Titel der Versuchs
    \author{Salahudin Smailagić and Thomas Karb}                     %Namen der Studenten
    \date{}                                                         %benötigt um automatisches Datum auszuschalten
    \maketitle                                                      %erzeugt Titelseite
    \vspace{-5ex}                                                   %verringert Abstand zur Überschrift
    \begin{abstract}%Beginn des Abstracts
        
        This abstract is soooo abstract.
        
        
        \\
        \\
        Measuerement made: 8th March 2024 \\       %Datum ändern!
        Submitted: 15th March 2024  \\
        
        \\
    \end{abstract}
    }]
    \section{Introduction}\label{sec:introdction}
    The thermal properties of bodies have been of huge interest since the 19th century.
    A better understanding of these properties, as well as experimental means to measure them accurately have been developing well into the 20th century.
    In this paper we present a possible experimental setup which tries to simulate a black body with a hollow box which is being held at constant temperature.
    With this setup, we tried verify the $T^4$ dependence of the radiation according to \textit{Stefan-Boltzmann Law}.
    We are also able to use the setup to look at the angle dependence of the black-body radiation, which according to \textit{Lambert} should follow a cosine function.
    Since the surface of a radiating body influences its radiative properties, we used a setup which allowed for comparison of different surfaces under controlled conditions.
    For the experiment, we used a \textit{Leslie} cube with four different surfaces on 4 of its sides.
    The cube was mounted on a rotating rig so that the various sides could easily be adjusted.
    
    
    
    
    \section{Examination of Different Surfaces}\label{sec:ExaminationOfDifferentSurfaces}
    In order to examine the radiative properties of various surfaces, the experimental setup shown in \autoref{} was used.
    A black, a white, a nickel plated polished and a matte surface were available for selection. 
    The cube was kept at a controlled temperature by a thermostat from \textit{Lauda E100}.
    This temperature was increased from $ 289\,$K to $363\,$K.
    The emitted thermal radiation was measured with a thermophile.
    Due to the construction of the thermopile, which makes use of the Seebeck effect, a constant set-of temperature must be accounted for in the measurement.
    Since this set-of is not exactly the ambient temperature, but the temperature at which the voltage reading is $0$, we linearize the first few measured voltage values 
    and find the intersection with the x-Axis.
    This is done for all the different surfaces and the mean value was taken as the set-of temperature.
    \begin{equation}\label{SetofTemp}
        T_0 = 292.5 \pm 0.4
    \end{equation}
    With this set-of temperature accounted for, we were able to try to verify the $T^4$ proportionality of the radiation.
    As can be seen in \autoref{fig:FirstPlot} we fit the function from \autoref{FitFunction} into the recorded data.
    \begin{equation}\label{FitFunction}
        U = a\cdot (T^b - T_0^b)
    \end{equation}
    
    \begin{figure}
               \begin{center}
               \includegraphics[width=\linewidth]{Generated/StefanBolzmannPlot}
               \caption{}
               \label{fig:FirstPlot}
               \end{center}
    \end{figure}
    The fitted parameters in the exponent can be seen in \autoref{tab:FittedExponent}
    \begin{table}[htbp]          %so funktionieren die Tabellen in LaTeX
        \centering
        \begin{tabular*}{0.9\linewidth}{@{\extracolsep{\fill}}cc}
            \hline
            \hline
            \rule[-7pt]{0pt}{23pt}  Surface  &  Fitted exponent 	 \\
            \hline
            \rule[-5pt]{0pt}{23pt}   Polished nickel   &   \ExponentPolished  	 \\
            \rule[-5pt]{0pt}{23pt}   Matte nickel   & \ExponentMatt     	 \\
            \rule[-5pt]{0pt}{23pt}   Black  &   \ExponentBlack  	 \\
            \rule[-5pt]{0pt}{23pt}   White   &   \ExponentWhite  	 \\
            \hline
            \hline
        \end{tabular*}
        \normalsize
        \caption[]{These are the values of the exponent $b$ obtained by fitting $U = a\cdot(T^b - T_0^b) $ into the data from \autoref{fig:FirstPlot}.
        The constant set-of temperature $T_0$ due to the internal constuction of the used thermopile was accounted for.
        The voltage was measured using a \textit{Agilent 3305 A} multimeter.}  %siehe Graphik: Beschriftung
        \label{tab:FittedExponent}                             %siehe Graphik: zum Zitieren
    \end{table}
    \autoref{fig:FirstPlot} shows the voltage as a function of temperature for the four available surfaces.
    As can immediately be seen from the figure, the white and the black surfaces exhibit almost identical behaviour.
    This is due to the fact, that although the surfaces have different absorption in the visible spectrum, they behave like the same colour in the examined infrared range.
    The matte surface also showed similarly high emission, this is due to the larger surface area.
    The polished surface naturally has a much higher reflectivity and should emit much less.
    Apart from that, the measurement for the reflective surface was tricky because the reflection of the experimenter influenced the measurement.
    For that reason, the error of that measurement is much larger.
    Nonetheless, all the fitted parameters are well within the theoretical value of $4$.
    The matte and white surface do seem to agree the best, while the polished surface deviates more.
    It also needs to be mentioned that this the heat radiation from the environment has a significant influence on the measurement, introducing a systematical error in our evaluation.
    Since the measurement also takes some time, it is possible that the reference temperature of the thermopile changes as the ambient temperature rises.
    This is also a possible error which we did not account for.
    
    
    
    
    \section{Simulating a Black Body}\label{BlackBodyRadiation}
    In order to simulate a black body as well as possible, we set up a hollow chamber with a small opening.
    The opening is covered with a water cooled aperture to make the measurement even cleaner.
    The temperature of the chamber was continuously heated and the voltage reading corresponding to the radiation through the opening was recorded on an \textit{Agilent 34405A}.
    The same fitting method as in \autoref{sec:ExaminationOfDifferentSurfaces} was used to fit the data.
    From the measurement in \autoref{fig:BlackBodyPlot} the following exponent was determined:
    \begin{equation}
        b = \BlackBodyFitParameter
    \end{equation}
    The error of this value was estimated, since just using the fitting error does not take into account the uncertainties in measurement that we encountered.
    The device which we used to heat up the chamber was rather inert and we had a hard time recording the displayed temperature and the voltage reading at the same time.
    We estimated the error of the temperature reading as $\pm 1\,$K (last digit).
    Although the cooling aperture was used in the experiment, it might not have been enough to shield the measurement from all environmental radiation.
    Due to these effects we estimated the error to be $5\%$\).
    \begin{figure}
        \begin{center}
            \includegraphics[width=\linewidth]{Generated/Aufbau2}
            \caption{}
            \label{fig:Aufbau2}
        \end{center}
    \end{figure}
    \begin{figure}
        \begin{center}
            \includegraphics[width=\linewidth]{Generated/BlackBodyPlot}
            \caption{}
            \label{fig:BlackBodyPlot}
        \end{center}
    \end{figure}
    Taking these error influences into account, the determined exponent does not quite have the theoretically predicted value of $b = 4 $.
    Nonetheless, given that the value deviates from the theoretical value only slightly within the error bars we conclude that the chamber setup with the cooled aperture
    is a good model for a black body radiator.
    In order for the measurement to be in better agreement with the theoretical value, one would need to control the environmental influences better. 
    
    
    
    
    \section{Infrared transmission of different materials}\label{sec:Transmission}
    In this experiment, we look at the transmission properties of different materials in the infrared range.
    Again, the setup from \autoref{} was used, but now with the black surface of the leslie cube and different filters between the cube and the thermopile.
    The transmission value for each material was determined as the amount of radiation measured behind the filter relative to the radiation value with no filter.
    The calculated values are depicted in \autoref{tab:TransistivityValues}.
    \begin{table}[htbp]          %so funktionieren die Tabellen in LaTeX
        \centering
        \begin{tabular*}{0.9\linewidth}{@{\extracolsep{\fill}}cc}
            \hline
            \hline
            \rule[-7pt]{0pt}{23pt}  Filter  &  Transmission coefficient 	 \\
            \hline
            \rule[-5pt]{0pt}{23pt}   No Filter  &   \TransistivityValue 	 \\
            \rule[-5pt]{0pt}{23pt}   Nickel   & \TransistivityValueOne    	 \\
            \rule[-5pt]{0pt}{23pt}   Window glass  &   \TransistivityValueTwo 	 \\
            \rule[-5pt]{0pt}{23pt}   Silicon  &   \TransistivityValueThree  	 \\
            \rule[-5pt]{0pt}{23pt}   NaCl  &   \TransistivityValueFour 	 \\
            \rule[-5pt]{0pt}{23pt}   Infrared  &   \TransistivityValueFive  	 \\
            \hline
            \hline
        \end{tabular*}
        \normalsize
        \caption[]{The calculated transmission coefficients for different materials using the setup from \autoref{}.
        The black surface of the leslie cube was used at a temperature of $353\,$K. The ambient temperature was measured to be $293\,$K .
        The voltage was measured using a \textit{Agilent 3305 A} multimeter.}  %siehe Graphik: Beschriftung
        \label{tab:TransistivityValues}                             %siehe Graphik: zum Zitieren
    \end{table}
    
\end{document}