%! Author = User
%! Date = 13.09.2023

% Preamble
\documentclass[a4paper,10pt,twocolumn]{article}

% Packages 
\usepackage[utf8]{inputenc}  %man kann Sonderzeiche wie ü,ö usw direkt eingeben
\usepackage{amsmath}           %macht
\usepackage{amsfonts}          %       Mathe
\usepackage{graphicx}          %erlaubt Graphiken einzubinden (.eps für dvi und ps sowie .jpg für pdf)
\usepackage[T1]{fontenc}       %Zeichenbelegung der verwendeten Schrift
\usepackage{ae}                %macht schöneres ß
\usepackage{typearea}
\usepackage{amstex}
\usepackage{siunitx}
\usepackage{hyperref}	         %ermöglicht änderung des Seitenspiegels
\usepackage{subcaption}

\newcommand{\quaterWavePlate}{$\frac{\lambda}{4}$-plate }
\newcommand{\quaterWavePlates}{$\frac{\lambda}{4}$-plates }
\newcommand{\halfWavePlate}{$\frac{\lambda}{2}$-plate }
\newcommand{\halfWavePlates}{$\frac{\lambda}{2}$-plates }

\newcommand{\NSF}{N-SF6}

\newcommand{\VTwo}{[V]_{\fyWavelengthTwo}}
\newcommand{\VThree}{[V]_{\fyWavelengthThree}}

\newcommand{\unit}[1]{\, \mathrm{#1}}
\newcommand{\nm}{\unit{nm}}
\newcommand{\unitSpecrot}{\unit{\frac{\degree \, cm^2}{g}}}
\newcommand{\unitVerdet}{\unit{\frac{rad}{T m}}}
\newcommand{\wavelengthYellow}{580 \nm}
\newcommand{\specRot}{[\alpha]_{\wavelengthYellow}}


\usepackage{amsmath}
\usepackage{tikz}
\usepackage{pgfplots}

\newcommand{\alphaNoError}{(4.047 \pm 0.036)}
\newcommand{\betaNoError}{(-4.73 \pm 0.29) \cdot 10^{-3}}
\newcommand{\halfTimeNoError}{(146.5 \pm 9.1)\ s}
\newcommand{\alphaGauss}{(4.04 \pm 0.10)}
\newcommand{\betaGauss}{(-4.62 \pm 0.95) \cdot 10^{-3}}
\newcommand{\halfTimeGauss}{(150 \pm 31)\ s}
\newcommand{\alphaPoisson}{(4.05 \pm 0.10)}
\newcommand{\betaPoisson}{(-4.75 \pm 0.95) \cdot 10^{-3}}
\newcommand{\halfTimePoisson}{(146 \pm 29)\ s}
\newcommand{\symN}{\delta N}



\pagestyle{scrheadings}        %sagt Koma-Skript, dass selbstdefiniers Kopfzeilen verwendet werden
\typearea{16}                  %stellt Seitenspiegel ein
\columnsep25pt								 %definiert Breite zwischen den zwei Spalten von \twocolumns

\renewcommand{\pnumfont}{%     %ändert die Schriftart der Seitennummerierung
    \normalfont\rmfamily\slshape}  %ändert die Schriftart der Seitennummerierung 



\begin{document}
    \twocolumn[{\csname @twocolumnfalse\endcsname                %erlaubt "Abstrakt" über beide Spalten
    \titlehead{                                                  %Kopfzeile
        \begin{tabular*}{\textwidth}[]{@{\extracolsep{\fill}}lr}   %Kopfzeile
            Supervisour: Dr. Martin Stehno & \today\\                          %Kopfzeile - Betreuer
        \end{tabular*}                                             %Kopfzeile
    }
    \title{Methods to create and detect linear and circular polarized light;
    Two applications: Optical acitvity of sucrose; Faraday effect}  %Titel der Versuchs
    \author{Salahudin Smailagić and Thomas Karb}                     %Namen der Studenten
    \date{}                                                         %benötigt um automatisches Datum auszuschalten
    \maketitle                                                      %erzeugt Titelseite
    \vspace{-5ex}                                                   %verringert Abstand zur Überschrift
    \begin{abstract}                                                %Beginn des Abstracts
        We present a method for studying the fundamental polarization properties of electromagnetic waves in
        the visible spectrum.
        First we want to introduce a setup, with which we can create and observe linear and
        circular polarized light.
        We use conventional polarizing filters to create linear polarized light.
        By appling the Jones' formalism, we can derive Malus' law, which we
        then show experimentally.
        Furthermore, with \quaterWavePlates we produce circular polarized light.
        We demonstrate its properties qualitatively.
        
        In the second part, we showcase two exemplary applications.
        Since sugar molecules are chiral, they are optical active, meaning a
        sugar solutions rotates linear polarized light.
        We measure the specific optical rotation of sucrose $\specRot = \sugarSpecificOpticalRotation$
        with a half-shade polarimeter and then apply it for determining the concentration of 
        an unknown sucrose solution.
        
        Similar to optically activity of chiral molecules functions the Faraday effect.
        Here we apply a magnetic field at a \NSF-glas rod and shine linear polarized light
        through it.
        We gauge the space-resolved magnetic field with a hall sensor and then measure the rotated 
        polarization vector of the transmitted light with a linear polarizing filter.
        We determine the Verdet's constant for two wavelengths 
        $\VTwo = \verdetConstantTwo$ and $\VThree = \verdetConstantThree$.
        
        \\
        \\
        Measuerement made: March 13th 2023\\       %Datum ändern!
        Submitted: March 20th 2023                %Datum ändern!
        \\
        \\
    \end{abstract}
    }] 
    \section{Methods to create and detect polarized light}
    \subsection{Introduction}
    In classical physics light is described as an electromagnetic wave.
    The electric and magnetic fields oscillate perpendicular to each other and to the propagation direction.
    The polarization refers to the direction of the electric field.
    If the wave is linear polarized, the field oscillates in a fixed plane.
    If the wave is circular polarized, the field vector rotates around the propagation direction.
    
    The polarization of light can be described with the Jones' formalism~\cite{gerth}.
    Linear polarized light is denoted with a real vector, pointing in the polarization direction.
    For example vertical and horizontal polarized light:
    \begin{align*}
        \vec{J}_H = \begin{pmatrix} 1 \\ 0 \end{pmatrix} \quad
        \vec{J}_V = \begin{pmatrix} 0 \\ 1 \end{pmatrix}
    \end{align*}
    
    Furthermore, left and right-handed circular polarized light are described with a complex vector:

    \begin{align*}
        \vec{J}_L = \frac{1}{\sqrt {2}} \begin{pmatrix} 1 \\ -i \end{pmatrix} \quad
        \vec{J}_R = \frac{1}{\sqrt {2}} \begin{pmatrix} 1 \\ i \end{pmatrix}
    \end{align*}
    
    Optical devices such as polarizing filters and \quaterWavePlates are described with matrices, which act
    upon the polarization vector $\vec{J}$. 
    
    \subsection{Linear polarization -- Malus' law}
    \label{subsec:MalusLaw}
    
    \figMalusFit{
    Linear v-polarized light is shone upone a polarizing filter, which is rotated by the angle $\Theta$.
    The resulting intensity is measured by a photo diode. 
        Its output current, which is proportional to the intensity, is depicted in the graphic.
    The intensity follows Malus' law ~\eqref{eq:theoMaluslaw}.
    We fit it in the dataset, with the default intensity $I_0$ and the offset angle $\Theta_0$ as
    free parameters.
    It is evident that Malus' law holds.
    This method can be used to determine the polarization angle $\Theta_0$ of a linear
    polarized light source as for example in ~\autoref{subsec:Faraday}
    }
    
    A linear polarizing filter is denoted in Jones' formalism with the matrix:
    \begin{align*}
        P_{\Theta} = \begin{pmatrix} sin^2\Theta & - cos\Theta sin\Theta \\
                        - cos\Theta sin\Theta & cos^2\Theta \end{pmatrix}
    \end{align*}
    Where $\Theta$ is the angle between the polarizing filter and the y-axis.
    If we let v-polarized light shine on the polarizing filter, it will be partially be absorbed.
    We can get the resulting intensity by calculating the magnitude of the Jones' vector:
    \begin{align}
        \label{eq:theoMaluslaw}
        \vec{J} &= P_{\Theta} \vec{J}_V = \begin{pmatrix} -cos\Theta sin\Theta \\
                                               cos^2\Theta\end{pmatrix} \\
        I &\propto | \vec{J} |^2 = cos^2(\Theta) 
    \end{align}
    This is Malus' law.
    We want to verify this law.
    
    In our setup we use an incandescent electric lamp, which emits unpolarized light in the full
    visible range.
    We polarize it by placing a linear polarizing filter in front of the lamp.
    A second polarizing filter, the analysator, is mounted behind that.
    It is rotated by the angle $\Theta$ relative to the first polarizing filter.
    A photo diode detects the resulting intensity.
    We measure its output current, which is proportional to the intensity.
    
    The data is shown in ~\autoref{fig:MalusFit}.
    We fit the function:
    \begin{align}
        \label{eq:malusFit}
        I = I_0 cos^2(\Theta + \Theta_0)
    \end{align}
     Where $I_0$ and $\Theta_0$ denote free parameters.
    From the graphic it is evident, that the data follows Malus' law.
    
    An application of this law is to determine the polarization angle $\Theta_0$ of an unknown
    linear polarized light source.
    By rotating the analysator around the whole $360\degree$ range and measuring the
    resulting intensity, the unknown angle $\Theta_0$ can be determined.
    This method could be applied in the Faraday experiment of ~\autoref{subsec:Faraday}.
    
    Furthermore, another improvement would be to make $I_0$ not a free parameter, but to calculate it
    via the photo diode's device specifications.
    This would enable more quantitative statements.
    Here we are satisfied with the simpler method,
    since the qualitative results suffice for the purposes of this paper.
    
    \subsection{Circular polarization}
    
    In a circular polarized, electromagnetic wave, the electric field vector rotates around the
    propagation direction of the wave.
    Circular polarized light can be created by shining diagonal polarized light upon a \quaterWavePlate.
    A \quaterWavePlate consists of an optical anisotropic material, where the propagation speed for
    v-polarized light, differs in comparison to h-polarized light.
    This results in a phase shift between the x- and y-component of the electromagnetic wave.
    If the \quaterWavePlate has the right thickness, so the shift between the x- and y-component
    is exactly $\frac{\pi}{2}$, we can convert d-polarized light into circular polarized light.
    
    In Jones' formalism a \quaterWavePlate is represented by the matrix:
    \begin{align*}
        Q_{\frac{\lambda}{4}} = \begin{pmatrix} 1 & 0 \\ 0 & i \end{pmatrix} 
    \end{align*}
    If d-polarized light is shone upon the \quaterWavePlate it is converted to right-handed circular light:
    \begin{align*}
        \vec{J} &= Q_{\frac{\lambda}{4}} \cdot \frac{1}{\sqrt {2}} \begin{pmatrix} 1 \\ 1 \end{pmatrix} =
        \frac{1}{\sqrt {2}} \begin{pmatrix} 1 \\ i \end{pmatrix}
    \end{align*}
    This only works if the light has the right wavelength $\lambda$, matching the 
    \quaterWavePlate.
    
    To demonstrate this relationship, we mount an interference filter before our lamp
    to get monochromatic light: $\lambda = \wavelengthYellow$.
    After it, we mount a linear polarizing filter to produce d-polarized light.
    Then the \quaterWavePlate is installed with a relative angle of $45 \degree$ to the polarizing filter.
    The resulting circular light can then be observed with a second linear polarizing filter, the analysator.
    For any orientation of the polarizing filter, the same amount of light passes through.
    The \quaterWavePlates we used in our experiment, are optimized for Natrium-D-Lamps,
    so for light of the wavelength $\lambda = 589 \nm$.
    Consequently, we do not actually get circular polarized light, but elliptical light.
    The elliptical light has the effect, that when rotating the polarizing filter, we see the 
    intensity varying. 
    In a $360\degree$-rotation we observe two minima and two maxima.
    
    By placing a second \quaterWavePlate behind the first one, we get a \halfWavePlate.
    A \halfWavePlate can described with:
    \begin{align*}
        Q_{\frac{\lambda}{2}} = \begin{pmatrix} 1 & 0 \\ 0 & -1 \end{pmatrix}
    \end{align*}
    This means, it mirrors linear polarized light around the y-axis.
    
    We demonstrate this by shing light with a polarization angle $\Theta$ upon the \halfWavePlate.
    We observe, the resulting light has the polarization $-\Theta$.
    Again the problem arises that the monochromatic light from the interference filter does not
    match the wavelength of the \halfWavePlate.
    It results in us observing elliptical polarized light.
    
    This experiment could be improved by using a laser instead of an interference filter,
    since the interference filter has a bandwidth of wavelengths.
    In contrast, a laser is monochromatic, with only a very small bandwidth.
    
    \subsection{Birefringence with a calcite crystal}

    \begin{figure}[htbp]
        \includegraphics[width=0.9\linewidth]{Positively_birefringent_material}
        \centering
        \caption{
        The graphic shows the principle of birefringence.
        We have a optical anisotropic material, where p-polarized light is refracted
        differently than s-polarized light.
        The s-polarized light is parallel to the optical axis (and is in the graphic called
            `parallel polarized').
        It is refrected under a greater angle and becomes the extraordinary ray,
            while p-polarized light sees a smaller refractive index and results in
        the ordinary ray. Image source~\cite{birefringenceGraphic}}
        \label{fig:birefringent}
    \end{figure}
    
    In optical anisotrop materials, the refractive index differs for light polarized
    parallel or perpendicular to the optical axis.
    This leads to the effect of birefringence, where the incoming light ray is split into 
    two -- an ordinary and extraordinary ray.
    The effect is shown in ~\autoref{fig:birefringent}.
    In the graphic one may observe that s-polarized light is greater refracted,
    resulting in the extraordinary ray,
    while p-polarized light has a smaller refractive angle.
    
    In our experiment we mount a calcite crystal, which is fixed inside a fringe,
    before our incandescent lamp. 
    We observe two light rays on the detection screen.
    The ordinary and extraordinary ray can be distinguished by rotating the cristal.
    We observe that the extraordinary ray rotates around the ordinary, since it has a 
    greater refractive angle.
    If we mount a linear polarizing filter after the calcite crystal, we can see that both rays are
    polarized, with a perpendicular polarization direction.
    
    For the experiment one has to pay attention that the incoming light ray is neither parallel nor
    perpendicular to the optical axis of the calcite crystal. 
    In both cases you would not see the split-up between the ordinary and extraordinary ray,
    since then you would only have one refractive index for the ray.
    
    \section{Applications}
    In the first part we showcased simple setups to create and detected polarized light.
    In the second part we now want to present two possible applications of those setups,
    for investigating advanced effects.
    \subsection{Optical activity}
    A chiral molecule does not posses an intrinsic mirror symmetry.
    If you have a solution with a greater amount of one type than its mirrored counterpart,
    the solution is optical active.
    It means right-handed circular light has a different refractive index than left-handed circular light.
    If you shine linear polarized light through an optical active solution, the
    polarization vector gets tilted around a specific angle $\Delta \Theta$.
    The angle $\Delta \Theta$ depends on the distance $l$ travelled through the solution,
    the concentration of the molecule $c$, and the specific optical rotation $\alpha$:
    \begin{align}
        \label{eq:specificOpticalRotation}
        \Delta \Theta = \alpha \, c \, l
    \end{align}
    The specific optical rotation $\alpha$ depends on the molecule and the wavelength of the light.
    
    In our experiment we want to determine the specific rotation of sucrose.
    For this, we use our incandescent lamp and mount the $\wavelengthYellow$-monochromator
    before it.
    Then the light is polarized with a linear polarizing filter and shines through
    a glas tube, containing the sugar solution.
    The glas tube has the length $l = \sugarGlasTubeLength$.
    The tilted angle $\Delta \Theta$ is then measured with a half-shade polarimeter.
    
    A half-shade polarimeter is a device consisting of two linear polarizing filters, which are mounted
    with a relative angle $\Delta \varphi$ next to each other.
    We first want to determine the relative angle $\Delta \varphi$ and the default angle $\Theta_0$ of
    the light's polarization vector.
    To do this, we remove the glas tube, containing the sugar solution, and measure the angle 
    $\varphi_1$, where the first polarizing filter lets all light pass, and $\varphi_2$, where
    the second filter lets all light pass.
    From a measurement series containing five data-points, we can determine the relative angle
    and the default angle:
    \begin{align*}
        \Delta \varphi & = \varphi_2 - \varphi_1 & = & \quad \halfShadowAngle \\
        \Theta_0 & = \frac{\varphi_1 + \varphi_2}{2} & = & \quad \defaultAngle
    \end{align*}
    The errors are derived from the variance of the measurement series and then propagated with
    the variance formula~\cite{errorPropagation} (also refered as gaussian error propagation).
    This procedure is always applied unless stated otherwise.
    
    We now want to measure the specific optical rotation.
    We insert the glas tube back into the apparatus.
    It contains a sugar solution with a concentration $c = \referenceConcentration$.
    We now measure the angle $\Theta$ of the rotated polarization vector.
    We do this by rotating the two polarizing filters of the polarimeter, so both let the same amount of
    light pass (they appear with the same brightness).
    We can calculate the relative rotation $\Delta \Theta$ and with equation ~\eqref{eq:specificOpticalRotation}
    we can determine the specific optical rotation:
    \begin{align*}
        \Delta \Theta &= \Theta - \Theta_0 = \referenceRotationAngle \\
        \specRot &= \sugarSpecificOpticalRotation
    \end{align*}
    
    The resulting value is consistent with the literature value:
    \begin{align*}
        [\alpha]_{589\nm} = 6.65 \unitSpecrot ~\cite{instr}
    \end{align*}
    The slight deviation can be contributed to the fact that for the literature value natrium-D-light
    is used, while we use a monochromator with $\lambda = \wavelengthYellow$.
    
    With the measured specific optical rotation we can measure an unknown sucrose concentration.
    We use a thinned-down solution from a different team.
    We measure the rotation angle and with ~\eqref{eq:specificOpticalRotation} we can determine
    the concentration:
    \begin{align*}
        \Delta \Theta &= \testRotationAngle \\
        c &= \sugarTestConcentration
     \end{align*}
    Unexpectedly we see that the concentration is in fact greater than before.
    The group, who gave use their solution, contributed it to the fact that there was some sugar left
    inside the glas with which they have thinned downed their sugar solution.
    So actually increasing the concentration.

    \subsection{Faraday effect}
    \label{subsec:Faraday}

    \newcommand{\BMean}{\bar{B}}
    \newcommand{\BMax}{B_{\mathrm{max}}}

    The Faraday effect occurs when a magnetic field is applied at a dielectric, such as glas.
    If light travels parallel to the B-field through the dielectric, the refractive index
    for left and right-handed circular light differs, depending on the field strength.
    It is caused by the Zeeman effect, where the degenerate atomic energy levels are split up by the magnetic field.
    Because of the dipole-transition rule, left and right-handed circular light couple differently to the atoms.
    The difference in the refractive index between left and right-handed circular light leads to a rotation
    $\Theta$ of linear polarized light.
    This is analog to chiral molecules.
    The angle $\Theta$ depends on the length of the dielectric $l$, the field strength $\BMean$ and
    the Verdet constant $V$:


    \begin{align}
        \label{eq:VerdetTheta}
        \Theta &= V \, \BMean \, l
    \end{align}
    The Verdet constant is wavelength dependent and differs for different materials.
    It can be approximated with the formula~\cite{gerth}:
    \begin{align}
        \label{eq:TheoVerdetCurve}
        V(\lambda) = - \frac{e}{2 m_e c} \frac{\mathrm{d}n}{\mathrm{d}\lambda}
    \end{align}
    Here $e$ is the elementary charge, $m_e$ is the electron mass, $c$ the speed of light and
    $\lambda$ is the wavelength of the transmitted light.
    Further, $n(\lambda)$ is the refractive index of the dielectric.
    It is wavelength dependent and differs for each material.

    \begin{figure}[htbp]
        \includegraphics[width=0.9\linewidth]{FaradaySetup}
        \centering
        \caption{
            The right side shows the setup we used for the Faraday effect.
            The magnetic field of the coils is amplified by the iron yoke, and bend so the magnetic field
            is parallel to the light ray.
            The \NSF-glas rod is placed inserted in the borehole of the yoke.
            The light travels through the boring and then through the glas rod.
            The left side of the picture shows the approximate setup to calculate the space-resolved magnetic field.
            For the fit ~\eqref{eq:coilApproximation} we asume that the magnetic field is created by two symetric coils
            with a length $d$ and a distance $a$ from the center.
            The measured field strength as function of $x$ and the approximate fit are depicted in \autoref{fig:fyLocalBField}.
        }
        \label{fig:faradaySetup}
    \end{figure}

    Our experimental setup consists of an incandescent lamp, after it is placed an interference and a
    polarizing filter to create monochromatic linear polarized light.
    The light then traverses the glas rod.
    We use the dense flint glas \NSF ~\cite{glassTubeDatasheet} from `Schott Advanced Optics'.
    The rod has the length $l = \glasCylinderLength$.
    It is placed inside a borehole in the iron yoke.
    The yoke connects two copper coils, which create the magnetic field.
    Compare ~\autoref{fig:faradaySetup}.
    A second polarizing filter, the analysator, is placed after the glas rod,
    and a photo diode detects the resulting intensity.

    \figfyLocalBField{For the Faraday effect we use the setup depicted at the left side of ~\autoref{fig:faradaySetup}.
    To measure the B-field at the position of the glas rod, we first remove it from the iron yoke and
    then insert the axial hall sensor~\cite{hallSensor}.
    We apply the current $I = \CurrentForLocalBField$ and
    for $2\unit{mm}$-intervals record the B-field.
    For the regression we approximate the field, with a magnetic field created by two symmetrical placed coils as
    shown an the right side of ~\autoref{fig:faradaySetup}.
    We fit equation ~\eqref{eq:coilApproximation} in the data and by integrating over the glas rod's length
        $l = \glasCylinderLength$, get the mean field $\BMean = \MeanBField$.}

    In equation ~\eqref{eq:VerdetTheta} it is assumed that the B-field is constant over the whole
    length of the dielectric.
    This is not the case for our setup.
    To account for this fact, we first have to determine the space-resolved magnetic field.
    We remove the glas rod from the iron yoke and
    use the axial hall sensor~\cite{hallSensor} from `PhyWe' to measure the magnetic field.
    It has an accuracy of $\pm 2 \% $.
    We apply a current $I = \CurrentForLocalBField$ at the coils and then measure
    in $2 \unit{mm}$ intervalls the B-Field as a function of $x$.
    The recorded data is depicted in ~\autoref{fig:fyLocalBField}.
    To find a suitable fit, we assume that the magnetic field is equal to a magnetic field created by two
    coils, which are symmetrical placed along the x-axis (See the left side of ~\autoref{fig:faradaySetup}).
    The coils have the length $d$, the winding number $N$, the radius $R$ and are placed at a distance $a$, away from the center.
    The resulting function, which we use for the fit is:

    \newcommand{\CoilEnd}[1]{\frac{#1}{\sqrt{R^2 + (#1)^2}}}
    \newcommand{\dHalf}{\frac{d}{2}}

    \begin{align}
        \label{eq:coilApproximation}
        \begin{split}
        B =  \frac{\mu_0 \, I \, N}{2} \scriptstyle ( \CoilEnd{x - a - \dHalf} - \CoilEnd{x - a + \dHalf} \\
         \scriptstyle - \CoilEnd{x + a - \dHalf} + \CoilEnd{x + a +\dHalf} \,)
            \end{split}
    \end{align}
    Here $d$,$N$,$R$,$a$ and the constant offset $x_0$ are free parameters in the fit.
    The offset $x_0$ is suppressed in ~\eqref{eq:coilApproximation} for clarity

    \newcommand{\lHalf}{\frac{l}{2}}
    \newcommand{\xInteg}{\int_{-\lHalf}^{\lHalf} dx \,}

    With the fit we now can calculate the mean B-Field $\BMean$, which we need
    to calculate the Verdet constant in equation ~\eqref{eq:VerdetTheta}.
    We determine it by integrating $B(x)$ over the length $l$ of the glas rod.
    \begin{align}
        \label{eq:meanBFieldInt}
        \begin{split}
        \BMean &= \frac{1}{l} \xInteg B(x) \\
            &= \MeanBField \quad \text{for} \quad I = \CurrentForLocalBField
        \end{split}
    \end{align}

    We determine the error by using the covariance matrix emerging from the scattering of the
    data-points around the fit, and then using the variance formula:

    \begin{align*}
        \sigma_{\mathrm{\BMean}} = \frac{1}{l} \left( \xInteg \sigma_{\mathrm{B}}^2 \right)^{\frac{1}{2}}
    \end{align*}

    \figfyMeanBField{
        To get the mean B-field as a function of the current $\BMean(I)$, it suffices to measure the
        magnetic field at the maximum $\BMax(I)$ and use \eqref{eq:meanBFromMaxB} to infer the
        mean field, since the magnetic field is proportional to the current $I$.
        The plot shows the from the measured data $\BMax$ calculated mean field $\BMean$.
        We fit a linear function to get the relation $\BMean(I)$.
    }

    We want to determine the mean field $\BMean$, as a function of the current $I$, for the entire
    current range $0\dots3\unit{A}$.
    In equation ~\eqref{eq:coilApproximation} we observe that the B-Field is proportional to the
    current $I$.
    So instead of repeating the previous procedure for every current, it suffices to
    measure the B-Field at just one fixed point and then infer the mean field by using the results from above.
    We chose to measure the B-field at the maximum, since here we have the highest accuracy.
    We use the following formula:
    \begin{align}
        \label{eq:meanBFromMaxB}
        \BMean (I) = \frac{\BMean(\CurrentForLocalBField)}{\BMax(\CurrentForLocalBField)} \cdot \BMax(I)
    \end{align}

    The measured data is depicted in \autoref{fig:fyMeanBField}.
    For high currents we have to pay attention that the coils do not get to hot, since this changes
    the magnetization and subsequent the B-field.
    We fit a linear function in the data and get the B-field as a function of the current.
    It has the slope $s = \MeanBFieldModelmu$ and the offset $m_0 = \MeanBFieldModelm$.
    The offset corresponds to remanence magnetization of the iron yoke.
%    In the following calculations we use the fit as $\BMean(I)$.
    The error is inferred from the regression's covariance matrix.
    
    \figfyAngleDependencePlot{
        The plot depicts the measured polarization vector's rotation angle $\Theta$ as a function of the B-field.
        With two interference filters we create monochromatic polarized light.
        After the filter we mount a \NSF-glas rod, where a magnetic field is applied.
        The Faraday effect causes the polarization vector to rotate.
        We measure the tilt-angle with a second polarizing filter, by manually search the intensity minimum.
        It has a high uncertanty $\sigma_\Theta = \fyErrorAngle$.
        We fit equation ~\eqref{eq:VerdetTheta} into the data to get the Verdet constants, which are
        listed in ~\autoref{tab:verdetConstants}.
    }
    
    Now we start measuring the polarization vector's rotation $\Theta$, caused by the Faraday effect.
    We insert the glas rod back into the iron yoke.
    With an interference and polarizing filter we create monochromatic polarized light, which
    shines through the glas rod. 
    We determine the light's tilted polarization vector with a second polarization filter,
    placed after the glas rod.
    We manually rotate the filter and search for the minimum of the transmitted intensity.
    We estimate the method's error to $\sigma_{\Theta} = \fyErrorAngle$.
    For the coil's current range $0\dots 3 \unit{A}$ we measure the rotation angle $\Theta$.
    This done for two monochromators.
    The data is depicted in ~\autoref{fig:fyAngleDependencePlot}.
    The B-field is calculated with the previous fit $\BMean(I)$.
    By fitting a linear function in the data we can calculate the Verdet-constants which are
    listed in ~\autoref{tab:verdetConstants}.

    \newcommand{\VTheo}{V_{\mathrm{theo}}}
    \renewcommand{\arraystretch}{1.5}
    \begin{table}[h!]
        \centering
        \begin{tabular}{ c c c }
            \hline \hline 
            $\lambda$ & $V = \frac{\Theta}{\BMean \, l} \ \text{in} \ \unitVerdet$ & $\VTheo \ \text{in} \ \unitVerdet$\\ 
            \hline
            $\fyWavelengthTwo$ & $\verdetConstantTwoNoUnit$ & $\idealVerdetConstantTwoNoUnit$ \\
            $\fyWavelengthThree$ & $\verdetConstantThreeNoUnit$ & $\idealVerdetConstantThreeNoUnit$ \\ 
            \hline \hline
            
        \end{tabular}
        \caption{
            For three wavelengths we measure the polarization rotation angle $\Theta$.
            Via a linear fit and with equation ~\eqref{eq:VerdetTheta} 
            we can calculate the Verdet constants, which are listed here.
            The uncertanty derives mainly from the great error, intruduced by manually measuring the
            polarization vector.
            With the approximation ~\eqref{eq:TheoVerdetCurve} and the Sellmeier equation for the
            refractive index ~\eqref{eq:Sellmeier} we can calculate the theoretical expected verdet
            constants.
            The errors derive from the spectral width of the wavelength $\lambda$.
            The great discrepancy between the expected and measured values
            may be caused by the manual method of determining the polarization angle, since
            it may lead to systematical errors, which we did not account for in our calculations.
            Furthermore, the yoke could have grown warm, since it reduces the magnetic field
            and consequently the Verdet constants appear smaller.
        }
        \label{tab:verdetConstants}
    \end{table}
    
    We did measure the Verdet constant for a third wavelength $\lambda = 440 \nm$.
    The value we did get $V = (25.3 \pm 3.6) \unitVerdet$ deviates significantly from the theoretical value 
    $V = 59.88 \unitVerdet$.
    It can be attributed to the interference filter being defective.
    It does not filter the light correctly, making the measurement worthless.
    Therefore, we omitted it from the rest of the evaluation.
    
    
    
%    As can be seen in ~\autoref{fig:fyAngleDependencePlot} the manual method is quite inaccurate,
%    which leads to the large errors of the Verdet constants.
%    The reason is that the intensity's minimum is not easy to find manually, since it follows Malus' law:
%    $cos^2\Theta$ which is flat at the extrema.
    
    \figfyVerdetPlot{
    The plot depicts the measured Verdet constants from ~\autoref{tab:verdetConstants} and
    the expected theoretical curve ~\eqref{eq:TheoVerdetCurve}.
    The theoretical curve depends on the refractive index $n(\lambda)$ of the used \NSF-glass.
        It can be calculated with the Sellmeier equation ~\eqref{eq:Sellmeier} and the parameters 
    from the datasheet~\cite{glassTubeDatasheet}.
    In the plot a discrepency between the data and the theoretical value is evident.
    It may be caused by the manual method of determining the polarization angle, since
    it may lead to systematical error, which we did not account for in our calculations.
    A further explination may be the yoke growing warm, since it reduces the magnetic field
    and the Verdet constants appear smaller.
    }
    
    The measured data points can be compared to the theoretical Verdet constants $\VTheo$.
    We use the Sellmeier-equation to determine the refractive index of the \NSF-glas rod:
    \newcommand{\SellmeierEqPart}[1]{\frac{B_{#1} \, \lambda^2}{\lambda^2 - C_{#1}}}
    
    \begin{align}
        \label{eq:Sellmeier}
        n^2(\lambda) = 1 + \SellmeierEqPart{1} + \SellmeierEqPart{2} + \SellmeierEqPart{3}
    \end{align}
    The six constants can be inferred from the datasheet~\cite{glassTubeDatasheet}.
    With the dispersion relation $n(\lambda)$ and the approximation ~\eqref{eq:TheoVerdetCurve}
    we calculate the theoretical wavelengths.
    The theoretical values are displayed in ~\autoref{tab:verdetConstants} and in ~\autoref{fig:fyVerdetPlot}.
    The in the table listed errors derive from the large spectral width of the light.
    
    A big discrepancy is evident between the theoretical and the measured values.
    A cause may be the manual method of finding the intensity minimum, since
    the intensity follows Malus' law, which flattens at the extrema.
    It introduces systematical errors not only because you have to manually judge the minimum, but
    also because diffuse light from different sources may seemingly shift the minimum.
    Another cause may be the warming of the yoke.
    If it heats up, the iron's magnetic susceptibility decreases, reducing the magnetic field.
    Consequently, the polarization vector is tilted less, which results in a seemingly smaller
    Verdet constant.
    
    The experiment could be improved by, instead of manually searching for the intensity minimum,
    measuring the intensity for the entire polarizing filters' angle range.
    As demonstrated in ~\autoref{subsec:MalusLaw}, the angle offset $\Theta$ is inferred
    by fitting Malus' law.
    It would require recording the data electronically, due to the large amount of data points.
    Further the accuracy could be increased by utilizing a laser.
    It has a far smaller spectral width than the monochromators.
    Lastly the temperature of the coils and the yoke could be controlled by an extra cooling,
    so the heat does not change the iron's susceptibility, distorting the results.
    
    
    \section{Summary}
    
    In the paper we showcase experimental setups to study linear and circular polarized light.
    In the first part we introduce the basic procedure, to create and detect polarized light.
    By mounting two linear polarizing filters after one another, we demonstrate Malus' law.
    It describes the intensity $I$ passing through the filters, depending on their relative orientation $\Theta$.
    We use an incandescent lamp to create, and a photo diode to detect the linear polarized light.
    This method can be used to determine the orientation of linear polarized light.
    
    Then we qualitatively study the effects of \quaterWavePlates on linear polarized light.
    With Jones' formalism we predict that a \quaterWavePlate converts diagonal polarized light 
    to circular polarized light.
    But only if the wavelength of the light matches to the \quaterWavePlate.
    Since the interference filters we use produce light of $\lambda = \wavelengthYellow$,
    and the \quaterWavePlates are for natrium-D-light $\lambda = 589 \nm$,
    we only observe the conversion to elliptical light.
    By combining to \quaterWavePlates we get a \halfWavePlate.
    It mirrors the polarization vector at the y-axis, which we are able to observe.
    
    As a third point we study birefringence.
    We use a calcite cristal, which splits incoming light into an ordinary and an extraordinary ray.
    We identify the extraordinary ray by rotating the cristal, since the extraordinary rotates around
    the ordinary ray.
    Furthermore, by placing a linear polarizing filter after the cristal, we observe that the ordinary and
    extraordinary ray are polarized perpendicular to each other.
    
    In the second part we showcase two possible applications of our setup.
    First we study the optical activity of sugar.
    Since sucrose is a chiral molecule, a solution is optical active, meaning linear polarized light
    traveling through the solution, will be tilted around an angle $\Theta$.
    We use the interference filter to create light $\lambda = \wavelengthYellow$.
    It is then polarized and shines through a tube filled with a sugar solution $c = \referenceConcentration$.
    We measure the tilt-angle with a half shade polarimeter, and calculate the specific optical rotation
    $\specRot = \sugarSpecificOpticalRotation$.
    The method can be used to also determine the concentration of an unknown sugar solution.
    
    As a second application we demonstrate the Faraday effect.
    When linear polarized light traverses through a dielectric, where a magnetic field is applied, the
    polarization vector gets also tilted.
    We use an iron yoke with two coils to create the magnetic field.
    The dielectric, we use a \NSF-glas rod, is placed inside a borehole in the yoke.
    We first gauge the space-resolved magnetic field inside the boring with a hall sensor.
    Afterward we measure the polarization vectors' tilt with a pol filter, by searching the intensity minimum.
    For two wavelengths, created by two interference filters, the Verdet constant is determined:
     $\VTwo = \verdetConstantTwo$ and $\VThree = \verdetConstantThree$.
    The values deviate from the literature values, which may be due to the inaccuracy of the manual
    measuring process.
    An improved approach would involve using the strategy from above, by measuring the intensity for the entire
    angle range and using Malus' law to determine the angle offset.
    
    
    %FF: Angabe der verwendeten Literatur mit Quellennachweis.
    \begin{thebibliography}{}    %so wird das Literaturverzeichnis erstellt
        \bibitem{instr} Physikalisches Grundpraktikum, Universitüt Würzburg, Modul C1, Versuch 40, Polarisation des Lichts, 2024
        \bibitem{gerth} Meschede, Dieter, Gerthsen Physik, 25. Auflage, Springer-Verlag, Berlin, 2015
        \bibitem{birefringenceGraphic} \url{https://en.wikipedia.org/wiki/Birefringence#/media/File:Positively_birefringent_material.svg}, last visited 19.03.2024
        \bibitem{errorPropagation} \url{https://nistdigitalarchives.contentdm.oclc.org/digital/collection/p16009coll6/id/99848}, last visited 19.03.2024
        \bibitem{glassTubeDatasheet} \url{https://www.schott.com/shop/advanced-optics/en/Optical-Glass/N-SF6/c/glass-N-SF6}, last visited 20.03.2024 
        \bibitem{hallSensor} \url{https://www.phywe.de/physik/elektrizitaet-und-magnetismus/elektrostatik-und-elektrisches-feld/hall-sonde-axial_2172_3103/}, last visited 20.03.2024
    \end{thebibliography}
    
\end{document}