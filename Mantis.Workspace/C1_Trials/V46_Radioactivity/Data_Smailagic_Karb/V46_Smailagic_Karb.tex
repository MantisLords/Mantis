%! Author = User
%! Date = 13.09.2023

% Preamble
\documentclass[a4paper,10pt,twocolumn]{article}

% Packages 
\usepackage[utf8]{inputenc}  %man kann Sonderzeiche wie ü,ö usw direkt eingeben
\usepackage{amsmath}           %macht
\usepackage{amsfonts}          %       Mathe
\usepackage{amssymb}           %              mächtiger
\usepackage{graphicx}          %erlaubt Graphiken einzubinden (.eps für dvi und ps sowie .jpg für pdf)
\usepackage[T1]{fontenc}       %Zeichenbelegung der verwendeten Schrift
\usepackage{ae}                %macht schöneres ß
\usepackage{typearea}
\usepackage{amstex}
\usepackage{siunitx}
\usepackage{hyperref}	         %ermöglicht änderung des Seitenspiegels
\usepackage{subcaption}


\usepackage{amsmath}
\usepackage{tikz}
\usepackage{pgfplots}

\newcommand{\alphaNoError}{(4.047 \pm 0.036)}
\newcommand{\betaNoError}{(-4.73 \pm 0.29) \cdot 10^{-3}}
\newcommand{\halfTimeNoError}{(146.5 \pm 9.1)\ s}
\newcommand{\alphaGauss}{(4.04 \pm 0.10)}
\newcommand{\betaGauss}{(-4.62 \pm 0.95) \cdot 10^{-3}}
\newcommand{\halfTimeGauss}{(150 \pm 31)\ s}
\newcommand{\alphaPoisson}{(4.05 \pm 0.10)}
\newcommand{\betaPoisson}{(-4.75 \pm 0.95) \cdot 10^{-3}}
\newcommand{\halfTimePoisson}{(146 \pm 29)\ s}
\newcommand{\symN}{\delta N}



\pagestyle{scrheadings}        %sagt Koma-Skript, dass selbstdefiniers Kopfzeilen verwendet werden
\typearea{16}                  %stellt Seitenspiegel ein
\columnsep25pt								 %definiert Breite zwischen den zwei Spalten von \twocolumns

\renewcommand{\pnumfont}{%     %ändert die Schriftart der Seitennummerierung
    \normalfont\rmfamily\slshape}  %ändert die Schriftart der Seitennummerierung 

\newcommand{\Cs}{Cs-137 }
\newcommand{\Bam}{Ba-137m }
\newcommand{\Ba}{Ba-137 }

\newcommand{\unit}[#1]{\, \mathrm{#1}}
\newcommand{\mSv}{\unit{mSv}]}
\newcommand{\yeardosis}{5 \mSv}
\newcommand{\Thalf}{T_{\mathrm{\frac{1}{2}}}}
\newcommand{\tgate}{t_{\mathrm{gate}}}
\newcommand{\tgateOne}{\tgate = 1 \unit{s}}
\newcommand{\tgateTen}{\tgate = 10 \unit{s}}
\newcommand{\HDot}{\dot{H}}
\newcommand{\muSv}{\unit{\mu Sv}}
\newcommand{\muSvh}{\unit{\frac{\mu Sv}{h}}}

\newcommand{\halfLifeCs}{30.2}


\begin{document}
    \twocolumn[{\csname @twocolumnfalse\endcsname                %erlaubt "Abstrakt" über beide Spalten
    \titlehead{                                                  %Kopfzeile
        \begin{tabular*}{\textwidth}[]{@{\extracolsep{\fill}}lr}   %Kopfzeile
            Supervisor: Leonid Bovkun & \today\\                          %Kopfzeile - Betreuer
        \end{tabular*}                                             %Kopfzeile
    }
    \title{Mega Sensation: Students from Würzburg discover Radioactivity!!!}  %Titel der Versuchs
    \author{Salahudin Smailagić and Thomas Karb}                     %Namen der Studenten
    \date{}                                                         %benötigt um automatisches Datum auszuschalten
    \maketitle                                                      %erzeugt Titelseite
    \vspace{-5ex}                                                   %verringert Abstand zur Überschrift
    \begin{abstract}                                                %Beginn des Abstracts
        
        We study how radioactive materials can be detected and statistical properties extracted.
        We use the $\gamma$-emitter \Cs and measure the equivalent dosis next to the sample $\HDot = \RadiationNear$ and
        $d$ away $\HDot = \RadiationFar$.
        So we do not exceed the annual threshold of $\yeardosis$~\cite{safety} for $\SafeTime$, which makes the sample save to use.
        We then test the Geiger conter's responsivity by placing the sample at a fixed distance.
        The percentage of detected $\gamma$-photons is $\eta$.
        Then we use the Geiger Müller counter to optain the half-life of the unstable \Bam isotope $\Thalf = \BariumHalfLifeFromFit$.
        Lastly we want to test the statistacly properties of radioacitivity.
        We measure the decay events for an uranium trioxide coated tile for gate times $\tgateOne$ and $\tgateTen$,
        to get ca.\ $800$ measurements each.
        We accept the hypothis that the decay events are Poission distributed at a $5\%$-significance level.
        The hyphothis that for $\tgateTen$ the events are approximately Gaussian distributed is rejected
        with a $5\%$-significance level.
        \\
        Measuerement made: March, 22nd 2024\\       %Datum ändern!
        Submitted: March, 28th 2024                %Datum ändern!
        \\
        \\
    \end{abstract}
    }] 
    \section{Introduction}
    
    Radioactive decay is a statistical process.
    Whether a particular atom will decay at a specific moment is random and can not be predicted.
    However, for a large assembly of atoms we can predict the emerging decay rate.
    A radioactive atom has a $50\,\%$ chance to decay in the half live time $\Thalf$.
    It follows, if we have an initial number $N_0$ of undecayed atoms, the number $N(t)$ of remaining radioactive atoms
    at time $t$ will be:
    \begin{align}
        \label{eq:decayRate}
        N(t) = N_0 \cdot 2^{-\frac{t}{\Thalf}}
    \end{align}
    The activity $A(t) = \dot{N}(t)$ follows the same exponential decay law.
    It denotes the number of decays per second.
    
    \section{Dosimetry and proper handling of radioactive substances}
    
    Before we start with our measurements, we test whether our work station is contaminated.
    We use a `NEVA' dosimeter.
    It is adjusted for $\gamma$-radiation, since all radioactive probes, we utilize, emit solely $\gamma$-radiation.
    The detected activity is $\HDot = 0.083 \muSvh$ in $8 \unit{min}$.
    Which makes the work place not contaminated, since it the natural radiation exposure for people in Central Europe is
    $\HDot = 0.1 \muSvh$~\cite{safety}.
    
    To comply with safety regulations, we use a second dosimeter and placed it around $1.5\unit{m}$ away from
    the radioactive probe.
    It corresponds roughly to distance we as experimenter have from the probe, so its measurement is a good estimate
    for the radiation we experience.
    We measured for the entire duration of the experiment the local dose rate in $15\unit{min}$ intervals.
    The average rate was $\HDot = \AmbientRadiation$ for the $2.5 \unit{hours}$, when we had the \Cs-sample exposed.
    It corresponds to a yearly dose of $H = \AmbientDosisPerAnnum$, which is well under the legal annual limit
    of $\yeardosis$.
    So we did not suffer radiation intoxication.
    The activity we measured is most likely background radiation.
    It comprises of cosmic rays and radiation from terrestrial sources like potassium, uranium
    and airborne radon.
    
    For the experiment we utilize a probe made up of \Cs.
    The unstable isotope \Cs has a half live of $\halfLifeCs \unit{years}$~\cite{instr} and decays with a $\beta$-decay to
    the metastable isotop \Bam.
    This is an excited state of \Ba.
    It will emit $\gamma$-radiation with $0.661 \unit{MeV}$ and return to its ground state.
    Furthermore, there is a $5\%$ chance, that \Cs will directly decay to the ground state of \Ba,
    emitting high energy beta radiation.
    
    Our probe is cased in a $1\unit{cm}$ thick aluminum cylinder, completely blocking the $\beta$-radiation.
    So only the $\gamma$-radiation can be measured.
    The aluminum cylinder in turn, is placed inside a $10\unit{cm}$ wide lead cube.
    The cube has a boring, where $\gamma$-radiation can exit.
    While no measurement is conducted, the boring is closed with a lead plug.
    The $\gamma$-radiation, which emits through the boring, can be approximated as a collimated beam.
    
    To further test the safety of our probe, we place the dosimeter right on top of the cube's borehole.
    We take a measurement series with $5\unit{s}$ intervalls and determine the dose rate $\HDot = \RadiationNear$.
    We repeat this measurement but place the dosimeter $1\unit{m}$ away from the probe.
    The dose rate is $\HDot = \RadiationFar$.
    We directly see that the radiation drastically declines with distance.
    This is why it is advised, to keep as much away from the probe as possible.
    Furthermore, we can extrapolate the dose rates for the $4$ hour duration of the entire experiment.
    The 4-hour equivalent dosis right on top the probe would be $H = \FourHourNear$, while in $1 \unit{m}$
    distance it is just $H = \FourHourFar$.
    This corresponds to $\YearlyPercentageCsRadiation \, \%$ of the annual limit.
    
    \section{Geiger Müller counter}
    \subsection{Characteristics}
    
    \GeigerMuellerPlot{
    Registered pulses in $5 \unit{min}$ as a function of the anode voltage $U$ applied at the Geiger Müller Counter.
    The counter was placed on top of the \Cs probe.
    At roughly $\ThresholdVoltage$ a plateau region starts, where the count of detected pulses does not increase
    with increasing voltage.
    }
    
    A Geiger counter consists of a metal cylinder, the cathode, filled with high pressure noble gas.
    Lengthwise a thin wire, the anode, is placed inside the cylinder.
    Between the anode and a cathode a high voltage $U$ is applied.
    If now $\gamma$-radiation enters the chamber, it will ionise the gas, producing free electrons.
    They are in turn attracted towards the anode.
    When enough electrons reach the anode, a current can be detected, signalling a decay.
    If the voltage $U$ is high enough, those attracted electrons, will collide again with gas atoms,
    producing secondary free electrons.
    This leads to an avalanche effect, where every radiation will produce a pulse.
    
    We place our Geiger-Müller-Counter on top of the \Cs probe.
    We investigate its behaviour by varying the anode voltage $U$ from $490 \unit{V}$ upto $600 \unit{V}$
    and recording the registered pulses in a $5 \unit{min}$ period.
    We see that starting from around $\ThresholdVoltage$, the count of registered pulses does not increase
    further, when increasing the voltage $U$.
    A plateau region is reached.
    At the plateau the from $\gamma$-radiation created free electrons, have enough kinetic energy to
    create secondary free electrons.
    This leads to the mentioned avalanche effect.
    In the plateau region a Geiger Müller counter should be operated to get reliable results.
    For the following measurements we operate it at $564\unit{V}$.
    
    \subsection{Null-Effect}
    
    As mentioned, on earth one has constant exposure to natural radiation.
    It is caused by cosmic radiation and traces natural occuring radioactive elements, such as uranium
    and radon.
    In our measurement it reveals itself as a constant background.
    To account for this, we place the Geiger-Müller counter away from the probes, and measure
    the registered pulses in a $5\unit{min}$ period.
    The pulse rate of the background is:
    \newcommand{\NBDot}{\dot{N}_{\mathrm{B}}}
    \begin{align*}
        \NBDot = \NullEffect
    \end{align*}
    The error stems from the poisson distribution $\sigma = \sqrt{\bar{N}}$
    As further comparison, we place the Geiger counter inside a lead box, to shield
    it from background radiation.
    We measure the pulse count for $5\unit{min}$ and determine the rate:
    \begin{align*}
        \NBDot = \NullEffectInLead
    \end{align*}
    The pulse rate is reduced by $\NullEffectPercentage$.
    Still $\gamma$-rays are able to penetrate the lead box.
    
    \subsection{Responsiveness}
    
    We want to get a rough estimate, on how responsive the Geiger Müller counter is.
    We want to determine what percentage of incoming $\gamma$-rays are actually detected as pulses.
    For this purpose, we remove the \Cs probe from its lead shield.
    It is still cased inside the aluminum cylinder, which shields all $\beta$-rays, but the
    $\gamma$-rays are emitted unobstructed.
    We now can approximate the probe as a point source, emitting $\gamma$-radiation in all directions equally.
    The activity of a point source declines quadratic with respect to the distance $r$.
    So the activity detected by the Geiger-Müller counter, which has a detection area of $a = 11\unit{cm^2}$ is:
    
    \newcommand{\ADet}{A_{\mathrm{det}}}
    
    \begin{align}
        \label{eq:activityDistance}
        A = A_0 \cdot \frac{a}{4 \pi r^2}
    \end{align}
    The \Cs probe we utilize was acquired August 1978, and had then an initial activity of $\ActivityPast$.
    With the decay law ~\eqref{eq:decayRate}, we can calculate its activity at the time of the measurement:
    \begin{align*}
        A_0 = \ActivityNow
    \end{align*}
    We place the Geiger Müller counter at a distance of $r = $ and record the registered pulses in a
    $5 \unit{min}$ period.
    \begin{align*}
        \ADet = 
    \end{align*}
    
    
    
    
    %FF: Angabe der verwendeten Literatur mit Quellennachweis.
    \begin{thebibliography}{}    %so wird das Literaturverzeichnis erstellt
        \bibitem{instr} Physikalisches Grundpraktikum, Universität Würzburg, Modul C1, Versuch 46, Dosimetrie, Halbwertszeit, 2021
        \bibitem{safety} Anlage zur Anweisung zum Schutz vor ionisierenden Strahlen, Universität Würzburg, Physikalisches Grundpraktikum, Modul C1
        \bibitem{gerth} Meschede, Dieter, Gerthsen Physik, 25. Auflage, Springer-Verlag, Berlin, 2015
        \bibitem{codata} P. J. Mohr, D. B. Newell, and B. N. Taylor: \grqq CODATA
        recommended values of the fundamental physical constants: 2014\grqq , Rev. Mod. Phys.
        88, 035009 (2016))
    \end{thebibliography}
    
\end{document}