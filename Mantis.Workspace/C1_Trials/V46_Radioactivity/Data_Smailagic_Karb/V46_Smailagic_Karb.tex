%! Author = User
%! Date = 13.09.2023

% Preamble
\documentclass[a4paper,10pt,twocolumn]{article}

% Packages 
\usepackage[utf8]{inputenc}  %man kann Sonderzeiche wie ü,ö usw direkt eingeben
\usepackage{amsmath}           %macht
\usepackage{amsfonts}          %       Mathe
\usepackage{amssymb}           %              mächtiger
\usepackage{graphicx}          %erlaubt Graphiken einzubinden (.eps für dvi und ps sowie .jpg für pdf)
\usepackage[T1]{fontenc}       %Zeichenbelegung der verwendeten Schrift
\usepackage{ae}                %macht schöneres ß
\usepackage{typearea}
\usepackage{amstex}
\usepackage{siunitx}
\usepackage{hyperref}	         %ermöglicht änderung des Seitenspiegels
\usepackage{subcaption}


\usepackage{amsmath}
\usepackage{tikz}
\usepackage{pgfplots}

\newcommand{\alphaNoError}{(4.047 \pm 0.036)}
\newcommand{\betaNoError}{(-4.73 \pm 0.29) \cdot 10^{-3}}
\newcommand{\halfTimeNoError}{(146.5 \pm 9.1)\ s}
\newcommand{\alphaGauss}{(4.04 \pm 0.10)}
\newcommand{\betaGauss}{(-4.62 \pm 0.95) \cdot 10^{-3}}
\newcommand{\halfTimeGauss}{(150 \pm 31)\ s}
\newcommand{\alphaPoisson}{(4.05 \pm 0.10)}
\newcommand{\betaPoisson}{(-4.75 \pm 0.95) \cdot 10^{-3}}
\newcommand{\halfTimePoisson}{(146 \pm 29)\ s}
\newcommand{\symN}{\delta N}



\pagestyle{scrheadings}        %sagt Koma-Skript, dass selbstdefiniers Kopfzeilen verwendet werden
\typearea{16}                  %stellt Seitenspiegel ein
\columnsep25pt								 %definiert Breite zwischen den zwei Spalten von \twocolumns

\renewcommand{\pnumfont}{%     %ändert die Schriftart der Seitennummerierung
    \normalfont\rmfamily\slshape}  %ändert die Schriftart der Seitennummerierung 

\newcommand{\Cs}{Cs-137 }
\newcommand{\Bam}{Ba-137m }

\newcommand{\unit}[#1]{\, \mathrm{#1}}
\newcommand{\Sv}{\unit{Sv}]}
\newcommand{\yeardosis}{5 \Sv}
\newcommand{\Thalf}{T_{\mathrm{\frac{1}{2}}}}
\newcommand{\tgate}{t_{\mathrm{gate}}}
\newcommand{\tgateOne}{\tgate = 1 \unit{s}}
\newcommand{\tgateTen}{\tgate = 10 \unit{s}}
\newcommand{\HDot}{\dot{H}}


\begin{document}
    \twocolumn[{\csname @twocolumnfalse\endcsname                %erlaubt "Abstrakt" über beide Spalten
    \titlehead{                                                  %Kopfzeile
        \begin{tabular*}{\textwidth}[]{@{\extracolsep{\fill}}lr}   %Kopfzeile
            Supervisor: Leonid Bovkun & \today\\                          %Kopfzeile - Betreuer
        \end{tabular*}                                             %Kopfzeile
    }
    \title{Mega Sensation: Students from Würzburg discover Radioactivity!!!}  %Titel der Versuchs
    \author{Salahudin Smailagić and Thomas Karb}                     %Namen der Studenten
    \date{}                                                         %benötigt um automatisches Datum auszuschalten
    \maketitle                                                      %erzeugt Titelseite
    \vspace{-5ex}                                                   %verringert Abstand zur Überschrift
    \begin{abstract}                                                %Beginn des Abstracts
        
        We study how radioactive materials can be detected and statistical properties extracted.
        We use the $\gamma$-emitter \Cs and measure the equivalent dosis next to the sample $\HDot = \RadiationNear$ and
        $d$ away $\HDot = \RadiationFar$.
        So we do not exceed the yearly threshold $\yeardosis$ for $\SafeTime$, which makes the sample save to use.
        We then test the Geiger conter's responsivity by placing the sample at a fixed distance.
        The percentage of detected $\gamma$-photons is $\eta$.
        Then we use the Geiger Müller counter to optain the half-life of the instable \Bam isotope $\Thalf = \BariumHalfLifeFromFit$.
        Lastly we want to test the statistacly properties of radioacitivity.
        We measure the decay events for an uranium trioxide coated tile for gate times $\tgateOne and $\tgateTen,
        to get ca.\ $800$ measurements each.
        We accept the hypothis that the decay events are Poission distributed at a $5\%$-significance level.
        The hyphothis that for $\tgateTen$ the events are approximately Gaussian distributed is rejected
        with a $5\%$-significance level.
        \\
        Measuerement made: March, 22nd 2024\\       %Datum ändern!
        Submitted: March, 28th 2024                %Datum ändern!
        \\
        \\
    \end{abstract}
    }] 
    \section{Introduction}
    
    Radioactive decay is a statistical process.
    Whether a particular atom will decay at a specific moment is random and can not be predicted.
    However, for a large assembly of atoms we can predict the emerging decay rate.
    A radioactive atom has a $50\,\%$ chance to decay in the half live time $\Thalf$.
    It follows, if we have an initial number $N_0$ of undecayed atoms, the number $N(t)$ of remaining radioactive atoms
    at time $t$ will be:
    \begin{align}
        \label{eq:decayRate}
        N(t) = N_0 \cdot 2^{-\frac{t}{\Thalf}}
    \end{align}
    The activity $A(t) = \dot{N}(t)$ follows the same exponential decay law.
    It denotes the number of decays per second.
    
    \section{Dosimetry and safe handling of radioactive materials}
    
    
    %FF: Angabe der verwendeten Literatur mit Quellennachweis.
    \begin{thebibliography}{}    %so wird das Literaturverzeichnis erstellt
        \bibitem{instr} Physikalisches Grundpraktikum, Universitüt Würzburg, Modul C1, Versuch 46, Dosimetrie, Halbwertszeit, 2021
        \bibitem{gerth} Meschede, Dieter, Gerthsen Physik, 25. Auflage, Springer-Verlag, Berlin, 2015
        \bibitem{codata} P. J. Mohr, D. B. Newell, and B. N. Taylor: \grqq CODATA
        recommended values of the fundamental physical constants: 2014\grqq , Rev. Mod. Phys.
        88, 035009 (2016))
    \end{thebibliography}
    
\end{document}