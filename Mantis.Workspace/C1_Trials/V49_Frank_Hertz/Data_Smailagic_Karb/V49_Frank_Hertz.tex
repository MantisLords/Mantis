%! Author = User
%! Date = 13.09.2023

% Preamble
\documentclass[a4paper,10pt,twocolumn]{article}

% Packages 
\usepackage[utf8]{inputenc}  %man kann Sonderzeiche wie ü,ö usw direkt eingeben
\usepackage{amsmath}           %macht
\usepackage{amsfonts}          %       Mathe
\usepackage{amssymb}           %              mächtiger
\usepackage{graphicx}          %erlaubt Graphiken einzubinden (.eps für dvi und ps sowie .jpg für pdf)
\usepackage[T1]{fontenc}       %Zeichenbelegung der verwendeten Schrift
\usepackage{ae}                %macht schöneres ß
\usepackage{typearea}
\usepackage{amstex}
\usepackage{siunitx}
\usepackage{hyperref}	         %ermöglicht änderung des Seitenspiegels
\usepackage{subcaption}


\usepackage{amsmath}
\usepackage{tikz}
\usepackage{pgfplots}

\newcommand{\alphaNoError}{(4.047 \pm 0.036)}
\newcommand{\betaNoError}{(-4.73 \pm 0.29) \cdot 10^{-3}}
\newcommand{\halfTimeNoError}{(146.5 \pm 9.1)\ s}
\newcommand{\alphaGauss}{(4.04 \pm 0.10)}
\newcommand{\betaGauss}{(-4.62 \pm 0.95) \cdot 10^{-3}}
\newcommand{\halfTimeGauss}{(150 \pm 31)\ s}
\newcommand{\alphaPoisson}{(4.05 \pm 0.10)}
\newcommand{\betaPoisson}{(-4.75 \pm 0.95) \cdot 10^{-3}}
\newcommand{\halfTimePoisson}{(146 \pm 29)\ s}
\newcommand{\symN}{\delta N}



\pagestyle{scrheadings}        %sagt Koma-Skript, dass selbstdefiniers Kopfzeilen verwendet werden
\typearea{16}                  %stellt Seitenspiegel ein
\columnsep25pt								 %definiert Breite zwischen den zwei Spalten von \twocolumns

\renewcommand{\pnumfont}{%     %ändert die Schriftart der Seitennummerierung
    \normalfont\rmfamily\slshape}  %ändert die Schriftart der Seitennummerierung 



\begin{document}
    \twocolumn[{\csname @twocolumnfalse\endcsname                %erlaubt "Abstrakt" über beide Spalten
    \titlehead{                                                  %Kopfzeile
        \begin{tabular*}{\textwidth}[]{@{\extracolsep{\fill}}lr}   %Kopfzeile
            Tutor: Leonid Bovkun & \today\\                          %Kopfzeile - Betreuer
        \end{tabular*}                                             %Kopfzeile
    }
    \title{Frank Hertz}  %Titel der Versuchs
    \author{Salahudin Smailagić and Thomas Karb}                     %Namen der Studenten
    \date{}                                                         %benötigt um automatisches Datum auszuschalten
    \maketitle                                                      %erzeugt Titelseite
    \vspace{-5ex}                                                   %verringert Abstand zur Überschrift
    \begin{abstract}                                                %Beginn des Abstracts
        
        this is to abstract
        \\
        Measuerement made: 21. September 2023\\       %Datum ändern!
        Submitted: 28. September 2023                %Datum ändern!
        \\
        \\
    \end{abstract}
    }] 
    \section{Introduction}


    
    
    %FF: Angabe der verwendeten Literatur mit Quellennachweis.
    \begin{thebibliography}{}    %so wird das Literaturverzeichnis erstellt
        \bibitem{instr} Physikalisches Grundpraktikum, Universitüt Würzburg, Modul C1, Versuch 42, Versuche mit Mikrowellen - Kristallinterferenzen mit Mikrowellen, 2021
        \bibitem{gerth} Meschede, Dieter, Gerthsen Physik, 25. Auflage, Springer-Verlag, Berlin, 2015
        \bibitem{codata} P. J. Mohr, D. B. Newell, and B. N. Taylor: \grqq CODATA
        recommended values of the fundamental physical constants: 2014\grqq , Rev. Mod. Phys.
        88, 035009 (2016))
        \bibitem{miller} \url{https://de.wikipedia.org/wiki/Datei:Miller_Indizes_Ebenen.png}, zuletzt aufgerufen am 14.09.2021
        \bibitem{pasco} Pasco Microwave Optics System (WA-9314C) Instruction Manual
        %\bibitem{missing} \url{https://en.wikipedia.org/wiki/Transverse_mode}, zuletzt aufgerufen am 8.10.2021
        \bibitem{imageMichelsonInterferometerWiki} \url{https://en.wikipedia.org/wiki/Michelson_interferometer#/media/File:Michelson_interferometer_with_labels.svg}, last visit 23.09.23
    \end{thebibliography}
    
\end{document}