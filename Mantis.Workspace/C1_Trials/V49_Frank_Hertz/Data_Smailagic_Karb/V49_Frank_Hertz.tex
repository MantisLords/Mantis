%! Author = User
%! Date = 13.09.2023

% Preamble
\documentclass[a4paper,10pt,twocolumn]{article}

% Packages 
\usepackage[utf8]{inputenc}  %man kann Sonderzeiche wie ü,ö usw direkt eingeben
\usepackage{amsmath}           %macht
\usepackage{amsfonts}          %       Mathe
\usepackage{amssymb}           %              mächtiger
\usepackage{graphicx}          %erlaubt Graphiken einzubinden (.eps für dvi und ps sowie .jpg für pdf)
\usepackage[T1]{fontenc}       %Zeichenbelegung der verwendeten Schrift
\usepackage{ae}                %macht schöneres ß
\usepackage{typearea}
\usepackage{amstex}
\usepackage{siunitx}
\usepackage{hyperref}	         %ermöglicht änderung des Seitenspiegels
\usepackage{subcaption}


\usepackage{amsmath}
\usepackage{tikz}
\usepackage{pgfplots}

\newcommand{\alphaNoError}{(4.047 \pm 0.036)}
\newcommand{\betaNoError}{(-4.73 \pm 0.29) \cdot 10^{-3}}
\newcommand{\halfTimeNoError}{(146.5 \pm 9.1)\ s}
\newcommand{\alphaGauss}{(4.04 \pm 0.10)}
\newcommand{\betaGauss}{(-4.62 \pm 0.95) \cdot 10^{-3}}
\newcommand{\halfTimeGauss}{(150 \pm 31)\ s}
\newcommand{\alphaPoisson}{(4.05 \pm 0.10)}
\newcommand{\betaPoisson}{(-4.75 \pm 0.95) \cdot 10^{-3}}
\newcommand{\halfTimePoisson}{(146 \pm 29)\ s}
\newcommand{\symN}{\delta N}



\pagestyle{scrheadings}        %sagt Koma-Skript, dass selbstdefiniers Kopfzeilen verwendet werden
\typearea{16}                  %stellt Seitenspiegel ein
\columnsep25pt								 %definiert Breite zwischen den zwei Spalten von \twocolumns

\renewcommand{\pnumfont}{%     %ändert die Schriftart der Seitennummerierung
    \normalfont\rmfamily\slshape}  %ändert die Schriftart der Seitennummerierung 


\newcommand{\wolframWa}{4.55\, \mathrm{eV}}
\newcommand{\volt}{\, \mathrm{V}}
\newcommand{\eV}{\, \mathrm{eV}}
\newcommand{\sixfiveV}{6.5 \volt}
\newcommand{\sevenV}{7 \volt}
\newcommand{\maxvoltage}{500 \volt}
\newcommand{\minvoltage}{2 \volt}
\newcommand{\neLevelA}{2\mathrm{p}^{5}3\mathrm{p}^{3}\mathrm{P}_0}
\newcommand{\HgLevelA}{6^3\mathrm{P}_1}
\newcommand{\TheoreticalTransitionHga}{4.89}
\newcommand{\TheoreticalTransitionNea}{18.9}
\newcommand{\TheoreticalTransitionNeb}{2.33}

\begin{document}
    \twocolumn[{\csname @twocolumnfalse\endcsname                %erlaubt "Abstrakt" über beide Spalten
    \titlehead{                                                  %Kopfzeile
        \begin{tabular*}{\textwidth}[]{@{\extracolsep{\fill}}lr}   %Kopfzeile
            Supervisor: Leonid Bovkun & \today\\                          %Kopfzeile - Betreuer
        \end{tabular*}                                             %Kopfzeile
    }
    \title{Frank Hertz}  %Titel der Versuchs
    \author{Salahudin Smailagić and Thomas Karb}                     %Namen der Studenten
    \date{}                                                         %benötigt um automatisches Datum auszuschalten
    \maketitle                                                      %erzeugt Titelseite
    \vspace{-5ex}                                                   %verringert Abstand zur Überschrift
    \begin{abstract}                                                %Beginn des Abstracts
        
        We want to study vacuum diodes, electron vaporisation and the Franck-Hertz-effect with a simple experimental setup.
        With a variable heating and accelerating voltage we are able to record the characteristic curve of a vacuum diode
        and observe the Langmuir-Schottky power law. 
        For heating voltages $\sixfiveV$ and $\sevenV$ we measure the exponents $\schottkySixFivePower$ and $\schottkySevenPower$ 
        respectivly.
        The deviation from the theoretical value $\frac{3}{2}$ of our setup are discussed.
        By varing the heating voltage we are able to see the Richarson law in the saturation region.
        It describes the carrier density in function of the heating temperature.
        We can estimate the work function $W_a = \richardsonModelWa$ of the cathode material.
        The strong deviation from the acctual work function $W_a = \wolframWa$~\cite{wolfram} of the wolfram cathode
        suggest, that we do not reach the actual saturation region with our maximum acceleration voltage $U_a = \maxvoltage$.
        
        In the second part we study the Franck-Hertz-effect with a similar setup.
        We use tetrodes filled with neon and mercury gas.
        From the graph we can estimate the excitation energy from the neon ground state to the $\neLevelA$ level
        as $\TransitionNeMeana$ and for mercury we get for the transition to the $\HgLevelA$ level $\TransitionHgMeana$.
        These energies are compared to the observed emitted light and with the theoretical transition energies.
        
        \\
        \\
        
        Measuerement made: 21. September 2023\\       %Datum ändern!
        Submitted: 28. September 2023                %Datum ändern!
        \\
        \\
    \end{abstract}
    }] 
    \section{Introduction}
    \section{Characteristic of a vacuum diode}

    \begin{figure}[htbp]
        \includegraphics[width=\linewidth]{VacuumDiodeSketch}
        \center
%        \caption{By varing the optical path length of the rays you see constructive or destructive interference on the
%        detector. This depends on the wavelength of the ligth, which enables you to measure it.
%        Graphic from~\cite{imageMichelsonInterferometerWiki}}
        \caption{
        Schematic setup of a vacuum diode.
        At the cathode electrons are evaporated by the heating voltage $U_h$.
        The emitted electrons are then accelerated towards the anode.
        An electric current can be detected.
        }
        \label{fig:vacuumDiodeSketch}
    \end{figure}
    
    \figcharacteristicCurve{
    Characteristic curve of a vacuum diode for the heating voltages $U_h = \heatingVoltageSixFive$ and 
    $\heatingVoltageSeven$ respectively.
    We use the full range of our voltage source from $0 \volt$ upto $\maxvoltage$.
    By reversing the polarity we can measure the current for negative accelerating voltages.
    As to be expected no current flows.
    The diode functions as a rectifier.
    Furthermore, we observe that for high accelerating voltages we reach a saturation point, while for
    low voltages we stay in the Schottky-Langmuir region.
    It is marked by the vertical lines.
    }
    
    In a vacuum diode free electrons are created via thermionic emission.
    As depicted in ~\autoref{fig:vacuumDiodeSketch} the heating voltage $U_h$ is applied at the cathode.
    By increasing the temperature, the electrons gather more and more energy, eventually enabling them, to overcome the work
    function of the wolfram cathode and leaving the wire.
    If you then apply an accelerating voltage $U_a$, the emitted electrons move towards the anode.
    We can measure a current $I_a$.
    
    In our setup we use for the heating voltage a source with the range $0 - 7 \volt$ and stability of $0.01 \volt$.
    For the acceleration voltage our source can generate voltages between $\minvoltage$ up to $\maxvoltage$ with an accuracy of
    ca. $1 \volt$.
    The voltages and the anode current $I_a$ are measured with the `Agilent Multimeter 34405A'.
    
    To exemplify the process, we record the characteristic curve of the vacuum diode in 
    ~\autoref{fig:characteristicCurve} for the heating voltages 
    $U_h = \heatingVoltageSixFive$ and $\heatingVoltageSeven$.
    We utilize the full range of our voltage source for the acceleration voltage.
    For high voltages we reach a saturation region where the current is limited by the available charge carriers,
    dictated by the Richardson-law.
    For low voltages the current follows the Schotty-Langmuir power law.
    If we reverse the polarisation we observe that no current is flowing.
    The diode functions as a rectifier.
    This is to be expected, since in the diode the charge carriers are electrons.
    They are only available at the cathode, so when reversing the polarity, no electrons can move from the anode towards
    the cathode.
    
    \subsection{Schottky-Langmuir region}
    
    \figschottky{
        The section of the data where the Schottky-Langmuir law holds.
        We estimate the region by looking where the graph as a positive curvature.
        We fit the power law~\eqref{eq:Schottky} into the data and get a exponent of
        $\schottkySixFivePower$ and $\schottkySevenPower$ for the heating voltages $\sixfiveV$ and $\sevenV$.
        It deviates from the theoretical value of $\frac{3}{2}$.
    }
    
    For low accelerating voltages we observe a power law for the anode current.
    It can be described by the Schottky-Langmuir law.
    It takes into account, that the thermionic emitted electrons gather around the cathode.
    This space charges partially shields the accelerating potential.
    The space-charge-limited-current (SCLC) can be described by:
    
    \begin{equation}
        \label{eq:theoreticalSchottky}
        j = \frac{4}{9} \epsilon_0 \sqrt {\frac{2e}{m}} \frac{U^\frac{3}{2}}{d^2}
    \end{equation}
    
    The Schottky-Langmuir-law only holds for a limited range.
    Since the function is strictly convex, we extract those data points where we can roughly observe a 
    positive curvature.
    These data points are depicted in ~\autoref{fig:schottky}.
    To verify the Schottky-Langmuir-law we fit the following function:
    \begin{equation}
        \label{eq:Schottky}
        I = a (U_a - U_k)^{b}
    \end{equation}
    
    Here $U_a$ is the accelerating voltage and $a$, $b$ are free parameters.
    The free parameter $U_k$ represents the contact voltage.
    It is the minimum voltage required for current to flow.
    
    \begin{table}[h!]
        \centering
        \begin{tabular}{c | c c}
            $U_h$ & $\sixfiveV$ & $\sevenV$ \\
            \hline 
            Exponent $b$ & $\schottkySixFivePower$ & $\schottkySevenPower$ \\
            $U_k$ & $\schottkySixFiveUk$ & $\schottkySevenUk$
        \end{tabular}
        \caption{These are the result for the fit of the Schottky-Langmuir equation~\eqref{eq:Schottky}.
        The data points inside the estimated SCLC-region are used.
        Compare ~\autoref{fig:schottky}.
        The errors only represent the statistical scattering of the data points around the fit.
        It does not account for error introduced through the estimation of the SCLC-region.}
        \label{tab:schottkyFit}
    \end{table}
    
    For the heating voltage $\sevenV$ the fitted exponent coincides with the theoretical value $\frac{3}{2}$, while
    for $\sixfiveV$ it deviates.
    An explanation for this, could be, that the assumption of a perfect vacuum is not valid.
    Instead, scattering events occur between the electrons and the remaining gas molecules.
    So the diode behaves like a normal resistor.
    This would suggest a blending between Ohm's law (linear) and the Schottky-Langmuir law, resulting in a lower
    exponent than the theoretical value of $\frac{3}{2}$.
    The effect would also be more prominent for lower heating voltages, since fewer electrons are available relative to
    the gas molecule.
    
    We can determine the contact voltage $U_k$ only via the fit, since our voltage source has a minimum output voltage of $\minvoltage$.
    This leads to large error bars for $U_k$, since we have to take data points up to $70 \volt$ into account,
    so we are able to extrapolate the contact voltage.
    
    
    \subsection{Saturation region - Richardson law}
    
    \figrichardson{
    We plot the saturation current depending on the heating voltage, with a acceleration voltage
        $U_a = \saturationAnodeVoltage$.
    Since at the saturation point all available charge carriers are sucked towards the anode,
    the current is proportional to that desinty.
    It can be described via the Richardson equation ~\eqref{eq:theoreticalRichardson}.
    We can calculate the temperature by the empirical approximation ~\eqref{eq:richardsonTemperature}.
    By combining those equations, we can fit the data.
    We get a work function of the anode material $W_a = \richardsonModelWa$.
    The value is higher than the literature value $W_a = \wolframWa$~\cite{wolfram}.
    It suggest, we did not reach the actual saturation point with the maximum accelerating voltage of our
    voltage source.
    }
    
    If you increase the acceleration voltage, at some point all available charge carriers are sucked towards the anode.
    When saturation region is reached, the current does not increase significantly anymore.
    
    The amount of emitted electrons at the cathode can be described with the Richardson equation.
    It is derived by assuming an ideal fermi gas, where the electrons have to overcome the work function
    $W_a$ of the cathode material:
    \begin{equation}
        \label{eq:theoreticalRichardson}
        j = C T^2 e^{- \frac{Wa}{k_B T}}
    \end{equation}
    Here $T$ is the cathode temperature, $C$ the Richardson constant 
    and $k_B = 1.38 \cdot 10^{-23} \frac{\mathrm{J}}{\mathrm{K}} $~\cite{boltzmanSource}.
    
    We used the following formula to calculate the cathode temperature depending on the heating voltage $U_h$:
    \begin{equation}
        \label{eq:richardsonTemperature}
        T = 1616 K + 97 \frac{\mathrm{K}}{\mathrm{V}} U_h
    \end{equation}
    This empirical approximation of the cathode temperature was made by the creators of our setup.
    See ~\cite{instr} for reference.
    
    To measure the saturation behaviour, we apply an acceleration 
    voltage of $U_a = \saturationAnodeVoltage$.
    We then vary the heating voltage from $5 \volt$ upto $7 \volt$ and measure the anode current.
    The data is depicted in ~\autoref{fig:richardson}.
    By combining the Richardson law ~\eqref{eq:theoreticalRichardson} and 
    cathode-temperature equation ~\eqref{eq:richardsonTemperature} we get a relation,
    which we can fit into the data.
    
    We get a work function of $W_a = \richardsonModelWa$.
    This value does not coincide with the literature value $W_a = \wolframWa$~\cite{wolfram} of our wolfram cathode.
    It may be explained by assuming that the maximum acceleration voltage generated by our voltage source is
    insufficient to reach the saturation point for high heating voltages.
    It means we underestimate the current for high heating voltages, so we overestimate the work function $W_a$.
    The fact that we do not reach saturation can also be seen in ~\autoref{fig:characteristicCurve} for $U_a = \sevenV$.
    Furthermore, the empirical approximation of the cathode temperature might be a source of the deviation.
    
    It shows that our setup is able to qualitatively but not quantitatively verify the Richardson law.
    A more accurate setup is required.
    
    \section{The Franck-Hertz effect}

    \begin{figure}[htbp]
        \includegraphics[width=\linewidth]{FranckHertzSketch}
        \center
%        \caption{By varing the optical path length of the rays you see constructive or destructive interference on the
%        detector. This depends on the wavelength of the ligth, which enables you to measure it.
%        Graphic from~\cite{imageMichelsonInterferometerWiki}}
        \caption{Schematic setup of the tetrode, which is filled by either mercury or neon gas.
        The electrons are evapurated at the cathode with the heating voltage $U_h$ and sucked towards
        the grid $G_1$ by the sucking voltage $U_1$.
        They are then accelerated by $U_2$.
        In this region the electrons scatter inelastically at the gas atoms. 
        But only if the kinetic energy of the electrons if high enough to excite the atoms.
        The counter voltage $U_3$ acts as a filter, so only electrons with enough kinetic energy reach
        the anode.
        The anode current $I$ is then measured.
        Image source~\cite{frankHertzSource}.
        }
        \label{fig:franckHertzSketch}
    \end{figure}
    
    In the second part we want to study an experimental setup similar to the previous part,
    with which we can demonstrate the Frank-Hertz effect.
    
    We use a tetrode filled with either mercury or neon gas.
    Compare ~\autoref{fig:franckHertzSketch}.
    The design and function of the tetrode is similar to that of the vacuum diode.
    It consists also of a cathode, where the electrons are vaporized.
    The electrons reaching the anode are measured as current $I$.
    In the tetrode two metal grids are inserted for tighter control of the electrons' kinetic energy.
    The vaporized electrons are sucked towards the first grid by the sucking voltage $U_1$.
    They are then accelerated by the potential $U_2$ between the first and second grid.
    This potential is equal to the total kinetic energy of the electrons.
    Between the second grid and the anode, the counter voltage $U_3$ is applied.
    It acts as a filter, so only electrons with sufficient kinetic energy reach the anode.
    
    The voltage source, we use, generates voltages with a stability of $0.1\volt$.
    We measure the accelerating voltage and anode current with the digital interface `PASCO Universal Interface 550'.
    Furthermore, we measure the cathode temperature with an accuracy of $0.1 \, ^\circ \mathrm{C}$
    
    \subsection{Franck-Hertz effect with mercury}
    
    \figFrankHertzHg{
    Characteristic curve of mercury obtained by the Franck-Hertz experiment.
    We use a tetrode as depicted in ~\autoref{fig:franckHertzSketch}.
    The mercury is vaporized by an oven, which heats the tetrode to $T = \AnodeTemperatureHg$.
    We see periodic declines of the anode current.
    The energy differences between the maxima are summarized in ~\autoref{tab:transitionHg}.
    The declines are caused by the electrons, which scatter at the mercury atoms and excite them from
    the ground state to the $\HgLevelA$ state.
    }

    \begin{table}[httb]
        \centering
        \begin{tabular}{c c}
            \hline \hline 
            Maxima & Energy difference $\Delta E \ \mathrm{in} \eV$ \\
            \hline
            $1 \rightarrow 2$ & $\TransitionHgOneTwoa$ \\
            $2 \rightarrow 3$ & $\TransitionHgTwoThreea$ \\
            $3 \rightarrow 4$ & $\TransitionHgThreeFoura$\\
            $4 \rightarrow 5$ & $\TransitionHgFourFivea$\\
            \hline
            Average & $\TransitionHgMeana$\\
            Theoretical & $\TheoreticalTransitionHga$ \\
            \hline 
            \hline
        \end{tabular}
        \caption{Energy differences between the maxima of the Franck-Hertz-tube filled with mercury gas.
        The maxima are estimated from ~\autoref{fig:FrankHertzHg}.
        The errors derive from manually reading the maxima from the graphic.
        The average coincides with the literature value of the $\HgLevelA$ transiton.
        Compare ~\autoref{fig:energyLevelsHg}.}
        \label{tab:transitionHg}
    \end{table}
    
    To produce the mercury gas we have to heat the tetrode.
    We use an oven, which covers the tetrode and vaporizes the mercury.
    The anode temperature is $T = \AnodeTemperatureHg$.
    
    We continuously vary the acceleration voltage $U_2$ from $0\volt$ to $30\volt$.
    It is limited by the anode current, which 
    may not exceed $I = 10 \, \mathrm{nA}$, since it would break the tetrode.
    The measured data is shown in ~\autoref{fig:FrankHertzHg}.
    
    We observe that the current increases with increasing $U_2$.
    But at certain intervals it drops.
    This is caused by the electrons, which scatter inelastically at the mercury atoms.
    When they scatter, they lose kinetic energy and can not pass the counter voltage $U_3$.
    The current drops.
    
    From ~\autoref{fig:FrankHertzHg} we estimate the maxima.
    The energy differences are displayed in ~\autoref{tab:transitionHg}.
    The mean is $\Delta E = \TransitionHgMeana \eV$.
    It coincides with the energy difference between the ground state and $\HgLevelA$ level
    of mercury.
    The theoretical value is $\TheoreticalTransitionHga \eV$.
    Compare the Grotrian diagram ~\autoref{fig:energyLevelsHg}.
    
    We do not observe any light emitted by the mercury atoms, since the energy $\Delta E$ corresponds to light in the
    ultraviolet spectrum.
    
    
    
    



    %FF: Angabe der verwendeten Literatur mit Quellennachweis.
    \begin{thebibliography}{}    %so wird das Literaturverzeichnis erstellt
        \bibitem{instr} Physikalisches Grundpraktikum, Universitüt Würzburg, Modul C1, Versuch 49, Kennlinien von Vakuumdioden
        und Franck-Hertz-Versuch, 2021
        \bibitem{gerth} Meschede, Dieter, Gerthsen Physik, 25. Auflage, Springer-Verlag, Berlin, 2015
        \bibitem{codata} P. J. Mohr, D. B. Newell, and B. N. Taylor: \grqq CODATA
        recommended values of the fundamental physical constants: 2014\grqq , Rev. Mod. Phys.
        88, 035009 (2016))
        \bibitem{miller} \url{https://de.wikipedia.org/wiki/Datei:Miller_Indizes_Ebenen.png}, zuletzt aufgerufen am 14.09.2021
        \bibitem{pasco} Pasco Microwave Optics System (WA-9314C) Instruction Manual
        %\bibitem{missing} \url{https://en.wikipedia.org/wiki/Transverse_mode}, zuletzt aufgerufen am 8.10.2021
        \bibitem{imageMichelsonInterferometerWiki} \url{https://en.wikipedia.org/wiki/Michelson_interferometer#/media/File:Michelson_interferometer_with_labels.svg}, last visit 23.09.23
        
        \bibitem{wolfram}  Hütte, Das Ingenieurwissen 34. Auflage, Tabelle16-6, Springer Vieweg 2012
        \bibitem{boltzmanSource}  \url{https://physics.nist.gov/cgi-bin/cuu/Value?k}, last visited 11.03.2024
        \bibitem{frankHertzSource} \url{https://images.saymedia-content.com/.image/t_share/MTc0NDM3MzE2ODA5NDY3MjQw/the-frank-hertz-experiment.png}, last visited 11.03.2024
    \end{thebibliography}
    
    \section{Appendix}

    \begin{figure}[htbp]
        \includegraphics[width=\linewidth]{FranckHertzSketch}
        \center
%        \caption{By varing the optical path length of the rays you see constructive or destructive interference on the
%        detector. This depends on the wavelength of the ligth, which enables you to measure it.
%        Graphic from~\cite{imageMichelsonInterferometerWiki}}
        \caption{}
        \label{fig:energyLevelsHg}
    \end{figure}

    \begin{figure}[htbp]
        \includegraphics[width=\linewidth]{FranckHertzSketch}
        \center
%        \caption{By varing the optical path length of the rays you see constructive or destructive interference on the
%        detector. This depends on the wavelength of the ligth, which enables you to measure it.
%        Graphic from~\cite{imageMichelsonInterferometerWiki}}
        \caption{}
        \label{fig:energyLevelsNe}
    \end{figure}


\end{document}