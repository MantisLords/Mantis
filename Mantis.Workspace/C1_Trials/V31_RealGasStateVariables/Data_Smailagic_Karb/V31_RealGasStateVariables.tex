%! Author = User
%! Date = 13.09.2023

% Preamble
\documentclass[a4paper,10pt,twocolumn]{article}

% Packages 
\usepackage[utf8]{inputenc}  %man kann Sonderzeiche wie ü,ö usw direkt eingeben
\usepackage{amsmath}           %macht
\usepackage{amsfonts}          %       Mathe
\usepackage{amssymb}           %              mächtiger
\usepackage{graphicx}          %erlaubt Graphiken einzubinden (.eps für dvi und ps sowie .jpg für pdf)
\usepackage[T1]{fontenc}       %Zeichenbelegung der verwendeten Schrift
\usepackage{ae}                %macht schöneres ß
\usepackage{typearea}
\usepackage{siunitx}
\usepackage{mathtools}
\usepackage{hyperref}
\usepackage{hhline}
\usepackage{caption}
\usepackage{biblatex}
\captionsetup{font=footnotesize}



\usepackage{amsmath}
\usepackage{tikz}
\usepackage{pgfplots}

\newcommand{\TemperatureHumidityCovariance}{102.3\ ^{\circ} C g / m^3}
\newcommand{\TemperatureHumidityCorrelation}{0.8571}
\newcommand{\RegressionOffset}{(1.8 \pm 1.2)\  g / m^3}
\newcommand{\RegressionSlope}{(0.343 \pm 0.057)\ g / m^3 / ^{\circ} C}


\pagestyle{scrheadings}        %sagt Koma-Skript, dass selbstdefiniers Kopfzeilen verwendet werden
\typearea{16}                  %stellt Seitenspiegel ein
\columnsep25pt								 %definiert Breite zwischen den zwei Spalten von \twocolumns

\renewcommand{\pnumfont}{%     %ändert die Schriftart der Seitennummerierung
    \normalfont\rmfamily\slshape}  %ändert die Schriftart der Seitennummerierung 



\begin{document}
    \twocolumn[{\csname @twocolumnfalse\endcsname                %erlaubt "Abstrakt" über beide Spalten
    \titlehead{                                                  %Kopfzeile
        \begin{tabular*}{\textwidth}[]{@{\extracolsep{\fill}}lr}   %Kopfzeile
            Tutor: Jannis Seufert & \today\\                          %Kopfzeile - Betreuer
        \end{tabular*}                                             %Kopfzeile
    }
    \title{Comparing Gas Models by examining Properties at the Critical Point}  %Titel der Versuchs
    \author{Salahudin Smailagić and Thomas Karb}                     %Namen der Studenten
    \date{}                                                         %benötigt um automatisches Datum auszuschalten
    \maketitle                                                      %erzeugt Titelseite
    \vspace{-5ex}                                                   %verringert Abstand zur Überschrift
    \begin{abstract}                                                %Beginn des Abstracts
        In this paper we recorded isotherms of a gas in order to examine their agreement with two current theories as well as compare the theories among themselves.
        As expected, the elementary model of the ideal gas followed the experimental data very poorly, which is why the second model - the Van der Waals model - expands on it by adding two more parameters.
        A method for determining these - gas specific - parameters was explored using the phase transition from gas to liquid.
        From the critical point i.e the point from which the two phases are indistinguishable, the parameters can be calculated.
        The used gas was SF_6\) and the parameters were calculated to be $a=\parameterA \frac{Pa\cdot m^6}{mol^2} $ and $b=\parameterB \frac{m^3}{mol} $ with the critical point values at $V_{\text{Crit}}=\criticalVolume$, $P_{\text{Crit}}=\criticalPressure$ and 
        $T_{\text{Crit}}=\criticalTemp\degree C$.
        We found that the Van der Waals model showed much better agreement with the recorded data but still failed at the phase transition region where the isotherm follows a constant pressure function.
       The functional temperature dependence of the constant pressure regions P(T) was also examined but our setup could not deliver conclusive evidence and verify the theoretical prediction of $P \propto e^{\frac{1}{T}} $.
       Nevertheless, under the assumption that the theoretical dependence is given we could calcualte the evaporation energy of SF_6\) to be $E_{\text{Evap}}=\evaporationEnergyPerMole \frac{J}{mol}$.
       \\
       \\
        Measuerement made: 26. September 2023\\       %Datum ändern!
        Submitted: 10. October 2023                %Datum ändern!
        \\
        \\
    \end{abstract}
    }]
    \section{Introduction}\label{sec:introdction}
    The technical application of gasses has been very important in a variety of engineering branches. 
    A better understanding of their properties through a sufficient theoretical model as well as precise empirical data is of high importance.
    Since every model has its limitations, it is important to know where they are and where the model can be used safely.
    Lastly, creating models with many parameters, in order to show better agreement with data can be done, but oftentimes at the expense of simplicity.
    To compare the two commonly used models, and explore their limitations, Sulfur hexafluoride (SF_6\)) was used.
    
    SF_6\) is a colorless, odorless, non-flammable and otherwise completely inert gas. 
    The molecule has an octahedral geometry consisting of six fluorine atoms attached to a central sulfur atom.
    Its easy to reach temperature and pressure for phase transition from gas to liquid make it ideal for the for testing the limits of our theories in a safe environment.
    For that, the state variables pressure, volume and temperature are measured, allowing later to calculate the critical point of the gas.
    \section{Theory: Ideal gas and Van der Waals equation}\label{sec:theory}
    Modeling a gas with the assumption of infinitely small molecules interacting only by bumping into each other, one gets the ideal gas equation
    \begin{align}
        pV=\nu RT
    \end{align}
    with p being the pressure, V the volume, $\nu$ the amount of substance, R the universal gas constant and T being the temperature of the gas.
    
    Van der Waal expands this theory by adding two more parameters.
    One accounting for the finite volume of each molecule and the second one accounting for a gas specific attractive intermolecular interaction.
    \begin{align}
    (p+\frac{\nu^2 a}{V^2})(V-\nu b) = \nu RT
    \end{align}
    The Van der Waals model, while showing better agreement with experiment, still fails to model the phase transition form gas to liquid.
    This phase transition exhibits a region of coexistence of the two phases. 
    By decreasing the volume in the coexistence region a part of the gas transitions into liquid leaving the pressure unchanged, which can be seen as a horizontal line in the P-V diagram.
    The coexistence region shows an inversely proportional temperature dependence, making it narrower for higher temperatures until vanishing completely when the critical point is reached.
    After that point, the liquid and gaseus phase are indistinguishable.
    \section{Recording of isotherms and determining the critical point}\label{sec:Measurement}
    For the measurements, a chamber with variable volume was used.
    The temperature was regulated with a water-bath, keeping the chamber at a constant predetermined temperature.
    With this construction, a total of 9 isotherms were recorded. 
    The volume was varied from bigger to smaller and the according reading on the manometer was written down.
    It was essential to wait for the thermodynamic equilibrium to set after every manipulation of the parameters, before collecting any data.
    The recorded data can be seen in Fig. \ref{fig:binodal} and Fig. \ref{fig:binodalZoom}.
    \begin{figure}
        \begin{center}
            \includegraphics[scale=0.18]{Data/Generated/fig_plot}\label{fig:binodal}
            \caption[]{Datapoints of the recorded isotherms of $\amountOfSubstance mol$ of SF_6\) at Temperature values ranging from 30.2$\degree C$ to 48.2$\degree C$ obtained by varying the volume and noting the corresponding pressure. The end points of the visible straight lines at the regions of constant pressure were
            used to determine the coexistence region which is shaded in blue here. The critical point is the extremum of the coexistence region. Its value is at $P=\criticalPressure$ and $V=\criticalVolume$ and an estimated
            T=$(46.3\pm0.3)\degree C$}
        \end{center}
    \end{figure}
    \begin{figure}
        \begin{center}
            \includegraphics[scale=0.18]{Data/Generated/plotZoom}\label{fig:binodalZoom}
            \caption[]{An enlarged view of the the critical point of SF_6\) determied from the recorded isotherms whose temperature values range from 43.1$\degree C$ to 48.2$\degree C$}
        \end{center}
        \end{figure}
    The coexistence regions in which the isotherms are horizontal are highlighted and a line - the Maxwell line - was fitted between the beginning and end point of the region.
    The end points were then used to interpolate a curve through them, which marks roughly the coexistence region as a whole.
    This also allowed extrapolation of the critical point of the gas which could not be determined through direct experiment due to the uncertainties.
    The curve marking the region of coexistence is called the binodal.
    The exact steps that were taken to fit the Maxwell lines lines are detailed in the appendix \ref{sec:apendix}. % Make clear it referece to Maxwell Lines
    The extremum of the binodal marks the critical point of the gas. 
    For SF_6\) it was calculated to be at
    \begin{align}
        P_{crit} &= \criticalPressure 
        \endline
        V_{crit} &= \criticalVolume 
        \end{align}
    The estimated temperature at the critical point is $T_{crit} = (46.3 \pm 0.3)$ since the maximum appears to be exactly between the 46.1\degree\) and the 46.6\degree\) isotherms.
    This value is eyeballed since this method does not allow a conclusive method of determining the temperature.
    The errors of the critical values for volume and pressure are estimates as well.
    Since the interpolation error is assumed to be small, the errors are mostly the reading errors of the volume and the pressure.
    For smaller volumes outside the coexistence region the gas exhibits a very rapid pressure increase, this is due to the fact that all the gas has turned into liquid which is incompressible.
    \subsection{Steam pressure curve}\label{subsec:steamPressureCurve}
    \begin{figure}
        \begin{center}
            \includegraphics[scale=0.18]{Data/Generated/fig_tempLogPlot}\label{fig:tempLogPLot}
            \caption[]{Examination of the functional dependence of P(T). The data points are obtained from the regions of constant pressure pairing the constant pressure value with the temperature of the isotherm.
            The assumed exponential dependence $P \propto e^{\frac{1}{T}} $ does seem given on the first look, but the value of $\chi_{\text{Red}}=\residual$ suggest a clear case of overfitting since we have
            too few data points. More data would be needed for more certainty. Form the intersection of the exponential curve and the critical pressure the critical temperature is $T_{\text{Crit}}=\criticalTemp\degree C$}
        \end{center}
    \end{figure}
    From the data in Fig. \ref{fig:binodal} and the constructed maxwell lines one can plot the temperature dependence of the pressure as can be seen in Fig. \ref{fig:tempLogPLot}.
    However, the theoretical form of the steam pressure curve is predicted as follows:
    \begin{align}
        p=b(T)e^\frac{E_{steam}}{k_B T}
    \end{align}
    This prediction can be explained by the Boltzmann distribution of the molecules, since only the molecules that have enough energy to leave the liquid phase evaporate.
    The height of this energy barrier is E_{steam}\) in this case.
    Since we do not have a lot of data-points the apparent exponential $P \propto e^{\frac{1}{T}} $ fit can not give certainty over of the functional dependence P(T).
    This is also obvious from the reduced $\chi$ value which in is in this case $\chi_{\text{Red}}=\residual$ suggesting a clear case of overfitting.
    For more certainty, we would need more data.
    From the fitted exponential parameter, one can calculate the evaporation energy E_{steam}\)
    \begin{align}
        E_{steam} = \evaporationEnergyPerMole \frac{J}{mol}
        \endline \text{with } k_B \text{ being } 1.380649\cdot 10^{-23}\frac{J}{K}
    \end{align}
    The evaporation energy is given in units of energy per mole since the evaporation energy of one molecule is not handy. 
    From the intersection point of the fit with the critical pressure, the critical temperature can be calculated $T_{crit}=\criticalTemp\degree C $.
    The value coincides with the literature value $T_{crit} = 45.54\degree C$ [2].
    This is largely due to the large error which comes from the fitting method.
    This value also coincides with the value estimated form the P-V diagram and is going to be used in all following calculations since the other value is a mere estimate.
    
    \subsection{Gas specific parameters of the Van der Waals model for SF_6\)}\label{subsec:gasSpecParams}
    The parameters a and b in the VdW-model can be calculated using the obtained values from the critical point of the gas (See Appendix. \ref{sec:apendix}).
    Their values are calculated to be $a = \parameterA \frac{Pa m^6}{mol^2}$ and $b =\parameterB \frac{m^3}{mol}$.
    These values do not coincide withe the literature values of $a_{\text{Lit}} = 0.786 \frac{Pa m^6}{mol^2}$ and $b_{\text{Lit}} = 8.79\cdot 10^{-5} \frac{m^3}{mol}$ \cite[crc]{2}.
    This deviation can have multiple causes.
    One possible explanation would be that the used gas was contaminated, meaning it had traces of other molecules in it.
    This could have happened by a leak in the chamber allowing water or air inside.
    Another possible explanation is that we did not wait long enough after varying the volume during our measurements (since we can not wait infinitely long) and by doing so, did not allow the thermodynamic equilibrium to settle.
    This would cause a systematic error in our measurements since the temperature of the gas at the point we measured was always higher than expected.
    It is likely that a combination of these possible causes caused the bad agreement of our data with the literature values.
    However, from our obtained values, the amount of substance could be calculated as well $\nu = \amountOfSubstance mol$.
    Finally, we can use the definition of the critical coefficient
    \begin{align}\label{eq:critcalCoeff}
        K_K = \frac{\nu RT}{pV}
    \end{align}
    to calculate the deviation of SF_6\) from a true VdW-gas.
    But since we assumed SF_6\) to be a VdW gas when we calculated the amount of substance, this would provide us no new information.
    
    \section{Comparing with theory}\label{sec:comaringTheory}
    To compare the two models mentioned in this paper, we plotted them in Fig.  \ref{fig:comparingTheories}.
    \begin{figure}
        \begin{center}
            \includegraphics[scale=0.2]{Data/Generated/comparationPlot}
            \caption[]{Comparing experimental data of $\amountOfSubstance mol$ of SF_6\) at $30.3\degree C$ with the Van der Waals model and the ideal gas model at the same temperature, using the parameters $a=\parameterA \frac{Pa\cdot m^6}{mol^2}$ and $b=\parameterB \frac{m^3}{mol}$ which
            were calculated form previous measurements}
            \label{fig:comparingTheories}
        \end{center}
    \end{figure}
    As expected, the data follows the ideal gas equation very poorly.
    The predicted pressures are high at every point of the graph since there is no modification on the pressure in the model and it fails completely to predict what happens in the phase transition region.
    While both models show better agreement with the data at large volumes, the Van der Waals model does a much better job of predicting the pressure over a lerger pressure region.
    However, the VdW model clearly fails at the phase transition region too.
    This goes to show that while adding an intermolecular interaction greatly improves agreement with experiment, the model still does not cover every situation a gas can be in.
    Perhaps a theory modelling both liquid and gas phase would be able to predict what happens at the phase transition region.
    In theory the Van der Waals model could describe a liquid, but for that, both the parameter governing the intermolecular attraction as well as the parameter correcting for the finite size of the molecules would need to be modified.
    
    While the Maxwell-criterium detailed in Sec.\ref{sec:apendix} helps approximate the behaviour of the gas in the coexistence region, the experimentally measured pressures are significantly lower than the predicted maxwell line.
    For technical purposes it is important to know the limits of the theories or the areas in which they show accurate predictions.
    The Van der Waals model does have its legitimacy, since it greatly improves agreement with experiment by comparably minor additions to the ideal gas model.
    
    
    \section[]{Conclusion}\label{sec:conclusion}
    With this experiment, we were able to compare the ideal gas model and the Van der Waals model with experimental data.
    We succeeded in determining the gas specific parameters of the Van der Walls model, by examining the critical point of the gas.
    The critical point was found using the coexistence regions of the liqiud and gas phase for different temperatures. 
    Since the coexistence region gets narrower for higher temperatures until completely vanishing at exactly the critical point, we recorded isotherms for different temperatures
    and approximated where the coexistence region lies in the P-V diagram.
    The extremum of the approximated region marks the critical point.
    The critical values for our used gas SF_6\) came out to be $P_{\text{Crit}}=\criticalPressure$ , $V_{\text{Crit}}=\criticalVolume$ and $T_{\text{Crit}}=(46.3\pm 0.3)\degree$, from which the parameters of the Van der Waals model could be determined
    $a=\parameterA \frac{Pa m^6}{mol^2} $ and $b=\parameterB \frac{m^3}{mol} $.
    These were used to compare the two models to experimental data.
    The Van der Waals model proved legitimate by showing greater agreement experimental data over a larger range of volumes.
    Nevertheless, the model still failed at predicting what happens at the phase transition.
    Even the Maxwell-criterium, which is supposed to offer a theoretical approximation of the isotherm and show improvement over the Van der Waals prediction,
    predicted a significantly higher pressure than the measured one.
    From the pressures at the coexistence region and the temperature of the according isotherm, temperature-pressure pairs could be formed to determine the functional dependence.
    The expected $P \propto e^{\frac{1}{T}} $ dependence could not be verified definitively.
    Reason being, that we could only use 5 data-points for the fit which is not enough.
    Under the assumption that the theoretical dependence is true we calculated the evaporation energy of the used gas $E_{\text{Evap}} =\evaporationEnergyPerMole $.
    Furthermore, the critical temperature could be calculated (as opposed to guessed, like in Sec \ref{sec:Measurement}) to be $T_{crit}=\criticalTemp \degree$.
    The graphical comparison of the two models and the experimental data showed the advantages of the Van der Waals model over the ideal gas model since it did a much better job predicting
    the pressure.
    However, both models failed at predicting the coexistence region at the phase transition from gas to liquid.
    Perhaps, an expansion on the Van der Waals model, which could model both the liquid and gas phase, would be needed in order to predict the phase transition.
    Nonetheless, the Van der Waals model showed its legitimacy, since it showed greater agreement with expariment over a larger volume region, while introducing comparatively small
    corrections to the Ideal gas model.
    \begin{thebibliography}{}    %so wird das Literaturverzeichnis erstellt
        \bibitem{instr} Physikalisches Grundpraktikum, Universität Würzburg, Modul C1, Versuch 31, Zustandsgrößen realer Gase, 2021
        \bibitem{crc} D. R. Lide: \grqq CRC Handbook of Chemistry
        and Physics\grqq , CRC Press, Boca Raton, 84th
        edition, 2004
        \bibitem{dilo} \url{https://dilo.eu/sf6-gas}, zuletzt aufgerufen am 02.10.2023
    \end{thebibliography}
    \clearpage
    \section[]{Appendix}\label{sec:apendix}
    \subsection{Critical parameters and the VdW parameters}\label{subsec:criticalParams}
    The parameters of the Van der Waals model are directly dependent on the critical parameters of the gas.
    Since the critical point is the point from which the liquid and gas phase are indistinguishable, the Van der Waals isotherm has a saddle point there.
    Mathematically this is given by:
    \begin{equation}\label{eq:anh1}
    \begin{split}
        0 & \overset{!}{=}\frac{dp}{dV}\\
        & =-\frac{\nu R T}{(V-\nu b)^2}+\frac{2 \nu^2 a}{V^3}
    \end{split}
    \end{equation}
\begin{equation}\label{eq:anh2}
\begin{split}
    0 & \overset{!}{=}\frac{d^2p}{dV^2}\\
    & =\frac{2 \nu R T}{(V-\nu b)^3}-\frac{b \nu^2 a}{V^4}
\end{split}
\end{equation}
These two equations can be solved for $\nu$RT and set equal, yielding:
\begin{align}
    V_\text{Crit} = 3\nu b
\end{align}
Put into Eq. (7) allows to calculate $T_{crit}$ and similarly $P_{crit}$
\begin{align}
    T_\text{Crit}=\frac{8a}{27bR}
\end{align}
\begin{align}
    p_\text{Crit}=\frac{a}{27b^2}
\end{align}
    From these equations the values of a and b can easily be obtained.
    \subsection{Maxwell lines method}\label{subsec:maxwellMethod}
    \subsubsection{From the data}\label{subsubsec:maxwellFormData}
    The experimental determination of the maxwell lines- like we did in \ref{sec:Measurement} is done by  grouping all the points which are seperated by less or equal the reading error (in this case 0.25 bar).
    Then, the pressure mean of these points is calculated and the line is placed at that pressure value. 
    Lastly, the end points of the line are set as the outermost points whose pressure difference with the line is less or equal the reading error.
    These end points are then used as data-points through which a the binodal is fitted.
    \subsubsection{Theoretically}\label{subsubsec:maxwellFromTheory}
    The Van der Waals model allows an approximation of the maxwell line since the area between the Van der Waals curve and a the maxwell line should be zero.
    Mathematically that means
    \begin{align}
        \int_{V1}^{V2} |VdW - P| dV
        \end{align}
    which delivers a condition for the pressure P:
    \begin{align}
        P=\log(\frac{V2-b}{V1-b}) RT - \frac{a}{v^2}
    \end{align}
    This condition along with Eq. (2) remaining fulfilled at the volumes V1 and V2 gives a system of equations which can be solved numerically.
    Since solving the system of highly nonlinear equations with three free parameters brute force is tricky, we evaded that by implementing a different algorithm
    which first finds the pressure region in which the constant pressure function has three intersections with the VdW curve and then minimizes the area function
    with the two outer intersection points being the borders of integration.
    The obtained value for P is $(59.335\pm 0.002) bar$ while the error is due to the numerical method.
    This was visualized in Fig. \ref{fig:homeworkPlotOne} in which the VdW curve with the maxwell line can be seen as well as the ideal gas curve for reference.
    \begin{figure}
        \begin{center}
            \includegraphics[scale=0.2]{Data/Generated/HomeworkPlotFirst}
            \caption[]{Theoretical isoterms of CO_2\) with following parameters:
            T = $288.7K$, $\nu=1 mol$, $a=3.658\cdot 10^6 \frac{bar cm^6}{mol^2}$, $b=42.75 \frac{cm^3}{mol}$, $P_{Maxwell}=59.334 bar$}
            \label{fig:homeworkPlotOne}
        \end{center}
    \end{figure}
    
\end{document}