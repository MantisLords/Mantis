%! Author = User
%! Date = 13.09.2023

% Preamble
\documentclass[a4paper,10pt,twocolumn]{article}

% Packages 
\usepackage[utf8]{inputenc}  %man kann Sonderzeiche wie ü,ö usw direkt eingeben
\usepackage{amsmath}           %macht
\usepackage{amsfonts}          %       Mathe
\usepackage{amssymb}           %              mächtiger
\usepackage{graphicx}          %erlaubt Graphiken einzubinden (.eps für dvi und ps sowie .jpg für pdf)
\usepackage[T1]{fontenc}       %Zeichenbelegung der verwendeten Schrift
\usepackage{ae}                %macht schöneres ß
\usepackage{typearea}
\usepackage{siunitx}
\usepackage{mathtools}
\usepackage{hyperref}
\usepackage{hhline}



\usepackage{amsmath}
\usepackage{tikz}
\usepackage{pgfplots}

\newcommand{\alphaNoError}{(4.047 \pm 0.036)}
\newcommand{\betaNoError}{(-4.73 \pm 0.29) \cdot 10^{-3}}
\newcommand{\halfTimeNoError}{(146.5 \pm 9.1)\ s}
\newcommand{\alphaGauss}{(4.04 \pm 0.10)}
\newcommand{\betaGauss}{(-4.62 \pm 0.95) \cdot 10^{-3}}
\newcommand{\halfTimeGauss}{(150 \pm 31)\ s}
\newcommand{\alphaPoisson}{(4.05 \pm 0.10)}
\newcommand{\betaPoisson}{(-4.75 \pm 0.95) \cdot 10^{-3}}
\newcommand{\halfTimePoisson}{(146 \pm 29)\ s}
\newcommand{\symN}{\delta N}


\pagestyle{scrheadings}        %sagt Koma-Skript, dass selbstdefiniers Kopfzeilen verwendet werden
\typearea{16}                  %stellt Seitenspiegel ein
\columnsep25pt								 %definiert Breite zwischen den zwei Spalten von \twocolumns

\renewcommand{\pnumfont}{%     %ändert die Schriftart der Seitennummerierung
    \normalfont\rmfamily\slshape}  %ändert die Schriftart der Seitennummerierung 



\begin{document}
    \twocolumn[{\csname @twocolumnfalse\endcsname                %erlaubt "Abstrakt" über beide Spalten
    \titlehead{                                                  %Kopfzeile
        \begin{tabular*}{\textwidth}[]{@{\extracolsep{\fill}}lr}   %Kopfzeile
            Betreuer: Lukas Elter & \today\\                          %Kopfzeile - Betreuer
        \end{tabular*}                                             %Kopfzeile
    }
    \title{Phase Transition and Critical Points of SF_6}  %Titel der Versuchs
    \author{Salahudin Smailagić and Thomas Karb}                     %Namen der Studenten
    \date{}                                                         %benötigt um automatisches Datum auszuschalten
    \maketitle                                                      %erzeugt Titelseite
    \vspace{-5ex}                                                   %verringert Abstand zur Überschrift
    \begin{abstract}                                                %Beginn des Abstracts
       
        \\
        Measuerement made: 26. September 2023\\       %Datum ändern!
        Submitted: 26. September 2023                %Datum ändern!
        \\
        \\
    \end{abstract}
    }]
    \section{Introduction}\label{sec:introdction}
    The technical application of gasses has been very important in a variety of engineering branches. 
    A better understanding of their properties through a \texit{good} theoretical model as well as precise empirical data is of high importance.
    The most trivial model of a gas is that of the ideal gas, which assumes infinitely small molecules interacting only by bumping the walls of the container.
    The comparison with experimental data reveals the limitations of the model as its \textit{region of applicability} is small and it fails completely at modeling phase transitions.
    An expansion of the model which tries to correct for the finite volume of the molecules as well as adding an intermolecular interaction by adding two more parameters was developed by Van der Waal.
    This model shows greater agreement with data but still fails to model a phase transition or a state of coexistence of two phases.
    Nonetheless, the Van der Waals model allows an approximation of the coexistence region by a maxwells methods which we will discuss in this paper.
    To compare the two models mentioned above, and explore their limitations Sulfur hexafluoride (SF_6\)) was used.
    
    SF_6\) is a colorless, odorless, non-flammable and otherwise completely inert gas. 
    The molecule has an octahedral geometry consisting of six fluorine atoms attached to a central sulfur atom.
    Its easy to reach temperature and pressure for phase transition from gas to liquid make it ideal for the for testing the limits of our theories in a safe environment.
    For that, the state variables pressure, volume and temperature are measured, allowing later to calculate the critical point of the gas.
    \section{Theory: Ideal gas and Van der Waals equation}\label{sec:theory}
    Modeling a gas with the assumption of infinitely small molecules interacting only by bumping into each other, one gets the ideal gas equation
    \begin{align}
        pV=\nu RT
    \end{align}
    with p being the pressure, V the volume, $\nu$ the amount of substance, R the universal gas constant and T being the temperature of the gas.
    
    Van der Waal expands this theory by adding two more parameters.
    One accounting for the finite volume of each molecule and the second one accounting for a gas specific attractive intermolecular interaction.
    \begin{align}
    (p+\frac{\nu^2 a}{V^2})(V-\nu b) = \nu RT
    \end{align}
    The Van der Waals model, while showing better agreement with theory still fails to model the phase transition form gas to liquid which exhibits a region of coexistence of the two phases. 
    By decreasing the volume in the coexistence region a part of the gas transitions into liquid leaving the pressure unchanged, which can be seen as a horizontal line in the P-V diagram.
    The coexistence region shows an inversely proportional temperature dependence making it narrower for higher temperatures until vanishing completely when the critical point is reached.
    After that point, the liquid and gaseus phase are indistinguishable.
    \section{Recording of isotherms and determining the critical point}\label{sec:Measurement}
    \begin{figure}
        \label{fig:construction}
        \begin{center}
        \includegraphics[scale=0.8]{Data/Generated/Construction}
        \caption[]{The Construciton of the measurement with}
        \end{center}
    \end{figure}
    With this Construction, a total of 9 isotherms were recorded. 
    For the recording it was necessary to keep the pressure chamber on constant temperature which was done using an external thermostat.
    The volume was then set with the Hg pillar and the according reading on the manometer was written down.
    It was essential to wait for the thermodynamic equilibrium to set after every manipulation of the parameters, before collecting any data.
    The redorded data can be seen in Fig. \ref{fig:binodal} and Fig.  \ref{fig:binodalZoom}.
    \begin{figure}
        \begin{center}
            \includegraphics[scale=0.2]{Data/Generated/fig_plot}\label{fig:binodal}
            \caption[]{The isoterms}
            \includegraphics[scale=0.2]{Data/Generated/plotZoom}\label{fig:binodalZoom}
            \caption[]{Zoomed in view}
        \end{center}
    \end{figure}
    The coexistence regions in which the isotherms are horizontal are highlighted and a line was drawn between the beginning and end point of the region.
    The end points were then used to interpolate a curve through them, which marks roughly the coexistence region as a whole as well as allowing extrapolation of the critical point of the gas which could not be
    determined through direct experiment due to the uncertainties.
    The curve marking the region of coexistence is called the binodal.
    The exact algorithm which was used is explained in detail in the Appendix.
    The extremum of the binodal marks teh critical point of the gas, SF_6\) in this case and is calculated to be at
    \begin{align}
        P_{crit} = \criticalPressure 
        \endline
        V_{crit} = \criticalVolume 
        \end{align}
    The estimated temperature at the critical point is $T_{crit} = (46.3 \pm 0.3)$ since the maximum appears to be exactly between the 46.1\degree\) and the 46.6\degree\) isotherms.
    This is value is eyeballed since this method does not allow a more precise conclusion.
    The errors of the critical values for volume and pressure are estimates as well.
    Since the interpolation error is assumed to be small, the errors are mostly the reading errors of the Hg pillar and the manometer.
    For smaller volumes outside the coexistence region the gas exhibits a very rapid pressure increase, this is due to the fact that all the gas has turned into liquid which is incompressible.
    \subsection{Steam pressure curve}\label{subsec:steamPressureCurve}
    From the data in Fig. \ref{fig:binodal} and the constructed maxwell horizontals one can plot the temperature dependence of the pressure as can be seen in the inlet in Fig. \ref{}.
    However, the theoretical form of the steam pressure curve is predicted as follows:
    \begin{align}
        p=b(T)e^\frac{E_{steam}}{k_B T}
    \end{align}
    This prediction can be explained by the Boltzmann distribution of the molecules, since only the molecules that have enough energy to leave the liquid phase evaporate.
    The height of this energy barrier is E_{steam}\) in this case.
    In the chosen depicting of the data, it seems they follow a linear trend allowing us to assume the model is right.
    With this, one can calculate the evaporation energy E_{steam}\) from the fitted exponential curve to be
    \begin{align}
        E_{steam} = \evaporationEnergyPerMole \frac{J}{mol}
        \endline \text{with } k_B \text{ being } 1.380649\cdot 10^{-23}\frac{J}{K}
    \end{align}
    The evaporation energy is given in units of energy per mole since the evaporation energy of one molecule is not handy. 
    From the intersection point with the critical pressure, the critical temperature can be calculated $T_{crit}=\criticalTemp K$.
    The value coincides with the literature value $T_{crit} = 45.54\degree C$ which is mostly due to the large error which comes from the fitting method.
    This value also coincides with the value estimated form the P-V diagram and is going to be used in all following calculations since the other value is a mere estimate.
    
    \subsection{Gas specific parameters for the Van der Waals model for SF_6\)}\label{subsec:gasSpecParams}
    The parameters a and b in the VdW-model can be calculated using the obtained values from the critical point of the gas (See Appendix. \ref{})
    Their values are calculated to be $a = \parameterA \frac{Pa m^6}{mol^2}$ and $b =\parameterB \frac{m^3}{mol}$.
    These values do not coincide withe the literature values of $a_{Lit} = 0.786 \frac{Pa m^6}{mol^2}$ and $b_{Lit} = \frac{m^3}{mol}$.
    From these, the amount of substance can be calculated as well $\nu = \amountOfSubstance mol$.
    Finally, we can use the definition of the critical coefficient
    \begin{align}\label{eq:critcalCoeff}
        K_K = \frac{\nu RT}{pV}
    \end{align}
    to calculate the deviation of SF_6\) from a true VdW-gas.
    But since we assumed SF_6\) to be a VdW gas when we calculated the amount of substance, this would provide us no new information.
    
    \section{Comparing with theory}\label{sec:comaringTheory}
    To compare the two models mentioned in this paper, we plotted them in Fig.  \ref{fig:comparingTheories}.
    \begin{figure}
        \begin{center}
            \includegraphics[scale=0.2]{Data/Generated/comparationPlot}
            \caption[]{comparing theories}
            \label{fig:comparingTheories}
        \end{center}
    \end{figure}
    As expected, the data follows the ideal gas equation very poorly.
    The predicted pressures are high at every point of the graph since there is no modification on the pressure in the model and it fails completely to predict what happens in the phase transition region.
    The Van der Waals model on the other hand, does a much better job of predicting the pressure although it also predicts slightly higher values than the observed ones.
    It is likely that the parameter governing the attractive intermolecular force does not modify the pressure enough.
    The model clearly fails at the phase transition region where the predicted maxwell line should divide the VdW curve in two equal parts, which could not be observed.
    This goes to show that while adding a intermolecular interaction greatly improves agreement with experiment, the model still does not cover every situation a gas can be in.
    Both models show better agreement with the data at large volumes.
    \section[]{Conclusion}\label{sec:conclusion}
    We were able to examine the properties of SF_6\) with our measurements.
    By recording isotherms at different volumes and according pressures we were able to extrapolate the critical point of SF_6\) in the P-V diagram.
    The critical values came out to be $P_{crit}=\criticalPressure$ , $V_{crit}=\criticalVolume$ and $T_{crit}=(46.3\pm 0.3)$.
    From the pressures at the coexistence region and their according temperatures pressure-temperature tuples could be used to determine their functional
    relation revealing to be exponential.
    Form the exponential fit the evaporation energy could be calculated and with that a more precise value for the critical temperature could be determined
    $T_{crit}=\criticalTemp \degree$.
    From the critical parameters, the free parameters of the Van der Waals model $a=\parameterA \frac{Pa m^6}{mol^2} $ and $b=\parameterB \frac{m^3}{mol} $ could be calculated, as well as the amount of substance
    $\nu = \amountOfSubstance mol$.
    The graphical comparison of the two models and the experimental data showed the advantages of the Van der Waals model over the ideal gas model since it did a much better job predicting
    the pressure.
    However, both models failed at predicting the coexistence region at the phase transition from gas to liquid.
    \begin{thebibliography}{}    %so wird das Literaturverzeichnis erstellt
        \bibitem{instr} Physikalisches Grundpraktikum, Universität Würzburg, Modul C1, Versuch 31, Zustandsgrößen realer Gase, 2021
        \bibitem{crc} D. R. Lide: \grqq CRC Handbook of Chemistry
        and Physics\grqq , CRC Press, Boca Raton, 84th
        edition, 2004
        \bibitem{dilo} \url{https://dilo.eu/sf6-gas}, zuletzt aufgerufen am 02.10.2023
    \end{thebibliography}
    \newpage
    \section[]{Appendix}\label{sec:apendix}
    \subsection{Critical parameters and the VdW parameters}\label{subsec:criticalParams}
    The parameters of the Van der Waals model are directly dependent on the critical parameters of the gas.
    Since the critical point is the point from which the liquid and gas phase are indistinguishable, the Van der Waals isotherm has a saddle point there.
    Mathematically this is given by:
    \begin{equation}\label{eq:anh1}
    \begin{split}
        0 & \overset{!}{=}\frac{dp}{dV}\\
        & =-\frac{\nu R T}{(V-\nu b)^2}+\frac{2 \nu^2 a}{V^3}
    \end{split}
    \end{equation}
\begin{equation}\label{eq:anh2}
\begin{split}
    0 & \overset{!}{=}\frac{d^2p}{dV^2}\\
    & =\frac{2 \nu R T}{(V-\nu b)^3}-\frac{b \nu^2 a}{V^4}
\end{split}
\end{equation}
These two equations can be solved for $\nu$RT and set equal, yielding:
\begin{align}
    V_\text{Crit} = 3\nu b
\end{align}
Put into Eq. (7) allows to calculate $T_{crit}$ and similarly $P_{crit}$
\begin{align}
    T_\text{Crit}=\frac{8a}{27bR}
\end{align}
\begin{align}
    p_\text{Crit}=\frac{a}{27b^2}
\end{align}
    From these equations the values of a and b can easily be obtained.
    \subsection{Maxwell horizontals method}\label{subsec:maxwellMethod}
    \subsubsection{From the data}\label{subsubsec:maxwellFormData}
    Explain and add a picture
    \subsubsection{Maxwell horizontals theoretically}\label{subsec:maxwellFromTheory}
    The Van der Waals model allows an aproximation of the maxwell horizontal since the area between the maxwell curve and the Van der Waals equation is should be zero integrated over the
    coexistence region.
    \begin{figure}
        \begin{center}
            \includegraphics[scale=0.2]{Data/Generated/HomeworkPlotFirst}
            \caption[]{comparing theories}
            \label{fig:homeworkPlotOne}
        \end{center}
    \end{figure}
    
\end{document}