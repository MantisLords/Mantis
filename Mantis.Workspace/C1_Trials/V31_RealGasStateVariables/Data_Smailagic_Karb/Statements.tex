%! Author = User
%! Date = 23.10.2023

% Preamble
\documentclass[a4paper,10pt]{article}

% Packages
\usepackage[utf8]{inputenc}  %man kann Sonderzeiche wie ü,ö usw direkt eingeben
\usepackage{amsmath}           %macht
\usepackage{amsfonts}          %       Mathe
\usepackage{amssymb}           %              mächtiger
\usepackage{graphicx}          %erlaubt Graphiken einzubinden (.eps für dvi und ps sowie .jpg für pdf)
\usepackage[T1]{fontenc}       %Zeichenbelegung der verwendeten Schrift
\usepackage{ae}                %macht schöneres ß
\usepackage{typearea}
\usepackage{amstex}
\usepackage{siunitx}
\usepackage{mathtools}
\usepackage{hyperref}
\usepackage{hhline}	         %ermöglicht änderung des Seitenspiegels
\usepackage{caption}
\usepackage{biblatex}
\usepackage{anyfontsize}
% Document
\begin{document}
    \vspace*{\stretch{1.0}}
    \begin{center}
        \Large\textbf{Statement}\\
        \large\text{S. Smailagić and T. H. Karb}\\
        \normal\today
    \end{center}
    \vspace*{\stretch{2.0}}

    We have submitted the paper and hereby make statements to the Peer-Review committee.
    The preliminary peer-review was received on the $19^{\text{th}}$ of October.

    \begin{enumerate}
        \item 
        a) The Error bars to the end points of the Maxwell lines were added and can be seen in the plots.
        
        b) The interpolation procedure was changed from a spline to a functional fit.
        The fitted function was chosen because it only used 4 parameters to fit the data.
        Nonetheless, the functional dependence is completely arbitrary and based on no physical background.
        The only reason the functional fit is better than the spline is reproducibility.
        
        c) Our estimation that the interpolation error is negligible compared to the reading error is wrong.
        We are now more aware of the errors which arise from the fitting method.
        For this reason, the errors of the critical point were estimated.

        \item The error bars for the extrapolated point were added. And the point for the $46.6\degree$ isotherm was added.
        We still think that from the reduced $\chi$ we can not say with certainty that the data follows the functional dependence, because we had similar results when linearising the data and fitting a line throuh them.
        This is due to the fact that the errors of the datapoints dont transform right under the linearisation, introducing other uncertainties.
        The error for the critical temperature was simply calculated wrong.
        We initially calculated it using the parameters from our regression, that is wrong because it ignores the whole covariance matrix for the error.
        If we evaluate the function at the given pressure right, the error is calculated using the proper covariance matrix.
        
        \item Critique taken into account.
    \end{enumerate}
        \textbf{Comments regarding the structure of the paper}
    \begin{enumerate}
        \item Critique taken into account.
        \item Critique taken into account.
        \item Critique taken into account.
        \item Critique taken into account.
    \end{enumerate}
    


    

\end{document}