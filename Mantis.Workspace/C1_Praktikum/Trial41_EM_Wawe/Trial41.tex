%! Author = User
%! Date = 12.09.2023

% Preamble
\documentclass[a4paper,10pt,twocolumn]{article}

% Packages
\usepackage{german}            %macht deutsche überschriften
\usepackage[utf8]{inputenc}  %man kann Sonderzeiche wie ü,ö usw direkt eingeben
\usepackage{amsmath}           %macht
\usepackage{amsfonts}          %       Mathe
\usepackage{amssymb}           %              mächtiger
\usepackage{graphicx}          %erlaubt Graphiken einzubinden (.eps für dvi und ps sowie .jpg für pdf)
\usepackage[T1]{fontenc}       %Zeichenbelegung der verwendeten Schrift
\usepackage{ae}                %macht schöneres ß
\usepackage{typearea}	         %ermöglicht änderung des Seitenspiegels



\pagestyle{scrheadings}        %sagt Koma-Skript, dass selbstdefiniers Kopfzeilen verwendet werden
\typearea{16}                  %stellt Seitenspiegel ein
\columnsep25pt								 %definiert Breite zwischen den zwei Spalten von \twocolumns

\renewcommand{\pnumfont}{%     %ändert die Schriftart der Seitennummerierung
    \normalfont\rmfamily\slshape}  %ändert die Schriftart der Seitennummerierung 



\begin{document}
    \twocolumn[{\csname @twocolumnfalse\endcsname                %erlaubt "Abstrakt" über beide Spalten
    \titlehead{                                                  %Kopfzeile
        \begin{tabular*}{\textwidth}[]{@{\extracolsep{\fill}}lr}   %Kopfzeile
            Betreuer: Ronny Thomale & \today\\                          %Kopfzeile      hier den Betreuer eintragen!!!
        \end{tabular*}                                             %Kopfzeile
    }
    \title{Der Steppenwolf, eine Analyse}  %Titel der Versuchs
    \author{Salahudin Smailagić und Thomas Karb}                     %Namen der Studenten
    \date{}                                                         %benötigt um automatisches Datum auszuschalten
    \maketitle                                                      %erzeugt Titelseite
    \vspace{-8ex}                                                   %verringert Abstand zur Überschrift
    \begin{abstract}                                                %Beginn des Abstracts
        Hermann Hesse wurde am 2. Juli 1877 im württembergischen Calw geboren. Schon mit seinem ersten Roman Peter Camenzind (1904) war er so erfolgreich, dass er als freier Schriftsteller leben konnte. Er ließ sich am Bodensee nieder, zog 1912 in die Schweiz und siedelte 1919 nach Montagnola im Tessin um, wo er bis zu seinem Tod 1962 lebte. 1946 bekam er, auch wegen seiner moralisch untadeligen Haltung während der Naziherrschaft, den Nobelpreis für Literatur. Die meisten seiner Werke beschreiben Einzelgänger, die gegen die Zwänge der Gesellschaft ihren eigenen Weg suchen. Die Ich-Suche, die Entwicklung der Persönlichkeit und die Selbstverwirklichung sind stets Hesses Hauptanliegen gewesen. Diese Themen bilden die Schwerpunkte in seinen Romanen und Erzählungen und haben ein breites Publikum gefunden: Hesse ist weltweit der meistgelesene deutschsprachige Autor des 20. Jahrhunderts. Auch in den USA und Japan werden seine Werke immer wieder neu aufgelegt, Der Steppenwolf ist einer seiner bekanntesten und erfolgreichsten Romane. Aufgrund der vielen Parallelen zwischen Autor und Protagonisten wird die Erzählung oft autobiographisch gedeutet. Hesse befasste sich 1924–27 mit der Steppenwolf-Thematik, stellte den Roman am 11. Januar 1927 fertig, die erste Auflage erschien im Mai dieses Jahres, kurz vor seinem fünfzigsten Geburtstag. Viele Begebenheiten, die im Roman vorkommen, beruhen tatsächlich auf Hesses eigenen Erlebnissen in \\ \\
        \\
        Versuchsdurchführung: 19. September 2023\\       %Datum ändern!
        Protokollabgabe: 26. September 2023                %Datum ändern!
        \\
        \\
    \end{abstract}
    }]
    \section{Einleitung}
    Der Steppenwolf erzählt von dem fast fünfzigjährigen 
    Harry Haller. Harry lebt einige Monate in einer Schweizer Stadt zur Untermiete.
    Vereinsamt, verbittert und krank bleibt er für sich und besucht nur hin und wieder
    Konzerte und die Bibliotheken der Stadt, selten trifft er andere Menschen. Dieses 
    Außenseiterdasein, die Verzweiflung darüber und die innere Zerrissenheit des Protagonisten 
    sind Hauptthema der Erzählung. Harry, der zwischen seinem Bedürfnis nach einem bürgerlichen,
    geselligen Leben und seinen animalischen, steppenwölfischen Trieben hin- und hergerissen wird, 
    ist auf der Suche nach seiner wahren Identität. Als er Hermine begegnet, einer hübschen jungen Frau,
    die sich von Männern aushalten lässt, ändert sich sein Leben. Er lernt tanzen, genießt die körperliche Liebe 
    und besucht öffentliche Veranstaltungen. Von Hermine und deren Freund, dem Jazz-Saxophonisten Pablo,
    wird er schließlich in ein magisches Theater geführt, das ihm auf der Suche zu sich selbst voranbringen soll.
    Es besteht aus vielen Räumen, in denen Harry verschiedenen Facetten seiner Persönlichkeit begegnet und die vielen
    Varianten des Lebens kennen- und schätzen lernt. Fast am Ende seiner Reise angekommen, lässt er sich von all den 
    Trugbildern des Theaters täuschen, fällt zurück in sein verbittertes Dasein und ersticht Hermine.
    Trotz vieler Rückschläge und Verwirrungen besinnt sich Harry letztlich doch noch auf die neu entdeckten Freuden, will mit Hilfe seiner Freunde und Idole aus seinen Fehlern lernen und seine negativen Ansichten zukünftig bessern. Die Geschichte endet mit einem positiven Lebensausblick. Der Roman ist in drei Teile gegliedert: Zunächst erhält der Leser im »Vorwort des Herausgebers« einige Informationen über den Protagonisten, dessen Wesen und Alltag. Es folgen »Harry Hallers Aufzeichnungen«, die nun von ihm selbst verfasst sind und Einblicke in seine Gedanken- und Gefühlswelt geben. Diese Aufzeichnungen werden durch den »Tractat vom Steppenwolf« unterteilt, der das in der Ich-Form geschilderte Geschehen aus einer weiteren, scheinbar objektiven Perspektive betrachtet und kommentiert.
\section{Theorie}
    Der Herausgeber will den Aufzeichnungen Harry Hallers, die er nach dessen Auszug aus der Wohnung fand, einige eigene Erinnerungen voranstellen. Er nennt ihn häufig »Steppenwolf«, nach einem Ausdruck Harrys selbst. Harry mietet bei der Tante des Herausgebers für neun oder zehn Monate zwei möblierte Zimmer in ihrem Haus. Zufällig ist der Neffe der Vermieterin, der Herausgeber der Aufzeichnungen, anwesend, als Harry nach dem Zimmer fragt: »Ich habe den sonderbaren und sehr zwiespältigen Eindruck nicht verges █Zwiespältiger Eindruck12 2. Inhaltsangabe sen, den er mir beim ersten Begegnen machte.« (S. 8) Dem Neffen fällt der unentschlossene Gang auf, das scharfe Profil und ein eigentümliches Lächeln: »alles schien ihm [Harry] zu gefallen und schien ihm doch zugleich irgendwie lächerlich« (S. 9). Als der Neffe am Abend von der Arbeit zurückkommt, erzählt ihm die Tante, dass der Mann ihr sympathisch ist. Harry bekommt die Wohnung. 
    In den nächsten Tagen spioniert der Neffe hinter dem neuen Mieter her, bewundert ihn und sein »interessantes, höchst bewegtes, ungemein zartes und sensibles Seelenleben« (S. 13). Der Neffe berichtet chronologisch von seinen Begegnungen mit Harry. Er hält ihn für »geistes- oder gemüts- oder charakterkrank« (S. 16). Später erkennt er, dass er »ein Genie des Leidens« (S. 16) ist, aber nicht aus Weltverachtung, sondern aus Selbstverachtung. Er meint, Harry hätte sicherlich strenge und sehr fromme Eltern gehabt, die seinen Willen brechen wollten. Das sei ihnen nicht gelungen, stattdessen hätten sie ihn zum Selbsthass erzogen. Er berichtet auch von Harrys Gewohnheiten und wirft sogar einen Blick in dessen Wohnung: Dort hängen Aquarelle, Fotos von süddeutschen Städtchen und einer hübschen jungen Frau. Bücher liegen und stehen überall, er holt sie sich aus Bibliotheken und bekommt viele per Post. Oft trinkt er Wein, steht morgens spät auf, isst manchmal den ganzen Tag nichts. Gesundheitliche Probleme fallen dem Neffen ebenfalls auf, besonders die »Hemmung in den Bei Sensibles Seelenleben13 2. Inhaltsangabe nen« (S. 20); Treppen steigen kann er nicht besonders gut.
\begin{align}
    E=mc^2
\end{align}
\end{document}