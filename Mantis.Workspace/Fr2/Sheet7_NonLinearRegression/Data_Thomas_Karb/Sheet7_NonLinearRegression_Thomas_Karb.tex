%! Author = Thomas_normal
%! Date = 06.06.2023

% Preamble
\documentclass[11pt]{article}
\usepackage{hyperref}

% Packages

\usepackage{amsmath}
\usepackage{tikz}
\usepackage{pgfplots}

\newcommand{\TemperatureHumidityCovariance}{102.3\ ^{\circ} C g / m^3}
\newcommand{\TemperatureHumidityCorrelation}{0.8571}
\newcommand{\RegressionOffset}{(1.8 \pm 1.2)\  g / m^3}
\newcommand{\RegressionSlope}{(0.343 \pm 0.057)\ g / m^3 / ^{\circ} C}


% Document
\begin{document}

    \author{Thomas Karb}
    \title{Fr2 - Sheet 2 - Non-linear regression}
    
    \maketitle

    \section{Function}
    We want to find the best fit for a decay data:
    \input{Generated/tab_RadioDecayData}
    For the error in N we can assume a poisson distribution so the error is simply the square root of the mean.
    Our model for the data is an exponential function:
    \begin{align*}
        N(t) = \alpha e^{\beta t}
    \end{align*}
    Where $\alpha$ and $\beta$ are our regression parameters.
    
    \section{Parameter Space}
    In our regression we want to minimize the residual
    \begin{align*}
        \chi^2(\alpha,\beta) = \Sigma \frac{(N_i - N_{\alpha,\beta}(t_i))^2}{(\delta N_i)^2}
    \end{align*}
    To do so we have to find good starting parameters $\alpha_0$ and $\beta_0$.
    A reasonable method is to linearize our data and perform a linear regression.
    If we do this for our data set we get the parameters:
    \begin{align*}
        \alpha_0 &= \LinearizedModelA \\
        \beta_0 &= \LinearizedModelB 
    \end{align*}
    We can now plot the function $\chi^2(\alpha,\beta)$ around those starting parameters:
    \begin{figure}[h!]
        \includegraphics{Generated/Fr2 A7 Residual Contour Plot}
        \caption{Residual for a exponential fit  Thomas Karb 06.06.2023}
        \label{fig:RisidualForAExpFit}
    \end{figure}
    \section{Levenberg Marquardt}
    To find the best parameters $\alpha$ and $\beta$ we can use the Levenberg Marquardt algorithm.
    In my implementation I have used these  \href{https://backend.orbit.dtu.dk/ws/portalfiles/portal/2721358/imm3215.pdf}{lecture notes}
    as a reference on how to program the LM-algorithm.
    For the algorithm we need the Jacobi-Matrix which for this simple model can be calculated as
    \begin{align*}
        J_{i1} &= \partial_{\apha} N(t_i) = e^{\beta t_i} \\
        J_{i2} &= \partial_{\beta} N(t_i) = \alpha e^{\beta t_i} \cdot t_i
    \end{align*}
    Furthermore we need our starting guesses $\alpha_0$ and $\beta_0$ which we have calculated in the previous section through linear regression.
    If we then perform the algorithm we get the following parameters as a result:
    \begin{align*}
        \alpha &= \NonlinearModelA \\
        \beta &= \NonlinearModelB
    \end{align*}
    To compare those parameters to the starting guess we can calculate the residual.
    For our parameters derived from the linear regression we have a residual of:
    \begin{align*}
        \chi^2(\alpha_0,\beta_0) = \LinearizedModelReducedResidual
    \end{align*}
    We can compare this to the residual of the LM-algorithm:
    \begin{align*}
        \chi^2(\alpha,\beta) = \NonlinearModelReducedResidual
    \end{align*}
    So indeed we have found a better fit through the non linear regression.
    Now we can calculate the half-life-time of the unknown radioactive substance:
    \begin{align*}
        T = \frac{-ln(2)}{\beta} = \HalfTime
    \end{align*}
    
    If we plot our results we get:
    \figNonLinearRegression
    
    


\end{document}