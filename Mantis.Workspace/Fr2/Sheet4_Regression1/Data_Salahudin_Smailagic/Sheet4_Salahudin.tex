%! Author = User
%! Date = 18.05.2023

% Preamble
\documentclass[11pt]{article}

\usepackage{arabicore}
\usepackage{shellesc}


\usepackage{amsmath}
\usepackage{tikz}
\usepackage{pgfplots}

\newcommand{\alphaNoError}{(4.047 \pm 0.036)}
\newcommand{\betaNoError}{(-4.73 \pm 0.29) \cdot 10^{-3}}
\newcommand{\halfTimeNoError}{(146.5 \pm 9.1)\ s}
\newcommand{\alphaGauss}{(4.04 \pm 0.10)}
\newcommand{\betaGauss}{(-4.62 \pm 0.95) \cdot 10^{-3}}
\newcommand{\halfTimeGauss}{(150 \pm 31)\ s}
\newcommand{\alphaPoisson}{(4.05 \pm 0.10)}
\newcommand{\betaPoisson}{(-4.75 \pm 0.95) \cdot 10^{-3}}
\newcommand{\halfTimePoisson}{(146 \pm 29)\ s}
\newcommand{\symN}{\delta N}


% Document
\begin{document}

    \author{Elias Schwarzkopf}
    \title{Blatt 4- Regression 1}

    \maketitle

    \section{Linearisierung durch Logarithmierung}

   In der Aufgabe soll die Halbwertszeit von Ba-137m bestimmt werden. Als Messwerte liegen uns die Anzahlen der Zerfälle {$\Delta N$} pro Zeitinterwall {$\Delta t$} vor. Diese ist in 20s Blöcke aufgeteilt.
    Die Zerfallszahl wird durch eine Poissonverteilung beschrieben, da die Zerfallswahrscheinlichkeit jedes Atoms gegen Null geht und die Anzahl der Atome gegen unendlich. Der Standardfehler wird als Wurzel des Wertes berechnet.
    \begin{equation*}
        \sigma_{\Delta N} = \sqrt {\Delta N}
    \end{equation*}
    Um die Halbwertszeit  $T_s$ zu ermitteln, wird angenommen, dass  $\Delta t << T_s$ weshalb die Änderung der Zerfallsrate innerhalb einer Klasse vernachlässigbar ist. Es ergibt sich:
    \begin{equation*}
        \Delta N = \Delta N_0 2^{-t / T_s} = \Delta N_0 e^{\beta t}
    \end{equation*}
    Nun können die Daten linearisiert werden, indem der Ln auf beiden Seiten genommen wird. Der Exponent kann dann als Faktor vorgezogen werden. Der Fehler von $ln(\Delta N)$ ergibt sich durch die Taylor-Entwicklung.
    
    \begin{equation*}
        ln(\DElta N) = \ln(\Delta N_0)+\beta t = \alpha + \beta t
    \end{equation*}
    Daraus wird die Zerfallszeit berechnet, mit:
    \begin{equation*}
       T_s =\frac{-\ln(2)}{\beta}
        \end{equation*}
    Berechnet.
    Linearisiert ergibt sich für die Daten:
    \begin{table}[h!]
\centering
\begin{tabular}
{| c | c | c |}
\hline
 Time $t$ in $s$ & Decay Count $\symN$  & $ln(\symN)$  \\ 
\hline
 20 & (43.0 \pm 6.6) & (3.76 \pm 0.15) \\ 
\hline
 40 & (39.0 \pm 6.2) & (3.66 \pm 0.16) \\ 
\hline
 60 & (36.0 \pm 6.0) & (3.58 \pm 0.17) \\ 
\hline
 80 & (34.0 \pm 5.8) & (3.53 \pm 0.17) \\ 
\hline
 100 & (30.0 \pm 5.5) & (3.40 \pm 0.18) \\ 
\hline
 120 & (26.0 \pm 5.1) & (3.26 \pm 0.20) \\ 
\hline
 140 & (25.0 \pm 5.0) & (3.22 \pm 0.20) \\ 
\hline
 160 & (23.0 \pm 4.8) & (3.14 \pm 0.21) \\ 
\hline
 180 & (23.0 \pm 4.8) & (3.14 \pm 0.21) \\ 
\hline
 200 & (16.0 \pm 4.0) & (2.77 \pm 0.25) \\ 
\hline
\end{tabular}
\label{tab:BaDecayData}
\caption{Decay Data of Ba-137}
\end{table}

    \pagebreak
   

    \section{Ungewichtete lineare Regression}
    Mit linearen Regression $y(t) = \alpha + \beta t$ ergibt sich:
    \begin{equation*}
        \alpha = \alphaNoError
    \end{equation*}
    \begin{equation*}
        \beta = \betaNoError
    \end{equation*}
    \begin{equation*}
        Ts =\frac{-\ln(2)}{\beta}= \halfTimeNoError
    \end{equation*}

   

    \section{Lineare Regression - Gewichtung nach Gausfehlern}
    Die lineare Regression wird mit den berechneten Fehlern in y durchgeführt. Die Fehler werden analog zu Blatt 3 berechnet.
    Für $y(t) = \alpha + \beta t$ kriegen wir:
    \begin{equation*}
        \alpha = \alphaGauss
    \end{equation*}
    \begin{equation*}
        \beta = \betaGauss
    \end{equation*}
    \begin{equation*}
        Ts =\frac{-\ln(2)}{\beta}= \halfTimeGauss
    \end{equation*}

    \section{Poissonsche Regression}
    Wie in der Vorlesung beschrieben wird bei der Regression bei Annahme poissonverteilter Unsicherheiten auf ein numerisches Verfahren zurückgegriffen. Dieses liefert wieder für $y(t) = \alpha + \beta t$:
    \begin{equation*}
        \alpha = \alphaPoisson
    \end{equation*}
    \begin{equation*}
        \beta = \betaPoisson
    \end{equation*}
    \begin{equation*}
        Ts =\frac{-\ln(2)}{\beta}= \halfTimePoisson
    \end{equation*}
    Für die Fehler wurde auf ein numerisches Verfahren verzichtet. Die Fehler wurden nach Gaußscher Fehlerrechnung berechnet.
    
    \section{Diskussion}
    Die berechneten Werte für {$T_s$} stimmen innerhalb der Fehlerbalken überein. Da die letzte Methode die genaueste ist, wird der dort berechnete Wert verwendet.
    \begin{equation*}
        Ts = \halfTimePoisson
    \end{equation*}
   Die deutlich aufwändigere Methode zur Berechnung der Poissonschen Regression ist hierbei nicht sinnvoll, da der Mehraufwand nur eine geringe Verbesserung des Ergebnisses bewirkt.
    


\end{document}