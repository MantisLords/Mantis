%! Author = Thomas_normal
%! Date = 16.05.2023

% Preamble
\documentclass[11pt]{article}
\usepackage{hyperref}


\usepackage{amsmath}
\usepackage{tikz}
\usepackage{pgfplots}

\newcommand{\alphaNoError}{(4.047 \pm 0.036)}
\newcommand{\betaNoError}{(-4.73 \pm 0.29) \cdot 10^{-3}}
\newcommand{\halfTimeNoError}{(146.5 \pm 9.1)\ s}
\newcommand{\alphaGauss}{(4.04 \pm 0.10)}
\newcommand{\betaGauss}{(-4.62 \pm 0.95) \cdot 10^{-3}}
\newcommand{\halfTimeGauss}{(150 \pm 31)\ s}
\newcommand{\alphaPoisson}{(4.05 \pm 0.10)}
\newcommand{\betaPoisson}{(-4.75 \pm 0.95) \cdot 10^{-3}}
\newcommand{\halfTimePoisson}{(146 \pm 29)\ s}
\newcommand{\symN}{\delta N}


% Document
\begin{document}

    \author{Thomas Karb}
    \title{Fr 2 - Sheet 4 - Regression 1}
    
    \maketitle
    
    \section{Linearization}
    
    We want to analyse data of a Ba-137m decay.
    We have measured the decay count $\symN$ in a fixed time frame $\Delta t$ depending on the overall passed time $t$.
    The decay count $\symN$ follows a poisson distribution. So we get the standard error by taking the square root:
    \begin{equation*}
        \sigma_{\symN} = \sqrt {\symN}
    \end{equation*}
    To now calculate the half-life-time $T_s$, we assume that $\Delta t << T_s$ so our decay law simplifies to
    \begin{equation*}
        \symN = \symN_0 \ 2^{-t / T_s} = \symN_0 e^{\beta t}
    \end{equation*}
    Now we can linearize the data by taking the natural log on both sides
    \begin{equation*}
        ln(\symN) = \ln(\symN_0)+\beta t = \alpha + \beta t
    \end{equation*}
    The error of $ln(\symN)$ can be calculated through a taylor approximation (identically to gauss).
    By applying these steps we transform our data to:
    
    \begin{table}[h!]
\centering
\begin{tabular}
{| c | c | c |}
\hline
 Time $t$ in $s$ & Decay Count $\symN$  & $ln(\symN)$  \\ 
\hline
 20 & (43.0 \pm 6.6) & (3.76 \pm 0.15) \\ 
\hline
 40 & (39.0 \pm 6.2) & (3.66 \pm 0.16) \\ 
\hline
 60 & (36.0 \pm 6.0) & (3.58 \pm 0.17) \\ 
\hline
 80 & (34.0 \pm 5.8) & (3.53 \pm 0.17) \\ 
\hline
 100 & (30.0 \pm 5.5) & (3.40 \pm 0.18) \\ 
\hline
 120 & (26.0 \pm 5.1) & (3.26 \pm 0.20) \\ 
\hline
 140 & (25.0 \pm 5.0) & (3.22 \pm 0.20) \\ 
\hline
 160 & (23.0 \pm 4.8) & (3.14 \pm 0.21) \\ 
\hline
 180 & (23.0 \pm 4.8) & (3.14 \pm 0.21) \\ 
\hline
 200 & (16.0 \pm 4.0) & (2.77 \pm 0.25) \\ 
\hline
\end{tabular}
\label{tab:BaDecayData}
\caption{Decay Data of Ba-137}
\end{table}

    
    \pagebreak
    
    \section{Unweighted Gaussian Regression}
    To get the half-life-time $T_s$ we need to do a linear regression $y(t) = \alpha + \beta t$. First we make a gaussian
    fit and ignore the errors like in the previous worksheet.
    For the data of Table~\ref{tab:BaDecayData} we get the following regression parameters
    \begin{equation*}
        \alpha = \NoErrorsA
    \end{equation*}
    \begin{equation*}
        \beta = \NoErrorsB
    \end{equation*}
    From the slope $\beta$ we can calculate the half-life-time (with gaussian error propagation):
    \begin{equation*}
        T_s = \frac{ln(2)}{\beta} = \NoErrorsHalfTime
    \end{equation*}
    
    If we plot the data and the fit it looks like the following:
    
    \figBaDecay
    
    \pagebreak

    \section{Gaussian Regression with y-Errors}
    Similarly we can apply a gaussian regression but now with the y-error of $ln(\symN)$ (Compare with lecture notes).
    For our decay data we get the following coefficients:
    \begin{equation*}
        \alpha = \GaussA
    \end{equation*}
    \begin{equation*}
        \beta = \GaussB
    \end{equation*}
    This means a half-life-time of:
    \begin{equation*}
        T_s = \GaussHalfTime
    \end{equation*}

    \section{Poisson Regression}
    In the last part we now take into account that our data is not actually normal distributed, as assumed in the 
    gaussian regression.
    Actually our data follows a poisson distribution.
    So if we now calculate the coefficients we must find the roots of the following function:
    \begin{align*}
       \vec{f}(\alpha,\beta) = 
        \begin{pmatrix}
            \Sigma \frac{y_i}{\alpha + \beta t_i} - N \\
            \Sigma \frac{y_i x_i}{\alpha + \beta t_i} - \Sigma x_i
        \end{pmatrix}
    \end{align*}
    We use a \href{https://en.wikipedia.org/wiki/Broyden%27s_method}{Broyden's method} to find the roots af this two
    dimensional function. 
    Since the order of the $\beta$-error of the previous to methods is the same, it is
    reasonably to assume that the $\beta$-error of the poisson regression has also the same order.
    So for the sake of economy we will use the $\beta$-error of the second method for the poisson regression.
    This leaves us with the following coefficients:
    \begin{equation*}
        \alpha = \PoissonA
    \end{equation*}
    \begin{equation*}
        \beta = \PoissonB
    \end{equation*}
    And we get the half-life-time:
    \begin{equation*}
        T_s = \PoissonHalfTime
    \end{equation*}
    
    All three processes are examples for linear regression.
    All of them need the linearization of the data.
    Out of the three the first one is definitively the worst, since it is ignoring data (the y-errors).
    This explains why the $\beta$-error is smaller than compared to the method where we take the y-errors into account.
    The second is more proficient.
    It is quite fast because it boils down to solving a linear equation.
    But it does not model our data correctly.
    It assumes the data points are normal distributed.
    The third method actually models the poisson distribution but at the price, that it is not anymore analytically 
    solvable.
    All three results are compatible in respects to their error bars.
    Since the last method has the most accurate model I choose it as the final result and so the measured half-life-time
    of Ba-137 is
    \begin{equation*}
        T_s = \PoissonHalfTime
    \end{equation*}

\end{document}