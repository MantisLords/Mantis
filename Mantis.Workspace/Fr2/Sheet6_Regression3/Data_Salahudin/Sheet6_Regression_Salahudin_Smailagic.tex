%! Author = User
%! Date = 02.06.2023

% Preamble
\documentclass[11pt]{article}


\usepackage{amsmath}
\usepackage{tikz}
\usepackage{pgfplots}

\newcommand{\TemperatureHumidityCovariance}{102.3\ ^{\circ} C g / m^3}
\newcommand{\TemperatureHumidityCorrelation}{0.8571}
\newcommand{\RegressionOffset}{(1.8 \pm 1.2)\  g / m^3}
\newcommand{\RegressionSlope}{(0.343 \pm 0.057)\ g / m^3 / ^{\circ} C}


\title{Blatt 6}
\author{Salahudin Smailagić}

% Document
    \begin{document}

        \maketitle

        \section{Streuexperiment}
        In diesem Experiment wurde flüssiger Wasserstoff mit Antiprotonen beschossen und die Winkelverteilung der gestreuten Antiprotonen gemessen.
        Es wird angenommen, dass die erfassten zählereignisse Poissonverteilt sind und somit der Standardfehler die Wurzel aus dem Mittelwert ist.
        Um die Daten zu bereinigen müssen wir noch die Hintergrundstreuung substituieren. 
        
        \begin{align*}
            N = N_{voll} - 2 N_{leer}
        \end{align*}
        Daraus bekommen wir folgende Daten.

        \include{Generated/tab_HScatteringData}
        Nun müssen wir ein Legendre-Polynom 2-ten Grades in die Daten fitten:
        
        \begin{align*}
            N(\cos{\phi}) = a_0 P_0 (\cos(\phi)) + a_1 P_1 (\cos(\phi)) + a_2 P_2 (\cos(\phi))
        \end{align*}

        Wenn wir nun die lineare Regression wie schon In Blatt 5 allgemein beschrieben machen bekommen wir die Regressionsparameter:
        \begin{align*}
            a_0 &= \regressionParameterA \\
            a_1 &= \regressionParameterB \\
            a_2 &= \regressionParameterC
        \end{align*}

        %\include{Generated/tab_HScatteringData}

        \section{Güte der Anpassung des Modells an die Daten}

        Das reduzierte Chi-Quadrat der Verteilung ergibt sich zu:
        \begin{align*}
            \chi^2 &= \ReducedChiSquared
        \end{align*}

        Da wir $\DegreesOfFreedom$ Freiheitsgrade haben entspricht das Chi-Quadrat der Wahrschienlichkeit von:
        \begin{align*}
            p &= \ProbabilityChiSquared
        \end{align*}

        Die Wahrscheinlichkeit sugerriert, dass der Fit gut ist.

        \section{Graphische Darstellung}

       Nun werden die Daten zusammen mit der bestimmten Regression graphisch dargestellt:

        \include{Generated/fig_HScattering}


    \end{document}