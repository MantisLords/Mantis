%! Author = Thomas_normal
%! Date = 01.06.2023

% Preamble
\documentclass[11pt]{article}


\usepackage{amsmath}
\usepackage{tikz}
\usepackage{pgfplots}

\newcommand{\TemperatureHumidityCovariance}{102.3\ ^{\circ} C g / m^3}
\newcommand{\TemperatureHumidityCorrelation}{0.8571}
\newcommand{\RegressionOffset}{(1.8 \pm 1.2)\  g / m^3}
\newcommand{\RegressionSlope}{(0.343 \pm 0.057)\ g / m^3 / ^{\circ} C}


\title{Fr2 - Sheet 6 - Regression 3}
\author{Thomas Karb}

% Document
\begin{document}

    \maketitle
    
    \section{Scattering experiment}
    
    In the experiment hydrogen atoms were shoot at with antiprotons.
    Depending on the angle the count of the scattered protons were measured.
    It is assumed that the scattering count is poisson distributed, so the standard error
    is just the square root of the mean. 
    Furthermore to calculate the cleansed scattering count we have to subtract the background scattering.
    \begin{align*}
        N = N_{full} - 2 N_{empty}
    \end{align*}
    If we do this we get the following data:
    
    \include{Generated/tab_HScatteringData}
    
    Now we have to fit a second order legendre polynomial on the data:
    \begin{align*}
        N(\cos{\phi}) = a_0 P_0 (\cos(\phi)) + a_1 P_1 (\cos(\phi)) + a_2 P_2 (\cos(\phi))
    \end{align*}
    
    If we do the linear regression as described in Sheet 5 we get the following parameters:
    \begin{align*}
        a_0 &= \regressionParameterA \\
        a_1 &= \regressionParameterB \\
        a_2 &= \regressionParameterC
    \end{align*}
    
    %\include{Generated/tab_HScatteringData}
    
    \section{Quality of the fit}
    
    If we calculate the reduced chi-squared of the regression we get
    \begin{align*}
        \chi^2 &= \ReducedChiSquared
    \end{align*}
    
    Since we have $\DegreesOfFreedom$ degrees of freedom the calculated chi-squared 
    corresponds to a probability of
    \begin{align*}
        p &= \ProbabilityChiSquared
    \end{align*}
    
    This is a reasonable probability so it suggest we have a good fit. 
    
    \section{Graphic}
    
    If we plot the regression we get:
    
    \include{Generated/fig_HScattering}


\end{document}