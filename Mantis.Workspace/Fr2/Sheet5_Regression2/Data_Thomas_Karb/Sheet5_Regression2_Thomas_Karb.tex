%! Author = Thomas_normal
%! Date = 18.05.2023

% Preamble
\documentclass[11pt]{article}
\usepackage{mathtools}


\usepackage{amsmath}
\usepackage{tikz}
\usepackage{pgfplots}

\newcommand{\TemperatureHumidityCovariance}{102.3\ ^{\circ} C g / m^3}
\newcommand{\TemperatureHumidityCorrelation}{0.8571}
\newcommand{\RegressionOffset}{(1.8 \pm 1.2)\  g / m^3}
\newcommand{\RegressionSlope}{(0.343 \pm 0.057)\ g / m^3 / ^{\circ} C}


\pgfplotsset{compat=1.18}

% Document
\begin{document}
    
    \author{Thomas Karb}
    \title{Fr2 - Sheet 5 - Regression 2}
    
    \maketitle

    \section{Gaussian Error of general linear regression}
    
    Let's say we have a model with $k$ linear coefficients $a_i$:
    
    \newcommand{\funcTrans}{\vec{f}^{T}}
    \newcommand{\aVec}{\vec{a}}
    \begin{align*}
        y(x) &= a_1 \ f_1(x) + a_2 \ f_2(x) + \dots + a_k \ f_k(x) \\
        &= \begin{pmatrix}
               f_1(x) & f_2(x) & \dots & f_k(x)
        \end{pmatrix} \cdot \begin{pmatrix}
                            a_1 \\
                            a_2 \\
                            \vdots \\
                            a_k
        \end{pmatrix} \\
        &= \funcTrans(x) \cdot \aVec
    \end{align*}
    
    Now we have measured $n$ data points $x_i$ and $y_i$.
    Where the $y_i$ have the standard error $\sigma_i$.
    We can define the matrix
    \newcommand{\yVec}{\vec{y}}
    \newcommand{\xMatrix}{\boldsymbol{X}}
    \newcommand{\xMatrixT}{\boldsymbol{X}^T}
    \begin{align*}
        \xMatrix &= \begin{pmatrix}
                        \funcTrans(x_1) \\
                        \funcTrans(x_2) \\
                        \vdots \\
                        \funcTrans(x_n)
        \end{pmatrix}
    \end{align*}
    
    And we can define an error matrix
    \newcommand{\eMatrix}{\boldsymbol{\sigma}_{y}}
    \newcommand{\eMatrixInv}{\boldsymbol{W}}
    \begin{align*}
        \eMatrix = \begin{pmatrix}
                       \sigma_{y_1}^2 \\
                       & \sigma_{y_2}^2\\
                       & & \ddots\\
                       & & & \sigma_{y_n}^2
        \end{pmatrix}
    \end{align*}
    We get the weight matrix as the inverse of the error matrix:
    \begin{align*}
        \eMatrixInv = \eMatrix^{-1}
    \end{align*}
    
    If we do not have any y-errors $\sigma_{y_i}$ as in section 2, the weight matrix is just the identity
    \begin{align*}
        \eMatrixInv = \mathds{1}
    \end{align*}
    
    With those matrices we can calculate our Chi-Squared:
    \begin{align*}
        \chi^2 = (\xMatrix \aVec - \yVec)^T \ \eMatrixInv \ (\xMatrix \aVec - \yVec)
    \end{align*}
    
    We have to minimize $\chi^2$ in order to get our regression parameters $\aVec$.
    We need to take the derivative:
    \begin{align*}
        \vec{0} &= \nabla_{\aVec} \ \chi^2 = 2 \xMatrixT \eMatrixInv (\xMatrix \aVec - \yVec)
    \end{align*}
    
    Which gives us a system of linear equations:
    \begin{align*}
        \xMatrixT \eMatrixInv \yVec &= \xMatrixT \eMatrixInv \xMatrix \aVec
    \end{align*}
    
    We can solve this by calculating the inverse matrix:
    \newcommand{\mMatrix}{\boldsymbol{M}}
    \begin{align*}
        \aVec &= (\xMatrixT \eMatrixInv \xMatrix)^{-1} \xMatrixT \eMatrixInv \yVec \coloneqq \mMatrix \yVec
    \end{align*}
    Note that it would be considerably faster to solve this system of linear equations directly instead of calculating
    the inverse matrix.
    But we need the inverse matrix to calculate the errors of $\aVec$ in the next step.
    For that we have to apply gaussian error calculation.
    We need to get the derivative (Jacobian-matrix) of $\aVec$ in respect to $\yVec$:
    
    \newcommand{\jacMatrix}{\boldsymbol{D}_{\yVec} \ \aVec}
    
    \begin{align*}
        \jacMatrix &= \jacMatrix \ \mMatrix \yVec = \mMatrix
    \end{align*}
    Now we get the error matrix of $\aVec$
    \newcommand{\eaMatrix}{\boldsymbol{\sigma}_{a}}
    \begin{align*}
        \eaMatrix &\coloneqq \jacMatrix \eMatrix (\jacMatrix)^T = \mMatrix \eMatrix \mMatrix^T
    \end{align*}
    The actual errors of $\aVec$ are on the diagonal of the error matrix:
    \begin{align*}
        \sigma_{a_i} = \sqrt {(\eaMatrix)_{i,i}}
    \end{align*}
    Again if we do not have any y-errors $\sigma_{y_i}$ we have to use the sample variance of the data set.
    We can calculate it via the dot product:
    \begin{align*}
        s^2 = \frac{1}{n - k} (\yVec - \xMatrix \aVec)^2
    \end{align*}
    The error matrix in the above formula gets substituted by the identity times the sample variance:
    \begin{align*}
        \eaMatrix = \mMatrix (s^2 \mathds{1}) \mMatrix^T
    \end{align*}
    This is the general procedure for calculating the error of a generic linear regression. 
    In the case of the quadratic polynomial we just have to insert the correct model:
    \begin{align*}
        y(x) &= a_1 + a_2 x + a_k x^2 \\
        &= \begin{pmatrix}
               1 & x & x^2
        \end{pmatrix} \cdot \aVec \\
        &= \funcTrans(x) \cdot \aVec
    \end{align*}
    
    \section{Getting the g-Factor via quadratic fit}

    We have a metal sheet with holes drilled into it.
    The holes have a distance of $\delta x = \holeDistance$.
    (As the error we take the last digit.
    Since this is our biggest inaccuracy.)
    We let the metal sheet fall and measure the time each hole passes a laser.
    With that we get the distance travelled $x$.
    
    \input{Generated/tab_MetalSheetFallData}
    \pagebreak
    
    
    There is a quadratic dependency between the distance $x$ and the time $t$.
    But we can not linearize the data since there is also a liner-coefficient $\beta$.
    We have to use the model:
    \begin{align*}
        x(t) &= \alpha + \beta t + \gamma t^2
    \end{align*}
    We can now use the equations from section 1 to do the regression:
    \begin{align*}
        \alpha &= \alphaPoly \\
        \beta &= \betaPoly \\
        \gamma &= \gammaPoly
    \end{align*}
    
    From the last coefficient we can derive the g-Factor simply by multiplying 2:
    \begin{align*}
        g &= 2 \gamma = \gFactor
    \end{align*}
    
    \section{Plot}
    If we plot the data from Table~\ref{tab:MetalSheetFallData} and combine it with the quadratic regression we get:
    
    \input{Generated/fig_MetalSheetFallGraph}


\end{document}