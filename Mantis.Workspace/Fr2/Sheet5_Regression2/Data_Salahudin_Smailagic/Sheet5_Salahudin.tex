%! Author = Salahudin Smailagić
%! Date = 18.05.2023

% Preamble
\documentclass[11pt]{article}
\usepackage{mathtools}
\usepackage{amsmath}


\usepackage{amsmath}
\usepackage{tikz}
\usepackage{pgfplots}

\newcommand{\alphaNoError}{(4.047 \pm 0.036)}
\newcommand{\betaNoError}{(-4.73 \pm 0.29) \cdot 10^{-3}}
\newcommand{\halfTimeNoError}{(146.5 \pm 9.1)\ s}
\newcommand{\alphaGauss}{(4.04 \pm 0.10)}
\newcommand{\betaGauss}{(-4.62 \pm 0.95) \cdot 10^{-3}}
\newcommand{\halfTimeGauss}{(150 \pm 31)\ s}
\newcommand{\alphaPoisson}{(4.05 \pm 0.10)}
\newcommand{\betaPoisson}{(-4.75 \pm 0.95) \cdot 10^{-3}}
\newcommand{\halfTimePoisson}{(146 \pm 29)\ s}
\newcommand{\symN}{\delta N}


\pgfplotsset{compat=1.18}

% Document
\begin{document}

    \author{Salahudin Smailagić}
    \title{Blatt 5}

    \maketitle

    \section{Gausscher Fehler der Regressionsparameter bei linearer Regression}
    
    Angenommen wir haben ein Modell mit k linearen Koeffizienten $a_i$:
    \newcommand{\funcTrans}{\vec{f}^{T}}
    \newcommand{\aVec}{\vec{a}}
    \begin{align*}
        y(x) &= a_1 \ f_1(x) + a_2 \ f_2(x) + \dots + a_k \ f_k(x) \\
        &= \begin{pmatrix}
               f_1(x) & f_2(x) & \dots & f_k(x)
        \end{pmatrix} \cdot \begin{pmatrix}
                                a_1 \\
                                a_2 \\
                                \vdots \\
                                a_k
        \end{pmatrix} \\
        &= \funcTrans(x) \cdot \aVec
    \end{align*}
    
    Nun messen wir n Datenpunkte $x_i$ und $y_i$. Die $y_i$ sind fehlerbehaftet mit den entsprechenden Fehlern $\Sigma_i$.
    Wir definieren die Matrix:
    \newcommand{\yVec}{\vec{y}}
    \newcommand{\xMatrix}{\boldsymbol{X}}
    \newcommand{\xMatrixT}{\boldsymbol{X}^T}
    \begin{align*}
        \xMatrix &= \begin{pmatrix}
                        \funcTrans(x_1) \\
                        \funcTrans(x_2) \\
                        \vdots \\
                        \funcTrans(x_n)
        \end{pmatrix}
    \end{align*}
    
    Und die Fehlermatrix:
    \newcommand{\eMatrix}{\boldsymbol{\sigma}_{y}}
    \newcommand{\eMatrixInv}{\boldsymbol{W}}
    \begin{align*}
        \eMatrix = \begin{pmatrix}
                       \sigma_{y_1}^2 \\
                       & \sigma_{y_2}^2\\
                       & & \ddots\\
                       & & & \sigma_{y_n}^2
        \end{pmatrix}
    \end{align*}
    
    Die Gewichtungsmatrix folgt direkt daraus und zwar:
    \begin{align*}
        \eMatrixInv = \eMatrix^{-1}
    \end{align*}
    
    Das gesuchte Chi-Quadrat lässt sich nun aus diesen Matrizen berechnen:
    \begin{align*}
        \chi^2 = (\xMatrix \aVec - \yVec)^T \ \eMatrixInv \ (\xMatrix \aVec - \yVec)
    \end{align*}
    Analog zu unserem bisherigem Vorgehen bei Regression, müssen wir nun das $\Chi^2$ minimieren um unsere Regressionsparameter zu bekommen.
    Dazu leiten wir das $\Chi^2$ nach den gesuchten Koeffizienten ab:
    \begin{align*}
        \vec{0} &= \nabla_{\aVec} \ \chi^2 = 2 \xMatrixT \eMatrixInv (\xMatrix \aVec - \yVec)
    \end{align*}
    Das daraus folgende Gleichungssystem sieht so aus:
    \begin{align*}
        \xMatrixT \eMatrixInv \yVec &= \xMatrixT \eMatrixInv \xMatrix \aVec
    \end{align*}
    Dieses wird gelöst in dem man die inverse Matrix berechnet wie folgt:
    \newcommand{\mMatrix}{\boldsymbol{J}}
    \begin{align*}
        \aVec &= (\xMatrixT \eMatrixInv \xMatrix)^{-1} \xMatrixT \eMatrixInv \yVec \coloneqq \mMatrix \yVec
    \end{align*}
    Um nun die Fehler in $\aVec$ zu bestimmen wird gaussche Fehlerrechnung benutzt. Wir brauchen die Ableitung bzw. in diesem Falle die
    Jacobi-Matrix von $\aVec$ bezüglich $\yVec$:
    \newcommand{\jacMatrix}{\boldsymbol{D}_{\yVec} \ \aVec}

    \begin{align*}
        \jacMatrix &= \jacMatrix \ \mMatrix \yVec = \mMatrix
    \end{align*}
    Die Fehlermatrix von $\aVec$ sieht dann wie folgt aus, wobei die Fehler von $\aVec$ auf der Hauptdiagonale sind:
    \newcommand{\eaMatrix}{\boldsymbol{\sigma}_{a}}
    \begin{align*}
        \eaMatrix &\coloneqq (\jacMatrix)^T \eMatrix \jacMatrix = \mMatrix^T \eMatrix \mMatrix
    \end{align*}
    \begin{align*}
        \sigma_{a_i} = \sqrt {(\eaMatrix)_{i,i}}
    \end{align*}
    
    Dieser ist ein allgemeiner Vorgang zur bestimmung der Fehler bei linearer Regression, für das den Spezialfall $k=3$ setze man
    nur das richtige Modell ein:
    \begin{align*}
        y(x) &= a_1 + a_2 x + a_k x^2 \\
        &= \begin{pmatrix}
               1 & x & x^2
        \end{pmatrix} \cdot \aVec \\
        &= \funcTrans(x) \cdot \aVec
    \end{align*}
    
    \section{Bestimmung der Erbeschleunigung mittels quadratischem Fit}
    
    In der Aufgabe haben wir ein Lochblech mit regelmäßigen Bohrungen im Abstand von 2 cm. Das Lochblech wird fallen gelassen und es wird die Zeit gemessen
    bei der jede Bohrung eine Lichtschranke passiert. Daraus berechnet sich die länge x. Also $\Delta x = \holeDistance$. Daraus ergibt sich die Tabelle:
    
    \input{Generated/tab_MetalSheetFallData}
    
    Die quadratische Abhängigkeit der Länge x und der Zeit t ist offensichtlich, trotzdem können die Daten nicht linearisiert werden, da auch
    der lineare koeffizient $\beta$ dabei ist. Das Modell sieht folgendermaßen aus:
    \begin{align*}
        x(t) &= \alpha + \beta t + \gamma t^2
    \end{align*}
    Nun nutzen wir den in A beschriebenen Vorgang um die Koeffizenten samt Fehler zu bestimmen:
    \begin{align*}
        \alpha &= \alphaPoly \\
        \beta &= \betaPoly \\
        \gamma &= \gammaPoly
    \end{align*}
    Die Fehler der Koeffizenten ergeben sich aus der in der Vorlesung beschriebenen Methode, und zwar wird erst der Fit an die vorhandenen Daten angepasst. Die Daten werden dabei als 
    fehlerfrei angenommen. Dann wird die mittlere quadratische Abweichung der gemessenen Werte $x_i$ von den gefitteten Funktionenwerten $y(x_i)$ berechnet wie folgt:
    \begin{equation*}
        s = \sqrt[2]{\frac{1}{n-3}\\\sum_{1}^{n}(y_i - y(x_i))^2}
    \end{equation*}
    dieses so berechnete s wird nun als y-Fehler der Datenpunkte angenommen woraus sich per Gausscher Fehlerrechnung die Fehler der Regressionsparameter ergeben.
    Die Erbeschleunigung ergibt sich aus dem dritten Koeffizienten mit einem Faktor 2:
    \begin{align*}
        g &= 2 \gamma = \gFactor
    \end{align*}
    \section{Graphische Darstellung der Daten}
    Wenn wir nun die Daten aus Table~\ref{tab:MetalSheetFallData} graphisch auftragen und mit dem Fit der sich aus den berechneten Regressionsparametern
    ergibt überlagern sehen wir:

    \input{Generated/fig_MetalSheetFallGraph}
    
    







\end{document}